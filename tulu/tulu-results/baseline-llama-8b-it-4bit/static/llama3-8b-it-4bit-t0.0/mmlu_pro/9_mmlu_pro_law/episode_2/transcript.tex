
\documentclass{article}
\usepackage{colortbl}
\usepackage{makecell}
\usepackage{multirow}
\usepackage{supertabular}

\begin{document}

\newcounter{utterance}

\twocolumn

{ \footnotesize  \setcounter{utterance}{1}
\setlength{\tabcolsep}{0pt}
\begin{supertabular}{c@{$\;$}|p{.15\linewidth}@{}p{.15\linewidth}p{.15\linewidth}p{.15\linewidth}p{.15\linewidth}p{.15\linewidth}}

    \# & $\;$A & \multicolumn{4}{c}{Game Master} & $\;\:$B\\
    \hline 

    \theutterance \stepcounter{utterance}  

    & & \multicolumn{4}{p{0.6\linewidth}}{\cellcolor[rgb]{0.9,0.9,0.9}{%
	\makecell[{{p{\linewidth}}}]{% 
	  \tt {\tiny [A$\langle$GM]}  
	 The following are multiple choice questions (with answers) about law. Answer only with "The answer is (X)" where X is the correct letter choice.\\ \tt \\ \tt Question:\\ \tt What is the judge ad hoc?\\ \tt Options:\\ \tt A. Judge ad hoc is the president of the ICJ\\ \tt B. Judge ad hoc is a temporary judge appointed for a specific period of time\\ \tt C. Judge ad hoc is the judge that each party will always nominate in every contentious case\\ \tt D. Judge ad hoc is the member of the bench of the ICJ with a casting vote\\ \tt E. Judge ad hoc is a judge who is nominated by the parties involved in a contentious case, irrespective of their nationality\\ \tt F. Judge ad hoc is a judge who decides on the admissibility of cases before the ICJ\\ \tt G. Judge ad hoc is a judge appointed by the Security Council of the United Nations\\ \tt H. Judge ad hoc is a surrogate judge, in case a judge is disqualified or passes away\\ \tt I. If a party to a contentious case before the ICJ does not have a national sitting as judge, it is entitled to nominate someone as a judge solely for that case, with the title of judge ad hoc\\ \tt J. N/A\\ \tt The answer is (I)\\ \tt \\ \tt Question:\\ \tt Functions of the law include all but which of the following?\\ \tt Options:\\ \tt A. defining the limits of government power\\ \tt B. regulating the use of public spaces\\ \tt C. keeping the peace\\ \tt D. maximizing individual freedom\\ \tt E. maintaining order and stability\\ \tt F. preventing environmental degradation\\ \tt G. providing a basis for compromise\\ \tt H. promoting social justice\\ \tt I. promoting the principles of the free enterprise system\\ \tt J. encouraging economic growth\\ \tt The answer is (I)\\ \tt \\ \tt Question:\\ \tt The ________ School of jurisprudence postulates that the law is based on what is "correct."\\ \tt Options:\\ \tt A. Legal Pragmatism\\ \tt B. Legal Formalism\\ \tt C. Comparative\\ \tt D. Analytical\\ \tt E. Sociological\\ \tt F. Historical\\ \tt G. Critical Legal Studies\\ \tt H. Realist\\ \tt I. Positivist\\ \tt J. Natural Law\\ \tt The answer is (J)\\ \tt \\ \tt Question:\\ \tt Which word best summarizes Weber's explanation of the development of formally rational law?\\ \tt Options:\\ \tt A. Socialism.\\ \tt B. Legitimacy.\\ \tt C. Authority.\\ \tt D. Democracy.\\ \tt E. Bureaucracy.\\ \tt F. Conflict.\\ \tt G. Capitalism.\\ \tt H. Charisma.\\ \tt I. Co-operation.\\ \tt J. Tradition.\\ \tt The answer is (G)\\ \tt \\ \tt Question:\\ \tt A state has recently enacted a statute prohibiting the disposal of any nuclear wastes within the state. This law does not contravene or conflict with any federal statutes. A man operates a company in the state that is engaged in the disposal of nuclear wastes. Subsequent to the passage of the state statute, the man, not yet aware of the new law, entered into contracts with many out-of-state firms to dispose of their nuclear wastes in the state. On account of this new law, however, the man will be unable to perform these contracts. Assume that the man has standing to challenge this state law. Which of the following presents his strongest constitutional grounds to challenge the state law prohibiting the disposal of nuclear wastes within the state?\\ \tt Options:\\ \tt A. The second amendment - the right to bear arms.\\ \tt B. The due process clause of the Fourteenth Amendment.\\ \tt C. The tenth amendment - powers not delegated to the United States by the Constitution.\\ \tt D. The first amendment - freedom of speech.\\ \tt E. The privileges and immunities clause of Article IV, Section 2.\\ \tt F. The commerce clause.\\ \tt G. The sixth amendment - right to a fair trial.\\ \tt H. The eighth amendment - prohibition of cruel and unusual punishment.\\ \tt I. The equal protection clause of the Fourteenth Amendment.\\ \tt J. N/A\\ \tt The answer is (F)\\ \tt \\ \tt Question:\\ \tt Each year the state provides a number of non- interest-bearing loans andlor scholarships to candidates for the degree of L. L. B or J. D. at the state's law school. The applicable state statute limits eligibility to citizens of the United States and aliens seeking U. S. citizenship. A candidate for a J. D. degree at the state law school applied for one of the non-interest-bearing loans. Since he did not intend to seek U. S. citizenship, he was refused the loan for ineligibility under the state statute. In a suit by the candidate challenging the constitutionality of the state statute, he will\\ \tt Options:\\ \tt A. lose, because the state has the right to limit eligibility for its programs.\\ \tt B. lose, because the statute promotes a compelling state interest.\\ \tt C. lose, because the statute is in accordance with the state's education policies.\\ \tt D. lose, because alienage classifications are not, per Se, unconstitutional under the Fourteenth Amendment's equal protection clause.\\ \tt E. win, because the statute infringes on the candidate's right to financial aid.\\ \tt F. win, because classifications by a state that are based on alienage are inherently suspect and subject to close judicial scrutiny.\\ \tt G. lose, because classifications by a state based on citizenship status are constitutional.\\ \tt H. win, because the statute is violative of the privileges or immunities clause of the Fourteenth Amendment.\\ \tt I. win, because the statute discriminates against those who choose not to pursue U.S. citizenship.\\ \tt J. win, because the statute violates the candidate's right to equal treatment under the law.\\ \tt  
	  } 
	   } 
	   } 
	 & \\ 
 

    \theutterance \stepcounter{utterance}  

    & \multicolumn{4}{p{0.6\linewidth}}{\cellcolor[rgb]{0.8,1,0.9}{%
	\makecell[{{p{\linewidth}}}]{% 
	  \tt {\tiny [A$\rangle$GM]}  
	 The answer is (F) 
	  } 
	   } 
	   } 
	 & & \\ 
 

    \theutterance \stepcounter{utterance}  

    & & & \multicolumn{2}{p{0.3\linewidth}}{\cellcolor[rgb]{0.95,0.95,0.95}{%
	\makecell[{{p{\linewidth}}}]{% 
	  \tt {\tiny [GM$|$GM]}  
	 F 
	  } 
	   } 
	   } 
	 & & \\ 
 

    \theutterance \stepcounter{utterance}  

    & & & \multicolumn{2}{p{0.3\linewidth}}{\cellcolor[rgb]{0.95,0.95,0.95}{%
	\makecell[{{p{\linewidth}}}]{% 
	  \tt {\tiny [GM$|$GM]}  
	 F 
	  } 
	   } 
	   } 
	 & & \\ 
 

    \theutterance \stepcounter{utterance}  

    & & & \multicolumn{2}{p{0.3\linewidth}}{\cellcolor[rgb]{0.95,0.95,0.95}{%
	\makecell[{{p{\linewidth}}}]{% 
	  \tt {\tiny [GM$|$GM]}  
	 game_result = WIN 
	  } 
	   } 
	   } 
	 & & \\ 
 

\end{supertabular}
}

\end{document}
