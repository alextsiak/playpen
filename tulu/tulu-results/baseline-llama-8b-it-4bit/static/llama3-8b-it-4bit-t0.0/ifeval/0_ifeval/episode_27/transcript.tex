
\documentclass{article}
\usepackage{colortbl}
\usepackage{makecell}
\usepackage{multirow}
\usepackage{supertabular}

\begin{document}

\newcounter{utterance}

\twocolumn

{ \footnotesize  \setcounter{utterance}{1}
\setlength{\tabcolsep}{0pt}
\begin{supertabular}{c@{$\;$}|p{.15\linewidth}@{}p{.15\linewidth}p{.15\linewidth}p{.15\linewidth}p{.15\linewidth}p{.15\linewidth}}

    \# & $\;$A & \multicolumn{4}{c}{Game Master} & $\;\:$B\\
    \hline 

    \theutterance \stepcounter{utterance}  

    & & \multicolumn{4}{p{0.6\linewidth}}{\cellcolor[rgb]{0.9,0.9,0.9}{%
	\makecell[{{p{\linewidth}}}]{% 
	  \tt {\tiny [A$\langle$GM]}  
	 Write a casual blog post about how the outer solar system is different from the inner solar system, and what that means for the possibility of life. Wrap your entire response with double quotation marks. Your response should contain 17 or more sentences. 
	  } 
	   } 
	   } 
	 & \\ 
 

    \theutterance \stepcounter{utterance}  

    & \multicolumn{4}{p{0.6\linewidth}}{\cellcolor[rgb]{0.8,1,0.9}{%
	\makecell[{{p{\linewidth}}}]{% 
	  \tt {\tiny [A$\rangle$GM]}  
	 "Exploring the Solar System's Diverse Regions: What it Means for Life Beyond Earth\\ \tt \\ \tt As we continue to explore our cosmic neighborhood, it's becoming increasingly clear that the outer solar system is vastly different from the inner solar system. This dichotomy has significant implications for the possibility of life beyond Earth. Let's dive into the key differences and what they might mean for the search for extraterrestrial life.\\ \tt \\ \tt The inner solar system, comprising the planets Mercury, Venus, Earth, and Mars, is characterized by rocky surfaces, relatively small sizes, and a strong gravitational pull from the Sun. In contrast, the outer solar system, which includes the gas giants Jupiter and Saturn, the ice giants Uranus and Neptune, and the dwarf planet Pluto, is marked by massive sizes, gaseous compositions, and much weaker gravitational forces.\\ \tt \\ \tt One of the most striking differences between the inner and outer solar system is the presence of moons. The gas giants, in particular, have numerous moons, some of which are thought to be capable of supporting life. For example, Jupiter's moon Europa has a thick icy crust covering a liquid water ocean, which is believed to be a promising environment for life. Similarly, Saturn's moon Enceladus has geysers of water vapor and organic compounds, suggesting that it may also harbor life.\\ \tt \\ \tt The outer solar system's gas giants are also thought to have formed in a region of the solar system where temperatures were too low for rocky planets to condense. This led to the formation of massive, gaseous worlds that are now the dominant features of the outer solar system. In contrast, the inner solar system's rocky planets formed in a region where temperatures were high enough for rocky material to condense, resulting in the formation of smaller, more terrestrial worlds.\\ \tt \\ \tt The differences between the inner and outer solar system also have implications for the search for extraterrestrial life. The outer solar system's gas giants and their moons offer a unique opportunity to search for life in environments that are distinct from those found in the inner solar system. For example, the presence of liquid water and organic compounds on Europa and Enceladus suggests that these moons may be capable of supporting life, even if it's in the form of microbial organisms.\\ \tt \\ \tt In addition, the outer solar system's gas giants are thought to have played a crucial role in the delivery of water and organic compounds to the inner solar system. This process, known as the "water delivery hypothesis," suggests that the gas giants may have provided the necessary ingredients for life to emerge on Earth. While this hypothesis is still speculative, it highlights the importance of the outer solar system in shaping the conditions for life in the inner solar system.\\ \tt \\ \tt In conclusion, the outer solar system is a vastly different region from the inner solar system, with its own unique characteristics and features. The presence of moons, the formation of gas giants, and the delivery of water and organic compounds all contribute to the possibility of life beyond Earth. As we continue to explore our cosmic neighborhood, it's clear that the outer solar system holds many secrets and surprises, and the search for extraterrestrial life is an exciting and ongoing adventure." 
	  } 
	   } 
	   } 
	 & & \\ 
 

    \theutterance \stepcounter{utterance}  

    & & & \multicolumn{2}{p{0.3\linewidth}}{\cellcolor[rgb]{0.95,0.95,0.95}{%
	\makecell[{{p{\linewidth}}}]{% 
	  \tt {\tiny [GM$|$GM]}  
	 {'startend:quotation': {}, 'length_constraints:number_sentences': {'relation': 'at least', 'num_sentences': 17}} 
	  } 
	   } 
	   } 
	 & & \\ 
 

    \theutterance \stepcounter{utterance}  

    & & & \multicolumn{2}{p{0.3\linewidth}}{\cellcolor[rgb]{0.95,0.95,0.95}{%
	\makecell[{{p{\linewidth}}}]{% 
	  \tt {\tiny [GM$|$GM]}  
	 game_result = WIN 
	  } 
	   } 
	   } 
	 & & \\ 
 

\end{supertabular}
}

\end{document}
