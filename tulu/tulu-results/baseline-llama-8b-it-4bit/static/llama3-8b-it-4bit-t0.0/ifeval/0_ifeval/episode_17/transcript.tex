
\documentclass{article}
\usepackage{colortbl}
\usepackage{makecell}
\usepackage{multirow}
\usepackage{supertabular}

\begin{document}

\newcounter{utterance}

\twocolumn

{ \footnotesize  \setcounter{utterance}{1}
\setlength{\tabcolsep}{0pt}
\begin{supertabular}{c@{$\;$}|p{.15\linewidth}@{}p{.15\linewidth}p{.15\linewidth}p{.15\linewidth}p{.15\linewidth}p{.15\linewidth}}

    \# & $\;$A & \multicolumn{4}{c}{Game Master} & $\;\:$B\\
    \hline 

    \theutterance \stepcounter{utterance}  

    & & \multicolumn{4}{p{0.6\linewidth}}{\cellcolor[rgb]{0.9,0.9,0.9}{%
	\makecell[{{p{\linewidth}}}]{% 
	  \tt {\tiny [A$\langle$GM]}  
	 The Jimenez family has 20 members. Each member has at least one pet. If there are 32 pets in total, what is the maximum number of members that can possibly have 3 pets? Solve this step by step, and wrap your entire response (reasoning process and the final answer) with double quotation marks. 
	  } 
	   } 
	   } 
	 & \\ 
 

    \theutterance \stepcounter{utterance}  

    & \multicolumn{4}{p{0.6\linewidth}}{\cellcolor[rgb]{0.8,1,0.9}{%
	\makecell[{{p{\linewidth}}}]{% 
	  \tt {\tiny [A$\rangle$GM]}  
	 "Let's break down the problem step by step.\\ \tt \\ \tt Step 1: We know that the total number of members is 20 and the total number of pets is 32.\\ \tt \\ \tt Step 2: We want to find the maximum number of members that can possibly have 3 pets. Let's assume that 'x' members have 3 pets. Then, the remaining members will have either 1 or 2 pets.\\ \tt \\ \tt Step 3: The total number of pets can be represented as the sum of the pets of the members who have 3 pets, the pets of the members who have 2 pets, and the pets of the members who have 1 pet.\\ \tt \\ \tt Step 4: Since each member has at least one pet, the minimum number of pets for the members who have 1 pet is 1, and the minimum number of pets for the members who have 2 pets is 2.\\ \tt \\ \tt Step 5: Let's represent the number of members who have 1 pet as 'y'. Then, the number of members who have 2 pets is (20 - x - y).\\ \tt \\ \tt Step 6: The total number of pets can be represented as:\\ \tt 3x + 2(20 - x - y) + y = 32\\ \tt \\ \tt Step 7: Simplifying the equation, we get:\\ \tt 3x + 40 - 2x - 2y + y = 32\\ \tt x + 40 - y = 32\\ \tt x - y = -8\\ \tt \\ \tt Step 8: We want to find the maximum value of 'x'. To do this, we need to minimize the value of 'y'. Since 'y' represents the number of members who have 1 pet, the minimum value of 'y' is 0.\\ \tt \\ \tt Step 9: If 'y' is 0, then 'x' will be equal to -8. However, since 'x' represents the number of members who have 3 pets, it cannot be negative.\\ \tt \\ \tt Step 10: To find the maximum value of 'x', we need to find the maximum value of 'y' such that 'x' is still non-negative.\\ \tt \\ \tt Step 11: Let's try to find the maximum value of 'y'. Since 'y' represents the number of members who have 1 pet, the maximum value of 'y' is 20.\\ \tt \\ \tt Step 12: If 'y' is 20, then 'x' will be equal to -8. However, since 'x' represents the number of members who have 3 pets, it cannot be negative.\\ \tt \\ \tt Step 13: Let's try to find the next possible value of 'y'. Since 'y' represents the number of members who have 1 pet, the next possible value of 'y' is 19.\\ \tt \\ \tt Step 14: If 'y' is 19, then 'x' will be equal to -7. However, since 'x' represents the number of members who have 3 pets, it cannot be negative.\\ \tt \\ \tt Step 15: Let's try to find the next possible value of 'y'. Since 'y' represents the number of members who have 1 pet, the next possible value of 'y' is 18.\\ \tt \\ \tt Step 16: If 'y' is 18, then 'x' will be equal to -10. However, since 'x' represents the number of members who have 3 pets, it cannot be negative.\\ \tt \\ \tt Step 17: Let's try to find the next possible value of 'y'. Since 'y' represents the number of members who have 1 pet, the next possible value of 'y' is 17.\\ \tt \\ \tt Step 18: If 'y' is 17, then 'x' will be equal to -11. However, since 'x' represents the number of members who have 3 pets, it cannot be negative.\\ \tt \\ \tt Step 19: Let's try to find the next possible value of 'y'. Since 'y' represents the number of members who have 1 pet, the next possible value of 'y' is 16.\\ \tt \\ \tt Step 20: If 'y' is 16, then 'x' will be equal to -12. However, since 'x' represents the number of members who have 3 pets, it cannot be negative.\\ \tt \\ \tt Step 21: Let's try to find the next possible value of 'y'. Since 'y' represents the number of members who have 1 pet, the next possible value of 'y' is 15.\\ \tt \\ \tt Step 22: If 'y' is 15, then 'x' will be equal to -13. However, since 'x' represents the number of members who have 3 pets, it cannot be negative.\\ \tt \\ \tt Step 23: Let's try to find the next possible value of 'y'. Since 'y' represents the number of members who have 1 pet, the next possible value of 'y' is 14.\\ \tt \\ \tt Step 24: If 'y' is 14, then 'x' will be equal to -14. However, since 'x' represents the number of members who have 3 pets, it cannot be negative.\\ \tt \\ \tt Step 25: Let's try to find the next possible value of 'y'. Since 'y' represents the number of members who have 1 pet, the next possible value of 'y' is 13.\\ \tt \\ \tt Step 26: If 'y' is 13, then 'x' will be equal to -15. However, since 'x' represents the number of members who have 3 pets, it cannot be negative.\\ \tt \\ \tt Step 27: Let's try to find the next possible value of 'y'. Since 'y' represents the number of members who have 1 pet, the next possible value of 'y' is 12.\\ \tt \\ \tt Step 28: If 'y' is 12, then 'x' will be equal to -16. However, since 'x' represents the number of members who have 3 pets, it cannot be negative.\\ \tt \\ \tt Step 29: Let's try to find the next possible value of 'y'. Since 'y' represents the number of members who have 1 pet, the next possible value of 'y' is 11.\\ \tt \\ \tt Step 
	  } 
	   } 
	   } 
	 & & \\ 
 

    \theutterance \stepcounter{utterance}  

    & & & \multicolumn{2}{p{0.3\linewidth}}{\cellcolor[rgb]{0.95,0.95,0.95}{%
	\makecell[{{p{\linewidth}}}]{% 
	  \tt {\tiny [GM$|$GM]}  
	 {'startend:quotation': {}} 
	  } 
	   } 
	   } 
	 & & \\ 
 

    \theutterance \stepcounter{utterance}  

    & & & \multicolumn{2}{p{0.3\linewidth}}{\cellcolor[rgb]{0.95,0.95,0.95}{%
	\makecell[{{p{\linewidth}}}]{% 
	  \tt {\tiny [GM$|$GM]}  
	 game_result = LOSE 
	  } 
	   } 
	   } 
	 & & \\ 
 

\end{supertabular}
}

\end{document}
