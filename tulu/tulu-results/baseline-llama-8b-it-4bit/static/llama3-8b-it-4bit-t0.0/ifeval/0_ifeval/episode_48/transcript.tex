
\documentclass{article}
\usepackage{colortbl}
\usepackage{makecell}
\usepackage{multirow}
\usepackage{supertabular}

\begin{document}

\newcounter{utterance}

\twocolumn

{ \footnotesize  \setcounter{utterance}{1}
\setlength{\tabcolsep}{0pt}
\begin{supertabular}{c@{$\;$}|p{.15\linewidth}@{}p{.15\linewidth}p{.15\linewidth}p{.15\linewidth}p{.15\linewidth}p{.15\linewidth}}

    \# & $\;$A & \multicolumn{4}{c}{Game Master} & $\;\:$B\\
    \hline 

    \theutterance \stepcounter{utterance}  

    & & \multicolumn{4}{p{0.6\linewidth}}{\cellcolor[rgb]{0.9,0.9,0.9}{%
	\makecell[{{p{\linewidth}}}]{% 
	  \tt {\tiny [A$\langle$GM]}  
	 Explain what happens when you sniff a flower to 3rd grade students. Please answer in Finnish, no other language is allowed throughout your answer. Make sure your response contains a title wrapped in double angular brackets, i.e. <<title>>. 
	  } 
	   } 
	   } 
	 & \\ 
 

    \theutterance \stepcounter{utterance}  

    & \multicolumn{4}{p{0.6\linewidth}}{\cellcolor[rgb]{0.8,1,0.9}{%
	\makecell[{{p{\linewidth}}}]{% 
	  \tt {\tiny [A$\rangle$GM]}  
	 <<Kukkien Hajuja ja Meitä>> \\ \tt \\ \tt Kun sinä haiset kukkaan, tapahtuu eriläinen prosessi. Ensinnäkin, kukka päättää vapauttaa hajua erityisillä soluilla, jotka kutsutaan hedelmöitymiskelpoisiksi soluiksi. Näillä soluilla on erityisiä osia, jotka kutsutaan aromaattisiksi yhdisteiksi. \\ \tt \\ \tt Kun kukka vapauttaa aromaattisia yhdisteitä, ne levittyvät ilmaan. Kun sinä haiset kukkaan, aromaattiset yhdisteet päättävät liikkua sinun nenäsi läpi. Kun ne päättävät tulla sinun nenään, ne päättävät myös lähettää signaalia sinun aivoihin. \\ \tt \\ \tt Sinun aivot päättävät, että sinulla on haju, ja sinä päättäät, että kukka on houkutteleva. Tämä on yksinkertainen kuvaus siitä, mitä tapahtuu, kun sinä haiset kukkaan. 
	  } 
	   } 
	   } 
	 & & \\ 
 

    \theutterance \stepcounter{utterance}  

    & & & \multicolumn{2}{p{0.3\linewidth}}{\cellcolor[rgb]{0.95,0.95,0.95}{%
	\makecell[{{p{\linewidth}}}]{% 
	  \tt {\tiny [GM$|$GM]}  
	 {'language:response_language': {'language': 'fi'}, 'detectable_format:title': {}} 
	  } 
	   } 
	   } 
	 & & \\ 
 

    \theutterance \stepcounter{utterance}  

    & & & \multicolumn{2}{p{0.3\linewidth}}{\cellcolor[rgb]{0.95,0.95,0.95}{%
	\makecell[{{p{\linewidth}}}]{% 
	  \tt {\tiny [GM$|$GM]}  
	 game_result = WIN 
	  } 
	   } 
	   } 
	 & & \\ 
 

\end{supertabular}
}

\end{document}
