
\documentclass{article}
\usepackage{colortbl}
\usepackage{makecell}
\usepackage{multirow}
\usepackage{supertabular}

\begin{document}

\newcounter{utterance}

\twocolumn

{ \footnotesize  \setcounter{utterance}{1}
\setlength{\tabcolsep}{0pt}
\begin{supertabular}{c@{$\;$}|p{.15\linewidth}@{}p{.15\linewidth}p{.15\linewidth}p{.15\linewidth}p{.15\linewidth}p{.15\linewidth}}

    \# & $\;$A & \multicolumn{4}{c}{Game Master} & $\;\:$B\\
    \hline 

    \theutterance \stepcounter{utterance}  

    & & \multicolumn{4}{p{0.6\linewidth}}{\cellcolor[rgb]{0.9,0.9,0.9}{%
	\makecell[{{p{\linewidth}}}]{% 
	  \tt {\tiny [A$\langle$GM]}  
	 Write an interesting and funny article about the biology of a banana peel. In your response, the word disappointed should appear at least 2 times, and at least six section should be highlighted with markdown,  i.e *banana peel*. 
	  } 
	   } 
	   } 
	 & \\ 
 

    \theutterance \stepcounter{utterance}  

    & \multicolumn{4}{p{0.6\linewidth}}{\cellcolor[rgb]{0.8,1,0.9}{%
	\makecell[{{p{\linewidth}}}]{% 
	  \tt {\tiny [A$\rangle$GM]}  
	 **The Peel-fectly Fascinating Biology of a Banana**\\ \tt \\ \tt Are you ready to go bananas (pun intended) over the amazing biology of a banana peel? If you're feeling disappointed that you've never stopped to think about the intricacies of a banana's outer layer, then let's dive in and explore the fascinating world of banana peels!\\ \tt \\ \tt ### *The Life Cycle of a Banana Peel*\\ \tt \\ \tt A banana peel's life is short-lived, but its impact is long-lasting. Born from the skin of a banana, the peel is a thin, yellow layer that protects the fruit from the outside world. As the banana ripens, the peel begins to dry out and wrinkle, signaling its impending doom. But don't be disappointed – the peel's demise is just the beginning of its journey.\\ \tt \\ \tt ### *The Peel's Composition*\\ \tt \\ \tt The banana peel is made up of several layers, including the epidermis (the outermost layer), the hypodermis (a layer of cells beneath the epidermis), and the mesocarp (the fleshy middle layer). The peel is also home to a variety of cells, including skin cells, nerve cells, and even tiny little hairs (because who doesn't love a good banana hairdo?). But don't be disappointed – the peel's composition is just the tip of the iceberg.\\ \tt \\ \tt ### *The Peel's Purpose*\\ \tt \\ \tt So, what's the point of a banana peel? Well, aside from being a delicious snack for monkeys and a slippery surface for banana enthusiasts, the peel serves several purposes. It protects the banana from damage, regulates water loss, and even helps to ripen the fruit. But if you're feeling disappointed that the peel doesn't have a more glamorous role, just remember – it's a tough job, but someone's gotta do it.\\ \tt \\ \tt ### *The Peel's Decomposition*\\ \tt \\ \tt As the banana peel reaches the end of its life cycle, it begins to decompose. This process is facilitated by microorganisms like bacteria and fungi, which break down the peel's complex molecules into simpler compounds. It's a bit like a banana-themed game of Tetris – the peel's pieces are constantly shifting and rearranging themselves. But don't be disappointed – decomposition is just the natural order of things.\\ \tt \\ \tt ### *The Peel's Legacy*\\ \tt \\ \tt Even after the banana peel has decomposed, its legacy lives on. The nutrients released during decomposition are absorbed by the surrounding soil, where they can be used by other plants to grow and thrive. It's a bit like a banana-themed game of "The Circle of Life" – the peel's energy is constantly being cycled back into the environment. So, the next time you're feeling disappointed about the fleeting nature of a banana peel, just remember – it's all part of the bigger picture.\\ \tt \\ \tt ### *The Peel's Place in History*\\ \tt \\ \tt Banana peels have been a part of human history for thousands of years. In ancient civilizations, they were used for everything from medicine to magic. And let's not forget the infamous banana peel prank – a classic joke that's been slipping up people for centuries. But if you're feeling disappointed that the peel's place in history isn't more glamorous, just remember – it's a peel-fectly fascinating story nonetheless. 
	  } 
	   } 
	   } 
	 & & \\ 
 

    \theutterance \stepcounter{utterance}  

    & & & \multicolumn{2}{p{0.3\linewidth}}{\cellcolor[rgb]{0.95,0.95,0.95}{%
	\makecell[{{p{\linewidth}}}]{% 
	  \tt {\tiny [GM$|$GM]}  
	 {'keywords:frequency': {'relation': 'at least', 'keyword': 'disappointed', 'frequency': 2}, 'detectable_format:number_highlighted_sections': {'num_highlights': 6}} 
	  } 
	   } 
	   } 
	 & & \\ 
 

    \theutterance \stepcounter{utterance}  

    & & & \multicolumn{2}{p{0.3\linewidth}}{\cellcolor[rgb]{0.95,0.95,0.95}{%
	\makecell[{{p{\linewidth}}}]{% 
	  \tt {\tiny [GM$|$GM]}  
	 game_result = WIN 
	  } 
	   } 
	   } 
	 & & \\ 
 

\end{supertabular}
}

\end{document}
