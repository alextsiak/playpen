
\documentclass{article}
\usepackage{colortbl}
\usepackage{makecell}
\usepackage{multirow}
\usepackage{supertabular}

\begin{document}

\newcounter{utterance}

\twocolumn

{ \footnotesize  \setcounter{utterance}{1}
\setlength{\tabcolsep}{0pt}
\begin{supertabular}{c@{$\;$}|p{.15\linewidth}@{}p{.15\linewidth}p{.15\linewidth}p{.15\linewidth}p{.15\linewidth}p{.15\linewidth}}

    \# & $\;$A & \multicolumn{4}{c}{Game Master} & $\;\:$B\\
    \hline 

    \theutterance \stepcounter{utterance}  

    & & \multicolumn{4}{p{0.6\linewidth}}{\cellcolor[rgb]{0.9,0.9,0.9}{%
	\makecell[{{p{\linewidth}}}]{% 
	  \tt {\tiny [A$\langle$GM]}  
	 You are an expert in causal inference. The following question is not a typical commonsense query, but rather a meticulously designed question created by a professor specializing in causal inference, intended to assess the students' mastery of the course content.\\ \tt Imagine a self-contained, hypothetical world with only the following conditions, and without any unmentioned factors or causal relationships: Gender has a direct effect on treatment and peanut allergy. Treatment has a direct effect on peanut allergy. For patients who are not male and not receiving treatment, the probability of being allergic to peanuts is 84%. For patients who are not male and receiving treatment, the probability of being allergic to peanuts is 71%. For patients who are male and not receiving treatment, the probability of being allergic to peanuts is 21%. For patients who are male and receiving treatment, the probability of being allergic to peanuts is 3%. The overall probability of male gender is 54%. For patients receiving treatment, would it be more likely to see being allergic to peanuts if the patient had received no treatment?\\ \tt Start your answer with 'Yes' or 'No', followed by additional reasoning or evidence to support your explanation. 
	  } 
	   } 
	   } 
	 & \\ 
 

    \theutterance \stepcounter{utterance}  

    & \multicolumn{4}{p{0.6\linewidth}}{\cellcolor[rgb]{0.8,1,0.9}{%
	\makecell[{{p{\linewidth}}}]{% 
	  \tt {\tiny [A$\rangle$GM]}  
	 No, for patients receiving treatment, it would not be more likely to see being allergic to peanuts if the patient had received no treatment.\\ \tt \\ \tt To understand this, let's break down the given probabilities:\\ \tt \\ \tt - P(Peanut Allergy | Not Male, No Treatment) = 84%\\ \tt - P(Peanut Allergy | Not Male, Treatment) = 71%\\ \tt - P(Peanut Allergy | Male, No Treatment) = 21%\\ \tt - P(Peanut Allergy | Male, Treatment) = 3%\\ \tt - P(Male) = 54%\\ \tt \\ \tt We are interested in the probability of peanut allergy given treatment, P(Peanut Allergy | Treatment). To calculate this, we can use the law of total probability:\\ \tt \\ \tt P(Peanut Allergy | Treatment) = P(Peanut Allergy | Male, Treatment) * P(Male | Treatment) + P(Peanut Allergy | Not Male, Treatment) * P(Not Male | Treatment)\\ \tt \\ \tt We know P(Male | Treatment) = P(Male) / P(Treatment), but we don't have P(Treatment) directly. However, we can use the fact that P(Treatment) = P(Male | Treatment) * P(Treatment | Male) + P(Not Male | Treatment) * P(Treatment | Not Male). Since P(Treatment | Male) = 1 - P(Peanut Allergy | Male, Treatment) = 1 - 0.03 
	  } 
	   } 
	   } 
	 & & \\ 
 

    \theutterance \stepcounter{utterance}  

    & & & \multicolumn{2}{p{0.3\linewidth}}{\cellcolor[rgb]{0.95,0.95,0.95}{%
	\makecell[{{p{\linewidth}}}]{% 
	  \tt {\tiny [GM$|$GM]}  
	 No 
	  } 
	   } 
	   } 
	 & & \\ 
 

    \theutterance \stepcounter{utterance}  

    & & & \multicolumn{2}{p{0.3\linewidth}}{\cellcolor[rgb]{0.95,0.95,0.95}{%
	\makecell[{{p{\linewidth}}}]{% 
	  \tt {\tiny [GM$|$GM]}  
	 yes 
	  } 
	   } 
	   } 
	 & & \\ 
 

    \theutterance \stepcounter{utterance}  

    & & & \multicolumn{2}{p{0.3\linewidth}}{\cellcolor[rgb]{0.95,0.95,0.95}{%
	\makecell[{{p{\linewidth}}}]{% 
	  \tt {\tiny [GM$|$GM]}  
	 game_result = LOSE 
	  } 
	   } 
	   } 
	 & & \\ 
 

\end{supertabular}
}

\end{document}
