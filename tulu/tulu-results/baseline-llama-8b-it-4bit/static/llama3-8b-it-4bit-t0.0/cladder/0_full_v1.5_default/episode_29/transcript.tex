
\documentclass{article}
\usepackage{colortbl}
\usepackage{makecell}
\usepackage{multirow}
\usepackage{supertabular}

\begin{document}

\newcounter{utterance}

\twocolumn

{ \footnotesize  \setcounter{utterance}{1}
\setlength{\tabcolsep}{0pt}
\begin{supertabular}{c@{$\;$}|p{.15\linewidth}@{}p{.15\linewidth}p{.15\linewidth}p{.15\linewidth}p{.15\linewidth}p{.15\linewidth}}

    \# & $\;$A & \multicolumn{4}{c}{Game Master} & $\;\:$B\\
    \hline 

    \theutterance \stepcounter{utterance}  

    & & \multicolumn{4}{p{0.6\linewidth}}{\cellcolor[rgb]{0.9,0.9,0.9}{%
	\makecell[{{p{\linewidth}}}]{% 
	  \tt {\tiny [A$\langle$GM]}  
	 You are an expert in causal inference. The following question is not a typical commonsense query, but rather a meticulously designed question created by a professor specializing in causal inference, intended to assess the students' mastery of the course content.\\ \tt Imagine a self-contained, hypothetical world with only the following conditions, and without any unmentioned factors or causal relationships: Eating citrus has a direct effect on vitmain C. Vitmain C has a direct effect on scurvy. For patients not consuming citrus, the probability of scurvy is 73%. For patients consuming citrus, the probability of scurvy is 54%. Does eating citrus positively affect scurvy through vitmain C?\\ \tt Start your answer with 'Yes' or 'No', followed by additional reasoning or evidence to support your explanation. 
	  } 
	   } 
	   } 
	 & \\ 
 

    \theutterance \stepcounter{utterance}  

    & \multicolumn{4}{p{0.6\linewidth}}{\cellcolor[rgb]{0.8,1,0.9}{%
	\makecell[{{p{\linewidth}}}]{% 
	  \tt {\tiny [A$\rangle$GM]}  
	 No\\ \tt \\ \tt To assess whether eating citrus positively affects scurvy through vitamin C, we need to consider the potential causal relationships between these variables. The given information suggests that eating citrus has a direct effect on vitamin C, and vitamin C has a direct effect on scurvy. However, the effect of eating citrus on scurvy is not entirely mediated by vitamin C.\\ \tt \\ \tt The probability of scurvy for patients not consuming citrus is 73%, while for patients consuming citrus, the probability of scurvy is 54%. This suggests that eating citrus has a protective effect against scurvy, but the question is whether this effect is due to the increase in vitamin C.\\ \tt \\ \tt To determine this, we can use the potential outcomes framework. Let's denote the potential outcomes as follows:\\ \tt \\ \tt - S: Scurvy\\ \tt - N: No scurvy\\ \tt - C: Citrus consumption\\ \tt - V: Vitamin C\\ \tt \\ \tt The given information implies the following causal relationships:\\ \tt \\ \tt - P(S|N) = 0.73\\ \tt - P(S|C) = 0.54\\ \tt \\ \tt We can also infer the following relationships:\\ \tt \\ \tt - P(V|C) = 1 (since eating citrus directly increases vitamin C)\\ \tt - P(S|V) = 0 (since vitamin C directly prevents scurvy)\\ \tt \\ \tt However, the question is whether eating citrus positively affects scurvy through vitamin C. To answer this, we need to consider the potential outcomes when vitamin C is present or absent, regardless of 
	  } 
	   } 
	   } 
	 & & \\ 
 

    \theutterance \stepcounter{utterance}  

    & & & \multicolumn{2}{p{0.3\linewidth}}{\cellcolor[rgb]{0.95,0.95,0.95}{%
	\makecell[{{p{\linewidth}}}]{% 
	  \tt {\tiny [GM$|$GM]}  
	 No 
	  } 
	   } 
	   } 
	 & & \\ 
 

    \theutterance \stepcounter{utterance}  

    & & & \multicolumn{2}{p{0.3\linewidth}}{\cellcolor[rgb]{0.95,0.95,0.95}{%
	\makecell[{{p{\linewidth}}}]{% 
	  \tt {\tiny [GM$|$GM]}  
	 no 
	  } 
	   } 
	   } 
	 & & \\ 
 

    \theutterance \stepcounter{utterance}  

    & & & \multicolumn{2}{p{0.3\linewidth}}{\cellcolor[rgb]{0.95,0.95,0.95}{%
	\makecell[{{p{\linewidth}}}]{% 
	  \tt {\tiny [GM$|$GM]}  
	 game_result = WIN 
	  } 
	   } 
	   } 
	 & & \\ 
 

\end{supertabular}
}

\end{document}
