
\documentclass{article}
\usepackage{colortbl}
\usepackage{makecell}
\usepackage{multirow}
\usepackage{supertabular}

\begin{document}

\newcounter{utterance}

\twocolumn

{ \footnotesize  \setcounter{utterance}{1}
\setlength{\tabcolsep}{0pt}
\begin{supertabular}{c@{$\;$}|p{.15\linewidth}@{}p{.15\linewidth}p{.15\linewidth}p{.15\linewidth}p{.15\linewidth}p{.15\linewidth}}

    \# & $\;$A & \multicolumn{4}{c}{Game Master} & $\;\:$B\\
    \hline 

    \theutterance \stepcounter{utterance}  

    & & \multicolumn{4}{p{0.6\linewidth}}{\cellcolor[rgb]{0.9,0.9,0.9}{%
	\makecell[{{p{\linewidth}}}]{% 
	  \tt {\tiny [A$\langle$GM]}  
	 You are an expert in causal inference. The following question is not a typical commonsense query, but rather a meticulously designed question created by a professor specializing in causal inference, intended to assess the students' mastery of the course content.\\ \tt Imagine a self-contained, hypothetical world with only the following conditions, and without any unmentioned factors or causal relationships: Gender has a direct effect on department competitiveness and admission status. Department competitiveness has a direct effect on admission status. For individuals who are not male and applicants to a non-competitive department, the probability of admission acceptance is 58%. For individuals who are not male and applicants to a competitive department, the probability of admission acceptance is 25%. For individuals who are male and applicants to a non-competitive department, the probability of admission acceptance is 59%. For individuals who are male and applicants to a competitive department, the probability of admission acceptance is 28%. For individuals who are not male, the probability of competitive department is 29%. For individuals who are male, the probability of competitive department is 77%. Does gender positively affect admission status through department competitiveness?\\ \tt Start your answer with 'Yes' or 'No', followed by additional reasoning or evidence to support your explanation. 
	  } 
	   } 
	   } 
	 & \\ 
 

    \theutterance \stepcounter{utterance}  

    & \multicolumn{4}{p{0.6\linewidth}}{\cellcolor[rgb]{0.8,1,0.9}{%
	\makecell[{{p{\linewidth}}}]{% 
	  \tt {\tiny [A$\rangle$GM]}  
	 No\\ \tt \\ \tt To assess whether gender positively affects admission status through department competitiveness, we need to examine the relationship between gender, department competitiveness, and admission status. \\ \tt \\ \tt First, let's define the variables and their relationships:\\ \tt \\ \tt - Let G be the gender (0 for not male, 1 for male)\\ \tt - Let D be the department competitiveness (0 for non-competitive, 1 for competitive)\\ \tt - Let A be the admission status (0 for rejection, 1 for acceptance)\\ \tt \\ \tt Given the information, we have the following conditional probabilities:\\ \tt \\ \tt - P(A=1|G=0, D=0) = 58%\\ \tt - P(A=1|G=0, D=1) = 25%\\ \tt - P(A=1|G=1, D=0) = 59%\\ \tt - P(A=1|G=1, D=1) = 28%\\ \tt - P(D=1|G=0) = 29%\\ \tt - P(D=1|G=1) = 77%\\ \tt \\ \tt We are interested in the effect of gender on admission status through department competitiveness. To do this, we need to examine the conditional probability of admission status given gender and department competitiveness, P(A=1|G, D).\\ \tt \\ \tt Using the law of total probability, we can calculate P(A=1|G, D) as follows:\\ \tt \\ \tt - P(A=1|G=0, D=0) = 58%\\ \tt - P(A=1|G=0, 
	  } 
	   } 
	   } 
	 & & \\ 
 

    \theutterance \stepcounter{utterance}  

    & & & \multicolumn{2}{p{0.3\linewidth}}{\cellcolor[rgb]{0.95,0.95,0.95}{%
	\makecell[{{p{\linewidth}}}]{% 
	  \tt {\tiny [GM$|$GM]}  
	 No 
	  } 
	   } 
	   } 
	 & & \\ 
 

    \theutterance \stepcounter{utterance}  

    & & & \multicolumn{2}{p{0.3\linewidth}}{\cellcolor[rgb]{0.95,0.95,0.95}{%
	\makecell[{{p{\linewidth}}}]{% 
	  \tt {\tiny [GM$|$GM]}  
	 no 
	  } 
	   } 
	   } 
	 & & \\ 
 

    \theutterance \stepcounter{utterance}  

    & & & \multicolumn{2}{p{0.3\linewidth}}{\cellcolor[rgb]{0.95,0.95,0.95}{%
	\makecell[{{p{\linewidth}}}]{% 
	  \tt {\tiny [GM$|$GM]}  
	 game_result = WIN 
	  } 
	   } 
	   } 
	 & & \\ 
 

\end{supertabular}
}

\end{document}
