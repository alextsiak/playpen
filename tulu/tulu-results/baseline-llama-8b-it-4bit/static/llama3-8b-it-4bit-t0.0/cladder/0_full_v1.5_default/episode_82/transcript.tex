
\documentclass{article}
\usepackage{colortbl}
\usepackage{makecell}
\usepackage{multirow}
\usepackage{supertabular}

\begin{document}

\newcounter{utterance}

\twocolumn

{ \footnotesize  \setcounter{utterance}{1}
\setlength{\tabcolsep}{0pt}
\begin{supertabular}{c@{$\;$}|p{.15\linewidth}@{}p{.15\linewidth}p{.15\linewidth}p{.15\linewidth}p{.15\linewidth}p{.15\linewidth}}

    \# & $\;$A & \multicolumn{4}{c}{Game Master} & $\;\:$B\\
    \hline 

    \theutterance \stepcounter{utterance}  

    & & \multicolumn{4}{p{0.6\linewidth}}{\cellcolor[rgb]{0.9,0.9,0.9}{%
	\makecell[{{p{\linewidth}}}]{% 
	  \tt {\tiny [A$\langle$GM]}  
	 You are an expert in causal inference. The following question is not a typical commonsense query, but rather a meticulously designed question created by a professor specializing in causal inference, intended to assess the students' mastery of the course content.\\ \tt Imagine a self-contained, hypothetical world with only the following conditions, and without any unmentioned factors or causal relationships: Kwox has a direct effect on kwoz. Swoq has a direct effect on kwoz. The overall probability of kwox is 1%. For those who are not kwox, the probability of kwoz is 47%. For those who are kwox, the probability of kwoz is 92%. Is kwoz less likely than not kwoz overall?\\ \tt Start your answer with 'Yes' or 'No', followed by additional reasoning or evidence to support your explanation. 
	  } 
	   } 
	   } 
	 & \\ 
 

    \theutterance \stepcounter{utterance}  

    & \multicolumn{4}{p{0.6\linewidth}}{\cellcolor[rgb]{0.8,1,0.9}{%
	\makecell[{{p{\linewidth}}}]{% 
	  \tt {\tiny [A$\rangle$GM]}  
	 No\\ \tt \\ \tt To address this question, we need to apply the principles of causal inference, specifically the concept of conditional probability. We are given the following information:\\ \tt \\ \tt 1. Kwox has a direct effect on kwoz.\\ \tt 2. Swoq has a direct effect on kwoz.\\ \tt 3. The overall probability of kwox is 1%.\\ \tt 4. For those who are not kwox, the probability of kwoz is 47%.\\ \tt 5. For those who are kwox, the probability of kwoz is 92%.\\ \tt \\ \tt We can use the law of total probability to calculate the overall probability of kwoz. Let's denote the probability of kwoz as P(kwoz), the probability of kwox as P(kwox), and the probability of not kwox as P(not kwox). We are given that P(kwox) = 1% = 0.01 and P(not kwox) = 1 - P(kwox) = 0.99.\\ \tt \\ \tt We can use the law of total probability to write:\\ \tt \\ \tt P(kwoz) = P(kwoz | kwox) \* P(kwox) + P(kwoz | not kwox) \* P(not kwox)\\ \tt \\ \tt Substituting the given values, we get:\\ \tt \\ \tt P(kwoz) = 0.92 \* 0.01 + 0.47 \* 0 
	  } 
	   } 
	   } 
	 & & \\ 
 

    \theutterance \stepcounter{utterance}  

    & & & \multicolumn{2}{p{0.3\linewidth}}{\cellcolor[rgb]{0.95,0.95,0.95}{%
	\makecell[{{p{\linewidth}}}]{% 
	  \tt {\tiny [GM$|$GM]}  
	 No 
	  } 
	   } 
	   } 
	 & & \\ 
 

    \theutterance \stepcounter{utterance}  

    & & & \multicolumn{2}{p{0.3\linewidth}}{\cellcolor[rgb]{0.95,0.95,0.95}{%
	\makecell[{{p{\linewidth}}}]{% 
	  \tt {\tiny [GM$|$GM]}  
	 yes 
	  } 
	   } 
	   } 
	 & & \\ 
 

    \theutterance \stepcounter{utterance}  

    & & & \multicolumn{2}{p{0.3\linewidth}}{\cellcolor[rgb]{0.95,0.95,0.95}{%
	\makecell[{{p{\linewidth}}}]{% 
	  \tt {\tiny [GM$|$GM]}  
	 game_result = LOSE 
	  } 
	   } 
	   } 
	 & & \\ 
 

\end{supertabular}
}

\end{document}
