
\documentclass{article}
\usepackage{colortbl}
\usepackage{makecell}
\usepackage{multirow}
\usepackage{supertabular}

\begin{document}

\newcounter{utterance}

\twocolumn

{ \footnotesize  \setcounter{utterance}{1}
\setlength{\tabcolsep}{0pt}
\begin{supertabular}{c@{$\;$}|p{.15\linewidth}@{}p{.15\linewidth}p{.15\linewidth}p{.15\linewidth}p{.15\linewidth}p{.15\linewidth}}

    \# & $\;$A & \multicolumn{4}{c}{Game Master} & $\;\:$B\\
    \hline 

    \theutterance \stepcounter{utterance}  

    & & \multicolumn{4}{p{0.6\linewidth}}{\cellcolor[rgb]{0.9,0.9,0.9}{%
	\makecell[{{p{\linewidth}}}]{% 
	  \tt {\tiny [A$\langle$GM]}  
	 You are an expert in causal inference. The following question is not a typical commonsense query, but rather a meticulously designed question created by a professor specializing in causal inference, intended to assess the students' mastery of the course content.\\ \tt Imagine a self-contained, hypothetical world with only the following conditions, and without any unmentioned factors or causal relationships: Husband has a direct effect on wife and alarm clock. Wife has a direct effect on alarm clock. The overall probability of alarm set by husband is 17%. The probability of alarm not set by husband and ringing alarm is 69%. The probability of alarm set by husband and ringing alarm is 3%. Is the chance of ringing alarm smaller when observing alarm set by husband?\\ \tt Start your answer with 'Yes' or 'No', followed by additional reasoning or evidence to support your explanation. 
	  } 
	   } 
	   } 
	 & \\ 
 

    \theutterance \stepcounter{utterance}  

    & \multicolumn{4}{p{0.6\linewidth}}{\cellcolor[rgb]{0.8,1,0.9}{%
	\makecell[{{p{\linewidth}}}]{% 
	  \tt {\tiny [A$\rangle$GM]}  
	 No, the chance of a ringing alarm is not smaller when observing alarm set by husband.\\ \tt \\ \tt To analyze this problem, we can use the concept of conditional probability and the causal diagram provided. Let's denote the event of the alarm being set by the husband as H, the event of the alarm being set by the wife as W, and the event of the alarm ringing as R.\\ \tt \\ \tt Given the probabilities:\\ \tt \\ \tt 1. P(H) = 17% (probability of alarm set by husband)\\ \tt 2. P(R|~H) = 69% (probability of alarm ringing given that it's not set by husband)\\ \tt 3. P(R|H) = 3% (probability of alarm ringing given that it's set by husband)\\ \tt \\ \tt We want to compare the probability of the alarm ringing when the husband sets it (P(R|H)) with the probability of the alarm ringing when the husband does not set it (P(R|~H)).\\ \tt \\ \tt To do this, we can use the law of total probability to calculate the overall probability of the alarm ringing:\\ \tt \\ \tt P(R) = P(R|H)P(H) + P(R|~H)P(~H)\\ \tt \\ \tt Since P(~H) = 1 - P(H) = 1 - 0.17 = 0.83, we can substitute the given values:\\ \tt \\ \tt P(R) = 0.03(0.17) + 0.69(0.83)\\ \tt = 0.0051 + 0 
	  } 
	   } 
	   } 
	 & & \\ 
 

    \theutterance \stepcounter{utterance}  

    & & & \multicolumn{2}{p{0.3\linewidth}}{\cellcolor[rgb]{0.95,0.95,0.95}{%
	\makecell[{{p{\linewidth}}}]{% 
	  \tt {\tiny [GM$|$GM]}  
	 No 
	  } 
	   } 
	   } 
	 & & \\ 
 

    \theutterance \stepcounter{utterance}  

    & & & \multicolumn{2}{p{0.3\linewidth}}{\cellcolor[rgb]{0.95,0.95,0.95}{%
	\makecell[{{p{\linewidth}}}]{% 
	  \tt {\tiny [GM$|$GM]}  
	 yes 
	  } 
	   } 
	   } 
	 & & \\ 
 

    \theutterance \stepcounter{utterance}  

    & & & \multicolumn{2}{p{0.3\linewidth}}{\cellcolor[rgb]{0.95,0.95,0.95}{%
	\makecell[{{p{\linewidth}}}]{% 
	  \tt {\tiny [GM$|$GM]}  
	 game_result = LOSE 
	  } 
	   } 
	   } 
	 & & \\ 
 

\end{supertabular}
}

\end{document}
