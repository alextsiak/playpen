
\documentclass{article}
\usepackage{colortbl}
\usepackage{makecell}
\usepackage{multirow}
\usepackage{supertabular}

\begin{document}

\newcounter{utterance}

\twocolumn

{ \footnotesize  \setcounter{utterance}{1}
\setlength{\tabcolsep}{0pt}
\begin{supertabular}{c@{$\;$}|p{.15\linewidth}@{}p{.15\linewidth}p{.15\linewidth}p{.15\linewidth}p{.15\linewidth}p{.15\linewidth}}

    \# & $\;$A & \multicolumn{4}{c}{Game Master} & $\;\:$B\\
    \hline 

    \theutterance \stepcounter{utterance}  

    & & \multicolumn{4}{p{0.6\linewidth}}{\cellcolor[rgb]{0.9,0.9,0.9}{%
	\makecell[{{p{\linewidth}}}]{% 
	  \tt {\tiny [A$\langle$GM]}  
	 You are an expert in causal inference. The following question is not a typical commonsense query, but rather a meticulously designed question created by a professor specializing in causal inference, intended to assess the students' mastery of the course content.\\ \tt Imagine a self-contained, hypothetical world with only the following conditions, and without any unmentioned factors or causal relationships: Education level has a direct effect on skill. Skill has a direct effect on black hair. The overall probability of college degree or higher is 6%. The probability of high school degree or lower and black hair is 25%. The probability of college degree or higher and black hair is 3%. Is the chance of black hair smaller when observing college degree or higher?\\ \tt Start your answer with 'Yes' or 'No', followed by additional reasoning or evidence to support your explanation. 
	  } 
	   } 
	   } 
	 & \\ 
 

    \theutterance \stepcounter{utterance}  

    & \multicolumn{4}{p{0.6\linewidth}}{\cellcolor[rgb]{0.8,1,0.9}{%
	\makecell[{{p{\linewidth}}}]{% 
	  \tt {\tiny [A$\rangle$GM]}  
	 No, the chance of black hair is not smaller when observing college degree or higher.\\ \tt \\ \tt To understand this, let's break down the given information:\\ \tt \\ \tt 1. Education level has a direct effect on skill.\\ \tt 2. Skill has a direct effect on black hair.\\ \tt 3. Overall probability of college degree or higher is 6%.\\ \tt 4. Probability of high school degree or lower and black hair is 25%.\\ \tt 5. Probability of college degree or higher and black hair is 3%.\\ \tt \\ \tt We can use the concept of conditional probability to analyze this situation. Let's define the following events:\\ \tt \\ \tt - A: College degree or higher\\ \tt - B: Black hair\\ \tt - S: Skill\\ \tt \\ \tt Given the direct effect of skill on black hair, we can assume that the probability of black hair given skill is a constant, say P(B|S). \\ \tt \\ \tt We can use the law of total probability to calculate the overall probability of black hair:\\ \tt \\ \tt P(B) = P(B|A)P(A) + P(B|A')P(A')\\ \tt \\ \tt where P(B|A) is the probability of black hair given college degree or higher, P(B|A') is the probability of black hair given high school degree or lower, and P(A') is the probability of high school degree or lower.\\ \tt \\ \tt We are given P(B|A) = 3% and P(B|A') = 25%. We can calculate P(A) = 6% and P(A') = 94% (since the overall probability of 
	  } 
	   } 
	   } 
	 & & \\ 
 

    \theutterance \stepcounter{utterance}  

    & & & \multicolumn{2}{p{0.3\linewidth}}{\cellcolor[rgb]{0.95,0.95,0.95}{%
	\makecell[{{p{\linewidth}}}]{% 
	  \tt {\tiny [GM$|$GM]}  
	 No 
	  } 
	   } 
	   } 
	 & & \\ 
 

    \theutterance \stepcounter{utterance}  

    & & & \multicolumn{2}{p{0.3\linewidth}}{\cellcolor[rgb]{0.95,0.95,0.95}{%
	\makecell[{{p{\linewidth}}}]{% 
	  \tt {\tiny [GM$|$GM]}  
	 no 
	  } 
	   } 
	   } 
	 & & \\ 
 

    \theutterance \stepcounter{utterance}  

    & & & \multicolumn{2}{p{0.3\linewidth}}{\cellcolor[rgb]{0.95,0.95,0.95}{%
	\makecell[{{p{\linewidth}}}]{% 
	  \tt {\tiny [GM$|$GM]}  
	 game_result = WIN 
	  } 
	   } 
	   } 
	 & & \\ 
 

\end{supertabular}
}

\end{document}
