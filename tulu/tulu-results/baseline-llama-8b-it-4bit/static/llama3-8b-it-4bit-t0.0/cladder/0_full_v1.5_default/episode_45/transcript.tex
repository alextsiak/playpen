
\documentclass{article}
\usepackage{colortbl}
\usepackage{makecell}
\usepackage{multirow}
\usepackage{supertabular}

\begin{document}

\newcounter{utterance}

\twocolumn

{ \footnotesize  \setcounter{utterance}{1}
\setlength{\tabcolsep}{0pt}
\begin{supertabular}{c@{$\;$}|p{.15\linewidth}@{}p{.15\linewidth}p{.15\linewidth}p{.15\linewidth}p{.15\linewidth}p{.15\linewidth}}

    \# & $\;$A & \multicolumn{4}{c}{Game Master} & $\;\:$B\\
    \hline 

    \theutterance \stepcounter{utterance}  

    & & \multicolumn{4}{p{0.6\linewidth}}{\cellcolor[rgb]{0.9,0.9,0.9}{%
	\makecell[{{p{\linewidth}}}]{% 
	  \tt {\tiny [A$\langle$GM]}  
	 You are an expert in causal inference. The following question is not a typical commonsense query, but rather a meticulously designed question created by a professor specializing in causal inference, intended to assess the students' mastery of the course content.\\ \tt Imagine a self-contained, hypothetical world with only the following conditions, and without any unmentioned factors or causal relationships: Kidney stone size has a direct effect on treatment and lip thickness. Treatment has a direct effect on lip thickness. For patients who have small kidney stones and not receiving treatment, the probability of thick lips is 6%. For patients who have small kidney stones and receiving treatment, the probability of thick lips is 38%. For patients who have large kidney stones and not receiving treatment, the probability of thick lips is 63%. For patients who have large kidney stones and receiving treatment, the probability of thick lips is 95%. The overall probability of large kidney stone is 50%. For patients receiving treatment, would it be more likely to see thick lips if the patient had received no treatment?\\ \tt Start your answer with 'Yes' or 'No', followed by additional reasoning or evidence to support your explanation. 
	  } 
	   } 
	   } 
	 & \\ 
 

    \theutterance \stepcounter{utterance}  

    & \multicolumn{4}{p{0.6\linewidth}}{\cellcolor[rgb]{0.8,1,0.9}{%
	\makecell[{{p{\linewidth}}}]{% 
	  \tt {\tiny [A$\rangle$GM]}  
	 No.\\ \tt \\ \tt To answer this question, we need to apply the concept of causal inference, specifically the counterfactual framework. We are interested in the effect of receiving no treatment on the probability of thick lips, conditional on receiving treatment.\\ \tt \\ \tt Let's define the following variables:\\ \tt \\ \tt - T: Treatment (0 = no treatment, 1 = treatment)\\ \tt - K: Kidney stone size (0 = small, 1 = large)\\ \tt - L: Lip thickness (0 = thin, 1 = thick)\\ \tt \\ \tt We are given the following probabilities:\\ \tt \\ \tt - P(L|K=0, T=0) = 0.06\\ \tt - P(L|K=0, T=1) = 0.38\\ \tt - P(L|K=1, T=0) = 0.63\\ \tt - P(L|K=1, T=1) = 0.95\\ \tt \\ \tt We are also given that the overall probability of large kidney stone is 0.5.\\ \tt \\ \tt To answer the question, we need to estimate the probability of thick lips if a patient had received no treatment, conditional on receiving treatment. This is equivalent to estimating the counterfactual probability P(L|K=0, T=1).\\ \tt \\ \tt Using the law of total probability, we can write:\\ \tt \\ \tt P(L|T=1) = P(L|K=0, T=1) \* P(K=0|T=1) + P(L|K=1, T=1) \* 
	  } 
	   } 
	   } 
	 & & \\ 
 

    \theutterance \stepcounter{utterance}  

    & & & \multicolumn{2}{p{0.3\linewidth}}{\cellcolor[rgb]{0.95,0.95,0.95}{%
	\makecell[{{p{\linewidth}}}]{% 
	  \tt {\tiny [GM$|$GM]}  
	 No 
	  } 
	   } 
	   } 
	 & & \\ 
 

    \theutterance \stepcounter{utterance}  

    & & & \multicolumn{2}{p{0.3\linewidth}}{\cellcolor[rgb]{0.95,0.95,0.95}{%
	\makecell[{{p{\linewidth}}}]{% 
	  \tt {\tiny [GM$|$GM]}  
	 no 
	  } 
	   } 
	   } 
	 & & \\ 
 

    \theutterance \stepcounter{utterance}  

    & & & \multicolumn{2}{p{0.3\linewidth}}{\cellcolor[rgb]{0.95,0.95,0.95}{%
	\makecell[{{p{\linewidth}}}]{% 
	  \tt {\tiny [GM$|$GM]}  
	 game_result = WIN 
	  } 
	   } 
	   } 
	 & & \\ 
 

\end{supertabular}
}

\end{document}
