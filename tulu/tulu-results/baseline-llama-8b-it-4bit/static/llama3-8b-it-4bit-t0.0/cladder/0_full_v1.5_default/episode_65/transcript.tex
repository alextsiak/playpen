
\documentclass{article}
\usepackage{colortbl}
\usepackage{makecell}
\usepackage{multirow}
\usepackage{supertabular}

\begin{document}

\newcounter{utterance}

\twocolumn

{ \footnotesize  \setcounter{utterance}{1}
\setlength{\tabcolsep}{0pt}
\begin{supertabular}{c@{$\;$}|p{.15\linewidth}@{}p{.15\linewidth}p{.15\linewidth}p{.15\linewidth}p{.15\linewidth}p{.15\linewidth}}

    \# & $\;$A & \multicolumn{4}{c}{Game Master} & $\;\:$B\\
    \hline 

    \theutterance \stepcounter{utterance}  

    & & \multicolumn{4}{p{0.6\linewidth}}{\cellcolor[rgb]{0.9,0.9,0.9}{%
	\makecell[{{p{\linewidth}}}]{% 
	  \tt {\tiny [A$\langle$GM]}  
	 You are an expert in causal inference. The following question is not a typical commonsense query, but rather a meticulously designed question created by a professor specializing in causal inference, intended to assess the students' mastery of the course content.\\ \tt Imagine a self-contained, hypothetical world with only the following conditions, and without any unmentioned factors or causal relationships: Swoy has a direct effect on rixq and xevu. Rixq has a direct effect on zuph. Zuph has a direct effect on xevu. Swoy is unobserved. For those who are not rixq and are not zuph, the probability of xevu is 46%. For those who are not rixq and are zuph, the probability of xevu is 41%. For those who are rixq and are not zuph, the probability of xevu is 66%. For those who are rixq and are zuph, the probability of xevu is 60%. For those who are not rixq, the probability of zuph is 61%. For those who are rixq, the probability of zuph is 34%. For those who are rixq, would it be more likely to see xevu if the individual was not rixq?\\ \tt Start your answer with 'Yes' or 'No', followed by additional reasoning or evidence to support your explanation. 
	  } 
	   } 
	   } 
	 & \\ 
 

    \theutterance \stepcounter{utterance}  

    & \multicolumn{4}{p{0.6\linewidth}}{\cellcolor[rgb]{0.8,1,0.9}{%
	\makecell[{{p{\linewidth}}}]{% 
	  \tt {\tiny [A$\rangle$GM]}  
	 No, it would not be more likely to see xevu if the individual was not rixq.\\ \tt \\ \tt To understand this, let's break down the given information. We have the following conditional probabilities:\\ \tt \\ \tt - P(xevu | not rixq, not zuph) = 46%\\ \tt - P(xevu | not rixq, zuph) = 41%\\ \tt - P(xevu | rixq, not zuph) = 66%\\ \tt - P(xevu | rixq, zuph) = 60%\\ \tt \\ \tt We also have the following conditional probabilities for zuph:\\ \tt \\ \tt - P(zuph | not rixq) = 61%\\ \tt - P(zuph | rixq) = 34%\\ \tt \\ \tt We want to compare the probability of xevu when the individual is not rixq versus when the individual is rixq. To do this, we need to calculate the overall probability of xevu for each group.\\ \tt \\ \tt First, let's calculate the probability of xevu for the group that is not rixq. We can use the law of total probability:\\ \tt \\ \tt P(xevu | not rixq) = P(xevu | not rixq, not zuph) * P(not zuph | not rixq) + P(xevu | not rixq, zuph) * P(zuph | not rixq)\\ \tt = 
	  } 
	   } 
	   } 
	 & & \\ 
 

    \theutterance \stepcounter{utterance}  

    & & & \multicolumn{2}{p{0.3\linewidth}}{\cellcolor[rgb]{0.95,0.95,0.95}{%
	\makecell[{{p{\linewidth}}}]{% 
	  \tt {\tiny [GM$|$GM]}  
	 No 
	  } 
	   } 
	   } 
	 & & \\ 
 

    \theutterance \stepcounter{utterance}  

    & & & \multicolumn{2}{p{0.3\linewidth}}{\cellcolor[rgb]{0.95,0.95,0.95}{%
	\makecell[{{p{\linewidth}}}]{% 
	  \tt {\tiny [GM$|$GM]}  
	 no 
	  } 
	   } 
	   } 
	 & & \\ 
 

    \theutterance \stepcounter{utterance}  

    & & & \multicolumn{2}{p{0.3\linewidth}}{\cellcolor[rgb]{0.95,0.95,0.95}{%
	\makecell[{{p{\linewidth}}}]{% 
	  \tt {\tiny [GM$|$GM]}  
	 game_result = WIN 
	  } 
	   } 
	   } 
	 & & \\ 
 

\end{supertabular}
}

\end{document}
