
\documentclass{article}
\usepackage{colortbl}
\usepackage{makecell}
\usepackage{multirow}
\usepackage{supertabular}

\begin{document}

\newcounter{utterance}

\twocolumn

{ \footnotesize  \setcounter{utterance}{1}
\setlength{\tabcolsep}{0pt}
\begin{supertabular}{c@{$\;$}|p{.15\linewidth}@{}p{.15\linewidth}p{.15\linewidth}p{.15\linewidth}p{.15\linewidth}p{.15\linewidth}}

    \# & $\;$A & \multicolumn{4}{c}{Game Master} & $\;\:$B\\
    \hline 

    \theutterance \stepcounter{utterance}  

    & & \multicolumn{4}{p{0.6\linewidth}}{\cellcolor[rgb]{0.9,0.9,0.9}{%
	\makecell[{{p{\linewidth}}}]{% 
	  \tt {\tiny [A$\langle$GM]}  
	 You are an expert in causal inference. The following question is not a typical commonsense query, but rather a meticulously designed question created by a professor specializing in causal inference, intended to assess the students' mastery of the course content.\\ \tt Imagine a self-contained, hypothetical world with only the following conditions, and without any unmentioned factors or causal relationships: Demand has a direct effect on supply and price. Yield per acre has a direct effect on supply. Supply has a direct effect on price. Demand is unobserved. The overall probability of increased supply is 78%. For situations with reduced supply, the probability of increased price is 44%. For situations with increased supply, the probability of increased price is 81%. Is increased price less likely than reduced price overall?\\ \tt Start your answer with 'Yes' or 'No', followed by additional reasoning or evidence to support your explanation. 
	  } 
	   } 
	   } 
	 & \\ 
 

    \theutterance \stepcounter{utterance}  

    & \multicolumn{4}{p{0.6\linewidth}}{\cellcolor[rgb]{0.8,1,0.9}{%
	\makecell[{{p{\linewidth}}}]{% 
	  \tt {\tiny [A$\rangle$GM]}  
	 No, increased price is not less likely than reduced price overall.\\ \tt \\ \tt To analyze this problem, we can use the given probabilities and the causal relationships between the variables. We are given the following information:\\ \tt \\ \tt 1. The probability of increased supply is 78%.\\ \tt 2. The probability of increased price given reduced supply is 44%.\\ \tt 3. The probability of increased price given increased supply is 81%.\\ \tt \\ \tt We want to compare the overall probability of increased price versus reduced price. To do this, we can use the law of total probability, which states that the overall probability of an event is the sum of the probabilities of the event given each possible cause, weighted by the probability of each cause.\\ \tt \\ \tt Let's define the following events:\\ \tt \\ \tt - I: Increased price\\ \tt - R: Reduced price\\ \tt - S: Increased supply\\ \tt - L: Reduced supply\\ \tt \\ \tt We are given the following conditional probabilities:\\ \tt \\ \tt P(S) = 0.78 (probability of increased supply)\\ \tt P(L) = 1 - P(S) = 0.22 (probability of reduced supply)\\ \tt P(I|L) = 0.44 (probability of increased price given reduced supply)\\ \tt P(I|S) = 0.81 (probability of increased price given increased supply)\\ \tt \\ \tt We want to compare P(I) and P(R). To do this, we can use the law of total probability:\\ \tt \\ \tt P(I) = P(I|S)P(S) + P(I|L)P(L)\\ \tt = 0.81(0 
	  } 
	   } 
	   } 
	 & & \\ 
 

    \theutterance \stepcounter{utterance}  

    & & & \multicolumn{2}{p{0.3\linewidth}}{\cellcolor[rgb]{0.95,0.95,0.95}{%
	\makecell[{{p{\linewidth}}}]{% 
	  \tt {\tiny [GM$|$GM]}  
	 No 
	  } 
	   } 
	   } 
	 & & \\ 
 

    \theutterance \stepcounter{utterance}  

    & & & \multicolumn{2}{p{0.3\linewidth}}{\cellcolor[rgb]{0.95,0.95,0.95}{%
	\makecell[{{p{\linewidth}}}]{% 
	  \tt {\tiny [GM$|$GM]}  
	 no 
	  } 
	   } 
	   } 
	 & & \\ 
 

    \theutterance \stepcounter{utterance}  

    & & & \multicolumn{2}{p{0.3\linewidth}}{\cellcolor[rgb]{0.95,0.95,0.95}{%
	\makecell[{{p{\linewidth}}}]{% 
	  \tt {\tiny [GM$|$GM]}  
	 game_result = WIN 
	  } 
	   } 
	   } 
	 & & \\ 
 

\end{supertabular}
}

\end{document}
