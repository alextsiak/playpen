
\documentclass{article}
\usepackage{colortbl}
\usepackage{makecell}
\usepackage{multirow}
\usepackage{supertabular}

\begin{document}

\newcounter{utterance}

\twocolumn

{ \footnotesize  \setcounter{utterance}{1}
\setlength{\tabcolsep}{0pt}
\begin{supertabular}{c@{$\;$}|p{.15\linewidth}@{}p{.15\linewidth}p{.15\linewidth}p{.15\linewidth}p{.15\linewidth}p{.15\linewidth}}

    \# & $\;$A & \multicolumn{4}{c}{Game Master} & $\;\:$B\\
    \hline 

    \theutterance \stepcounter{utterance}  

    & & \multicolumn{4}{p{0.6\linewidth}}{\cellcolor[rgb]{0.9,0.9,0.9}{%
	\makecell[{{p{\linewidth}}}]{% 
	  \tt {\tiny [A$\langle$GM]}  
	 You are an expert in causal inference. The following question is not a typical commonsense query, but rather a meticulously designed question created by a professor specializing in causal inference, intended to assess the students' mastery of the course content.\\ \tt Imagine a self-contained, hypothetical world with only the following conditions, and without any unmentioned factors or causal relationships: Qwiu has a direct effect on jyka and kwox. Jyka has a direct effect on yupt. Yupt has a direct effect on kwox. Qwiu is unobserved. We know that qwiu causes jyka. jyka causes yupt. qwiu or yupt causes kwox. We observed an individual is not qwiu. Would an individual is not kwox if not jyka instead of jyka?\\ \tt Start your answer with 'Yes' or 'No', followed by additional reasoning or evidence to support your explanation. 
	  } 
	   } 
	   } 
	 & \\ 
 

    \theutterance \stepcounter{utterance}  

    & \multicolumn{4}{p{0.6\linewidth}}{\cellcolor[rgb]{0.8,1,0.9}{%
	\makecell[{{p{\linewidth}}}]{% 
	  \tt {\tiny [A$\rangle$GM]}  
	 No\\ \tt \\ \tt In this hypothetical world, we have the following causal relationships:\\ \tt \\ \tt 1. Qwiu causes jyka.\\ \tt 2. Jyka causes yupt.\\ \tt 3. Yupt causes kwox.\\ \tt 4. Qwiu or yupt causes kwox.\\ \tt \\ \tt We are given that the individual is not qwiu. This implies that the causal effect of qwiu on kwox is absent.\\ \tt \\ \tt Now, we need to consider the scenario where the individual is not jyka instead of jyka. If jyka is absent, then the causal effect of jyka on yupt is also absent. However, since yupt causes kwox, the absence of yupt would imply that the causal effect of yupt on kwox is also absent.\\ \tt \\ \tt Since qwiu or yupt causes kwox, and both qwiu and yupt are absent, it is still possible that kwox is caused by some other factor (not mentioned in the problem). However, given the information provided, we cannot conclude that the individual is not kwox if not jyka instead of jyka.\\ \tt \\ \tt In other words, the absence of jyka does not necessarily imply the absence of kwox, because the causal effect of qwiu on kwox is absent, and the causal effect of yupt on kwox is also absent. Therefore, the answer is No. 
	  } 
	   } 
	   } 
	 & & \\ 
 

    \theutterance \stepcounter{utterance}  

    & & & \multicolumn{2}{p{0.3\linewidth}}{\cellcolor[rgb]{0.95,0.95,0.95}{%
	\makecell[{{p{\linewidth}}}]{% 
	  \tt {\tiny [GM$|$GM]}  
	 No 
	  } 
	   } 
	   } 
	 & & \\ 
 

    \theutterance \stepcounter{utterance}  

    & & & \multicolumn{2}{p{0.3\linewidth}}{\cellcolor[rgb]{0.95,0.95,0.95}{%
	\makecell[{{p{\linewidth}}}]{% 
	  \tt {\tiny [GM$|$GM]}  
	 yes 
	  } 
	   } 
	   } 
	 & & \\ 
 

    \theutterance \stepcounter{utterance}  

    & & & \multicolumn{2}{p{0.3\linewidth}}{\cellcolor[rgb]{0.95,0.95,0.95}{%
	\makecell[{{p{\linewidth}}}]{% 
	  \tt {\tiny [GM$|$GM]}  
	 game_result = LOSE 
	  } 
	   } 
	   } 
	 & & \\ 
 

\end{supertabular}
}

\end{document}
