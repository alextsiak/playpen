
\documentclass{article}
\usepackage{colortbl}
\usepackage{makecell}
\usepackage{multirow}
\usepackage{supertabular}

\begin{document}

\newcounter{utterance}

\twocolumn

{ \footnotesize  \setcounter{utterance}{1}
\setlength{\tabcolsep}{0pt}
\begin{supertabular}{c@{$\;$}|p{.15\linewidth}@{}p{.15\linewidth}p{.15\linewidth}p{.15\linewidth}p{.15\linewidth}p{.15\linewidth}}

    \# & $\;$A & \multicolumn{4}{c}{Game Master} & $\;\:$B\\
    \hline 

    \theutterance \stepcounter{utterance}  

    & & \multicolumn{4}{p{0.6\linewidth}}{\cellcolor[rgb]{0.9,0.9,0.9}{%
	\makecell[{{p{\linewidth}}}]{% 
	  \tt {\tiny [A$\langle$GM]}  
	 You are an expert in causal inference. The following question is not a typical commonsense query, but rather a meticulously designed question created by a professor specializing in causal inference, intended to assess the students' mastery of the course content.\\ \tt Imagine a self-contained, hypothetical world with only the following conditions, and without any unmentioned factors or causal relationships: CEO has a direct effect on director and manager. Manager has a direct effect on employee. Director has a direct effect on employee. For CEOs who fire employees and managers who don't sign termination letters, the probability of employee being fired is 24%. For CEOs who fire employees and managers who sign termination letters, the probability of employee being fired is 60%. For CEOs who fire employees and managers who don't sign termination letters, the probability of employee being fired is 39%. For CEOs who fire employees and managers who sign termination letters, the probability of employee being fired is 74%. The overall probability of CEO's decision to fire the employee is 18%. Will manager signing the termination letter increase the chance of employee being fired?\\ \tt Start your answer with 'Yes' or 'No', followed by additional reasoning or evidence to support your explanation. 
	  } 
	   } 
	   } 
	 & \\ 
 

    \theutterance \stepcounter{utterance}  

    & \multicolumn{4}{p{0.6\linewidth}}{\cellcolor[rgb]{0.8,1,0.9}{%
	\makecell[{{p{\linewidth}}}]{% 
	  \tt {\tiny [A$\rangle$GM]}  
	 Yes, manager signing the termination letter will increase the chance of employee being fired.\\ \tt \\ \tt To analyze this problem, we can use the potential outcomes framework. Let's define the following variables:\\ \tt \\ \tt - Y: Employee being fired\\ \tt - X: Manager signing the termination letter\\ \tt - W: CEO's decision to fire the employee\\ \tt \\ \tt We are given the following conditional probabilities:\\ \tt \\ \tt - P(Y|W=1, X=0) = 24% (CEO fires, manager doesn't sign)\\ \tt - P(Y|W=1, X=1) = 60% (CEO fires, manager signs)\\ \tt - P(Y|W=0, X=0) = 39% (CEO doesn't fire, manager doesn't sign)\\ \tt - P(Y|W=0, X=1) = 74% (CEO doesn't fire, manager signs)\\ \tt \\ \tt We are also given the overall probability of CEO's decision to fire the employee:\\ \tt \\ \tt - P(W=1) = 18%\\ \tt \\ \tt We want to estimate the effect of manager signing the termination letter on the probability of employee being fired. To do this, we can use the counterfactual framework and estimate the average treatment effect (ATE) of manager signing on employee being fired.\\ \tt \\ \tt First, let's calculate the probability of employee being fired when the manager signs, given that the CEO fires:\\ \tt \\ \tt P(Y|W=1, X=1) = 0.18 * 0.60 + 0.82 * 0.74 ≈ 
	  } 
	   } 
	   } 
	 & & \\ 
 

    \theutterance \stepcounter{utterance}  

    & & & \multicolumn{2}{p{0.3\linewidth}}{\cellcolor[rgb]{0.95,0.95,0.95}{%
	\makecell[{{p{\linewidth}}}]{% 
	  \tt {\tiny [GM$|$GM]}  
	 Yes 
	  } 
	   } 
	   } 
	 & & \\ 
 

    \theutterance \stepcounter{utterance}  

    & & & \multicolumn{2}{p{0.3\linewidth}}{\cellcolor[rgb]{0.95,0.95,0.95}{%
	\makecell[{{p{\linewidth}}}]{% 
	  \tt {\tiny [GM$|$GM]}  
	 yes 
	  } 
	   } 
	   } 
	 & & \\ 
 

    \theutterance \stepcounter{utterance}  

    & & & \multicolumn{2}{p{0.3\linewidth}}{\cellcolor[rgb]{0.95,0.95,0.95}{%
	\makecell[{{p{\linewidth}}}]{% 
	  \tt {\tiny [GM$|$GM]}  
	 game_result = WIN 
	  } 
	   } 
	   } 
	 & & \\ 
 

\end{supertabular}
}

\end{document}
