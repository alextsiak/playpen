
\documentclass{article}
\usepackage{colortbl}
\usepackage{makecell}
\usepackage{multirow}
\usepackage{supertabular}

\begin{document}

\newcounter{utterance}

\twocolumn

{ \footnotesize  \setcounter{utterance}{1}
\setlength{\tabcolsep}{0pt}
\begin{supertabular}{c@{$\;$}|p{.15\linewidth}@{}p{.15\linewidth}p{.15\linewidth}p{.15\linewidth}p{.15\linewidth}p{.15\linewidth}}

    \# & $\;$A & \multicolumn{4}{c}{Game Master} & $\;\:$B\\
    \hline 

    \theutterance \stepcounter{utterance}  

    & & \multicolumn{4}{p{0.6\linewidth}}{\cellcolor[rgb]{0.9,0.9,0.9}{%
	\makecell[{{p{\linewidth}}}]{% 
	  \tt {\tiny [A$\langle$GM]}  
	 You are an expert in causal inference. The following question is not a typical commonsense query, but rather a meticulously designed question created by a professor specializing in causal inference, intended to assess the students' mastery of the course content.\\ \tt Imagine a self-contained, hypothetical world with only the following conditions, and without any unmentioned factors or causal relationships: Jyka has a direct effect on lirg. Hwax has a direct effect on lirg. The overall probability of jyka is 14%. For those who are not jyka, the probability of lirg is 71%. For those who are jyka, the probability of lirg is 66%. Is lirg less likely than not lirg overall?\\ \tt Start your answer with 'Yes' or 'No', followed by additional reasoning or evidence to support your explanation. 
	  } 
	   } 
	   } 
	 & \\ 
 

    \theutterance \stepcounter{utterance}  

    & \multicolumn{4}{p{0.6\linewidth}}{\cellcolor[rgb]{0.8,1,0.9}{%
	\makecell[{{p{\linewidth}}}]{% 
	  \tt {\tiny [A$\rangle$GM]}  
	 No\\ \tt \\ \tt To answer this question, we need to apply the principles of causal inference, specifically the concept of conditional probability and the potential outcome framework. \\ \tt \\ \tt Let's define the following variables:\\ \tt \\ \tt - Jyka (J): a binary variable indicating whether an individual is Jyka (1) or not (0)\\ \tt - Lirg (L): a binary variable indicating whether an individual has Lirg (1) or not (0)\\ \tt \\ \tt We are given the following information:\\ \tt \\ \tt 1. P(J = 1) = 0.14 (the probability of being Jyka is 14%)\\ \tt 2. P(L = 1 | J = 0) = 0.71 (for those who are not Jyka, the probability of Lirg is 71%)\\ \tt 3. P(L = 1 | J = 1) = 0.66 (for those who are Jyka, the probability of Lirg is 66%)\\ \tt \\ \tt We want to determine whether Lirg is less likely than not Lirg overall, which is equivalent to asking whether P(L = 1) < 0.5.\\ \tt \\ \tt To calculate the overall probability of Lirg, we can use the law of total probability:\\ \tt \\ \tt P(L = 1) = P(L = 1 | J = 1) \* P(J = 1) + P(L = 1 | J = 0) \* P(J = 0)\\ \tt \\ \tt We know P 
	  } 
	   } 
	   } 
	 & & \\ 
 

    \theutterance \stepcounter{utterance}  

    & & & \multicolumn{2}{p{0.3\linewidth}}{\cellcolor[rgb]{0.95,0.95,0.95}{%
	\makecell[{{p{\linewidth}}}]{% 
	  \tt {\tiny [GM$|$GM]}  
	 No 
	  } 
	   } 
	   } 
	 & & \\ 
 

    \theutterance \stepcounter{utterance}  

    & & & \multicolumn{2}{p{0.3\linewidth}}{\cellcolor[rgb]{0.95,0.95,0.95}{%
	\makecell[{{p{\linewidth}}}]{% 
	  \tt {\tiny [GM$|$GM]}  
	 no 
	  } 
	   } 
	   } 
	 & & \\ 
 

    \theutterance \stepcounter{utterance}  

    & & & \multicolumn{2}{p{0.3\linewidth}}{\cellcolor[rgb]{0.95,0.95,0.95}{%
	\makecell[{{p{\linewidth}}}]{% 
	  \tt {\tiny [GM$|$GM]}  
	 game_result = WIN 
	  } 
	   } 
	   } 
	 & & \\ 
 

\end{supertabular}
}

\end{document}
