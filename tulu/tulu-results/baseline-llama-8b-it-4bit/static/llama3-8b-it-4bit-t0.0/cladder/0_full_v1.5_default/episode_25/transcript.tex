
\documentclass{article}
\usepackage{colortbl}
\usepackage{makecell}
\usepackage{multirow}
\usepackage{supertabular}

\begin{document}

\newcounter{utterance}

\twocolumn

{ \footnotesize  \setcounter{utterance}{1}
\setlength{\tabcolsep}{0pt}
\begin{supertabular}{c@{$\;$}|p{.15\linewidth}@{}p{.15\linewidth}p{.15\linewidth}p{.15\linewidth}p{.15\linewidth}p{.15\linewidth}}

    \# & $\;$A & \multicolumn{4}{c}{Game Master} & $\;\:$B\\
    \hline 

    \theutterance \stepcounter{utterance}  

    & & \multicolumn{4}{p{0.6\linewidth}}{\cellcolor[rgb]{0.9,0.9,0.9}{%
	\makecell[{{p{\linewidth}}}]{% 
	  \tt {\tiny [A$\langle$GM]}  
	 You are an expert in causal inference. The following question is not a typical commonsense query, but rather a meticulously designed question created by a professor specializing in causal inference, intended to assess the students' mastery of the course content.\\ \tt Imagine a self-contained, hypothetical world with only the following conditions, and without any unmentioned factors or causal relationships: Gender has a direct effect on department competitiveness and admission status. Residency status has a direct effect on department competitiveness and admission status. Department competitiveness has a direct effect on admission status. Residency status is unobserved. For individuals who are not male, the probability of admission acceptance is 40%. For individuals who are male, the probability of admission acceptance is 32%. Will male gender increase the chance of admission acceptance?\\ \tt Start your answer with 'Yes' or 'No', followed by additional reasoning or evidence to support your explanation. 
	  } 
	   } 
	   } 
	 & \\ 
 

    \theutterance \stepcounter{utterance}  

    & \multicolumn{4}{p{0.6\linewidth}}{\cellcolor[rgb]{0.8,1,0.9}{%
	\makecell[{{p{\linewidth}}}]{% 
	  \tt {\tiny [A$\rangle$GM]}  
	 No\\ \tt \\ \tt In this hypothetical world, the causal relationships are as follows:\\ \tt \\ \tt 1. Gender has a direct effect on department competitiveness and admission status.\\ \tt 2. Residency status has a direct effect on department competitiveness and admission status.\\ \tt 3. Department competitiveness has a direct effect on admission status.\\ \tt \\ \tt Given that Residency status is unobserved, we cannot directly measure its effect on department competitiveness and admission status. However, we can infer its impact through the observed relationships.\\ \tt \\ \tt Since department competitiveness has a direct effect on admission status, and gender has a direct effect on department competitiveness, we can conclude that gender indirectly affects admission status through department competitiveness.\\ \tt \\ \tt However, the given probabilities of admission acceptance (40% for non-male individuals and 32% for male individuals) do not provide sufficient information to determine whether male gender increases the chance of admission acceptance. \\ \tt \\ \tt To assess the effect of male gender on admission acceptance, we need to consider the potential confounding effect of residency status. If residency status is positively correlated with male gender (e.g., males are more likely to be residents), then the observed lower probability of admission acceptance for males (32%) might be due to the lower residency status, which in turn affects department competitiveness and admission status.\\ \tt \\ \tt Without additional information about the relationship between residency status and male gender, we cannot conclude that male gender directly increases the chance of admission acceptance. Therefore, the answer is 'No', but with the caveat that the true effect of male gender on admission acceptance might be confounded by the un 
	  } 
	   } 
	   } 
	 & & \\ 
 

    \theutterance \stepcounter{utterance}  

    & & & \multicolumn{2}{p{0.3\linewidth}}{\cellcolor[rgb]{0.95,0.95,0.95}{%
	\makecell[{{p{\linewidth}}}]{% 
	  \tt {\tiny [GM$|$GM]}  
	 No 
	  } 
	   } 
	   } 
	 & & \\ 
 

    \theutterance \stepcounter{utterance}  

    & & & \multicolumn{2}{p{0.3\linewidth}}{\cellcolor[rgb]{0.95,0.95,0.95}{%
	\makecell[{{p{\linewidth}}}]{% 
	  \tt {\tiny [GM$|$GM]}  
	 no 
	  } 
	   } 
	   } 
	 & & \\ 
 

    \theutterance \stepcounter{utterance}  

    & & & \multicolumn{2}{p{0.3\linewidth}}{\cellcolor[rgb]{0.95,0.95,0.95}{%
	\makecell[{{p{\linewidth}}}]{% 
	  \tt {\tiny [GM$|$GM]}  
	 game_result = WIN 
	  } 
	   } 
	   } 
	 & & \\ 
 

\end{supertabular}
}

\end{document}
