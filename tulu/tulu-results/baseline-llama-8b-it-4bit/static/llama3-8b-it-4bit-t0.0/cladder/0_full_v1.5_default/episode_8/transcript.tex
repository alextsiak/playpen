
\documentclass{article}
\usepackage{colortbl}
\usepackage{makecell}
\usepackage{multirow}
\usepackage{supertabular}

\begin{document}

\newcounter{utterance}

\twocolumn

{ \footnotesize  \setcounter{utterance}{1}
\setlength{\tabcolsep}{0pt}
\begin{supertabular}{c@{$\;$}|p{.15\linewidth}@{}p{.15\linewidth}p{.15\linewidth}p{.15\linewidth}p{.15\linewidth}p{.15\linewidth}}

    \# & $\;$A & \multicolumn{4}{c}{Game Master} & $\;\:$B\\
    \hline 

    \theutterance \stepcounter{utterance}  

    & & \multicolumn{4}{p{0.6\linewidth}}{\cellcolor[rgb]{0.9,0.9,0.9}{%
	\makecell[{{p{\linewidth}}}]{% 
	  \tt {\tiny [A$\langle$GM]}  
	 You are an expert in causal inference. The following question is not a typical commonsense query, but rather a meticulously designed question created by a professor specializing in causal inference, intended to assess the students' mastery of the course content.\\ \tt Imagine a self-contained, hypothetical world with only the following conditions, and without any unmentioned factors or causal relationships: Gender has a direct effect on department competitiveness and admission status. Department competitiveness has a direct effect on admission status. The overall probability of male gender is 54%. For individuals who are not male, the probability of admission acceptance is 49%. For individuals who are male, the probability of admission acceptance is 24%. Is admission acceptance less likely than admission rejection overall?\\ \tt Start your answer with 'Yes' or 'No', followed by additional reasoning or evidence to support your explanation. 
	  } 
	   } 
	   } 
	 & \\ 
 

    \theutterance \stepcounter{utterance}  

    & \multicolumn{4}{p{0.6\linewidth}}{\cellcolor[rgb]{0.8,1,0.9}{%
	\makecell[{{p{\linewidth}}}]{% 
	  \tt {\tiny [A$\rangle$GM]}  
	 No, admission acceptance is not less likely than admission rejection overall.\\ \tt \\ \tt To understand this, let's break down the given information:\\ \tt \\ \tt 1. The probability of male gender is 54%.\\ \tt 2. For individuals who are not male (i.e., female), the probability of admission acceptance is 49%.\\ \tt 3. For individuals who are male, the probability of admission acceptance is 24%.\\ \tt \\ \tt We can calculate the overall probability of admission acceptance by considering the probabilities of being male and female, and the corresponding probabilities of admission acceptance for each gender.\\ \tt \\ \tt Let P(A) be the overall probability of admission acceptance. We can express P(A) as a weighted average of the probabilities of admission acceptance for males and females, where the weights are the probabilities of being male and female:\\ \tt \\ \tt P(A) = 0.54 \* P(A|male) + 0.46 \* P(A|female)\\ \tt \\ \tt where P(A|male) is the probability of admission acceptance given that the individual is male, and P(A|female) is the probability of admission acceptance given that the individual is female.\\ \tt \\ \tt We are given that P(A|male) = 0.24 and P(A|female) = 0.49. Substituting these values into the equation above, we get:\\ \tt \\ \tt P(A) = 0.54 \* 0.24 + 0.46 \* 0.49\\ \tt = 0.1304 + 0.2276\\ \tt = 0.358 
	  } 
	   } 
	   } 
	 & & \\ 
 

    \theutterance \stepcounter{utterance}  

    & & & \multicolumn{2}{p{0.3\linewidth}}{\cellcolor[rgb]{0.95,0.95,0.95}{%
	\makecell[{{p{\linewidth}}}]{% 
	  \tt {\tiny [GM$|$GM]}  
	 No 
	  } 
	   } 
	   } 
	 & & \\ 
 

    \theutterance \stepcounter{utterance}  

    & & & \multicolumn{2}{p{0.3\linewidth}}{\cellcolor[rgb]{0.95,0.95,0.95}{%
	\makecell[{{p{\linewidth}}}]{% 
	  \tt {\tiny [GM$|$GM]}  
	 yes 
	  } 
	   } 
	   } 
	 & & \\ 
 

    \theutterance \stepcounter{utterance}  

    & & & \multicolumn{2}{p{0.3\linewidth}}{\cellcolor[rgb]{0.95,0.95,0.95}{%
	\makecell[{{p{\linewidth}}}]{% 
	  \tt {\tiny [GM$|$GM]}  
	 game_result = LOSE 
	  } 
	   } 
	   } 
	 & & \\ 
 

\end{supertabular}
}

\end{document}
