
\documentclass{article}
\usepackage{colortbl}
\usepackage{makecell}
\usepackage{multirow}
\usepackage{supertabular}

\begin{document}

\newcounter{utterance}

\twocolumn

{ \footnotesize  \setcounter{utterance}{1}
\setlength{\tabcolsep}{0pt}
\begin{supertabular}{c@{$\;$}|p{.15\linewidth}@{}p{.15\linewidth}p{.15\linewidth}p{.15\linewidth}p{.15\linewidth}p{.15\linewidth}}

    \# & $\;$A & \multicolumn{4}{c}{Game Master} & $\;\:$B\\
    \hline 

    \theutterance \stepcounter{utterance}  

    & & \multicolumn{4}{p{0.6\linewidth}}{\cellcolor[rgb]{0.9,0.9,0.9}{%
	\makecell[{{p{\linewidth}}}]{% 
	  \tt {\tiny [A$\langle$GM]}  
	 You are an expert in causal inference. The following question is not a typical commonsense query, but rather a meticulously designed question created by a professor specializing in causal inference, intended to assess the students' mastery of the course content.\\ \tt Imagine a self-contained, hypothetical world with only the following conditions, and without any unmentioned factors or causal relationships: CEO has a direct effect on director and manager. Manager has a direct effect on employee. Director has a direct effect on employee. The overall probability of manager signing the termination letter is 35%. For managers who don't sign termination letters, the probability of employee being fired is 76%. For managers who sign termination letters, the probability of employee being fired is 42%. Is employee being fired more likely than employee not being fired overall?\\ \tt Start your answer with 'Yes' or 'No', followed by additional reasoning or evidence to support your explanation. 
	  } 
	   } 
	   } 
	 & \\ 
 

    \theutterance \stepcounter{utterance}  

    & \multicolumn{4}{p{0.6\linewidth}}{\cellcolor[rgb]{0.8,1,0.9}{%
	\makecell[{{p{\linewidth}}}]{% 
	  \tt {\tiny [A$\rangle$GM]}  
	 No, employee being fired is not more likely than employee not being fired overall.\\ \tt \\ \tt To understand this, let's break down the information given:\\ \tt \\ \tt 1. The probability of a manager signing the termination letter is 35%.\\ \tt 2. For managers who don't sign termination letters, the probability of employee being fired is 76%.\\ \tt 3. For managers who sign termination letters, the probability of employee being fired is 42%.\\ \tt \\ \tt We can calculate the overall probability of employee being fired as follows:\\ \tt \\ \tt - Probability of manager signing the termination letter and employee being fired: 0.35 * 0.42 = 0.147\\ \tt - Probability of manager not signing the termination letter and employee being fired: 0.65 * 0.76 = 0.494\\ \tt - Total probability of employee being fired: 0.147 + 0.494 = 0.641\\ \tt \\ \tt However, to determine if employee being fired is more likely than employee not being fired overall, we need to compare the probability of employee being fired to the probability of employee not being fired.\\ \tt \\ \tt - Probability of employee not being fired: 1 - 0.641 = 0.359\\ \tt \\ \tt Since 0.359 (probability of employee not being fired) is greater than 0.641 (probability of employee being fired), employee being fired is not more likely than employee not being fired overall.\\ \tt \\ \tt This result might seem counterintuitive at first, but it's due to the fact that the probability of a manager signing the 
	  } 
	   } 
	   } 
	 & & \\ 
 

    \theutterance \stepcounter{utterance}  

    & & & \multicolumn{2}{p{0.3\linewidth}}{\cellcolor[rgb]{0.95,0.95,0.95}{%
	\makecell[{{p{\linewidth}}}]{% 
	  \tt {\tiny [GM$|$GM]}  
	 No 
	  } 
	   } 
	   } 
	 & & \\ 
 

    \theutterance \stepcounter{utterance}  

    & & & \multicolumn{2}{p{0.3\linewidth}}{\cellcolor[rgb]{0.95,0.95,0.95}{%
	\makecell[{{p{\linewidth}}}]{% 
	  \tt {\tiny [GM$|$GM]}  
	 yes 
	  } 
	   } 
	   } 
	 & & \\ 
 

    \theutterance \stepcounter{utterance}  

    & & & \multicolumn{2}{p{0.3\linewidth}}{\cellcolor[rgb]{0.95,0.95,0.95}{%
	\makecell[{{p{\linewidth}}}]{% 
	  \tt {\tiny [GM$|$GM]}  
	 game_result = LOSE 
	  } 
	   } 
	   } 
	 & & \\ 
 

\end{supertabular}
}

\end{document}
