
\documentclass{article}
\usepackage{colortbl}
\usepackage{makecell}
\usepackage{multirow}
\usepackage{supertabular}

\begin{document}

\newcounter{utterance}

\twocolumn

{ \footnotesize  \setcounter{utterance}{1}
\setlength{\tabcolsep}{0pt}
\begin{supertabular}{c@{$\;$}|p{.15\linewidth}@{}p{.15\linewidth}p{.15\linewidth}p{.15\linewidth}p{.15\linewidth}p{.15\linewidth}}

    \# & $\;$A & \multicolumn{4}{c}{Game Master} & $\;\:$B\\
    \hline 

    \theutterance \stepcounter{utterance}  

    & & \multicolumn{4}{p{0.6\linewidth}}{\cellcolor[rgb]{0.9,0.9,0.9}{%
	\makecell[{{p{\linewidth}}}]{% 
	  \tt {\tiny [A$\langle$GM]}  
	 You are an expert in causal inference. The following question is not a typical commonsense query, but rather a meticulously designed question created by a professor specializing in causal inference, intended to assess the students' mastery of the course content.\\ \tt Imagine a self-contained, hypothetical world with only the following conditions, and without any unmentioned factors or causal relationships: Zuph has a direct effect on rixq and xevu. Rixq has a direct effect on xevu. For those who are not zuph and are not rixq, the probability of xevu is 15%. For those who are not zuph and are rixq, the probability of xevu is 50%. For those who are zuph and are not rixq, the probability of xevu is 32%. For those who are zuph and are rixq, the probability of xevu is 85%. The overall probability of zuph is 84%. Will rixq increase the chance of xevu?\\ \tt Start your answer with 'Yes' or 'No', followed by additional reasoning or evidence to support your explanation. 
	  } 
	   } 
	   } 
	 & \\ 
 

    \theutterance \stepcounter{utterance}  

    & \multicolumn{4}{p{0.6\linewidth}}{\cellcolor[rgb]{0.8,1,0.9}{%
	\makecell[{{p{\linewidth}}}]{% 
	  \tt {\tiny [A$\rangle$GM]}  
	 No, rixq will not increase the chance of xevu.\\ \tt \\ \tt To understand this, let's break down the given probabilities and apply the concept of causal inference, specifically the potential outcome (PO) and natural direct effect (NDE) frameworks. \\ \tt \\ \tt First, we need to calculate the overall probability of xevu. We can do this by considering the four groups:\\ \tt \\ \tt 1. Not zuph and not rixq: 15% (probability of xevu)\\ \tt 2. Not zuph and rixq: 50% (probability of xevu)\\ \tt 3. Zuph and not rixq: 32% (probability of xevu)\\ \tt 4. Zuph and rixq: 85% (probability of xevu)\\ \tt \\ \tt The overall probability of xevu is a weighted average of these probabilities, where the weights are the probabilities of each group. Since the overall probability of zuph is 84%, we can assume that the probability of not being zuph is 16% (1 - 0.84).\\ \tt \\ \tt Let's calculate the weighted average:\\ \tt \\ \tt Weighted average = (0.16 * 15%) + (0.16 * 50%) + (0.84 * 32%) + (0.84 * 85%)\\ \tt Weighted average = (0.16 * 0.15) + (0.16 * 0.5) + (0.84 * 0 
	  } 
	   } 
	   } 
	 & & \\ 
 

    \theutterance \stepcounter{utterance}  

    & & & \multicolumn{2}{p{0.3\linewidth}}{\cellcolor[rgb]{0.95,0.95,0.95}{%
	\makecell[{{p{\linewidth}}}]{% 
	  \tt {\tiny [GM$|$GM]}  
	 No 
	  } 
	   } 
	   } 
	 & & \\ 
 

    \theutterance \stepcounter{utterance}  

    & & & \multicolumn{2}{p{0.3\linewidth}}{\cellcolor[rgb]{0.95,0.95,0.95}{%
	\makecell[{{p{\linewidth}}}]{% 
	  \tt {\tiny [GM$|$GM]}  
	 yes 
	  } 
	   } 
	   } 
	 & & \\ 
 

    \theutterance \stepcounter{utterance}  

    & & & \multicolumn{2}{p{0.3\linewidth}}{\cellcolor[rgb]{0.95,0.95,0.95}{%
	\makecell[{{p{\linewidth}}}]{% 
	  \tt {\tiny [GM$|$GM]}  
	 game_result = LOSE 
	  } 
	   } 
	   } 
	 & & \\ 
 

\end{supertabular}
}

\end{document}
