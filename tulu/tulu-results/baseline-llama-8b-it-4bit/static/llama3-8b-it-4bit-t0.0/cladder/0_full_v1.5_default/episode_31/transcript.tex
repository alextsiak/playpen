
\documentclass{article}
\usepackage{colortbl}
\usepackage{makecell}
\usepackage{multirow}
\usepackage{supertabular}

\begin{document}

\newcounter{utterance}

\twocolumn

{ \footnotesize  \setcounter{utterance}{1}
\setlength{\tabcolsep}{0pt}
\begin{supertabular}{c@{$\;$}|p{.15\linewidth}@{}p{.15\linewidth}p{.15\linewidth}p{.15\linewidth}p{.15\linewidth}p{.15\linewidth}}

    \# & $\;$A & \multicolumn{4}{c}{Game Master} & $\;\:$B\\
    \hline 

    \theutterance \stepcounter{utterance}  

    & & \multicolumn{4}{p{0.6\linewidth}}{\cellcolor[rgb]{0.9,0.9,0.9}{%
	\makecell[{{p{\linewidth}}}]{% 
	  \tt {\tiny [A$\langle$GM]}  
	 You are an expert in causal inference. The following question is not a typical commonsense query, but rather a meticulously designed question created by a professor specializing in causal inference, intended to assess the students' mastery of the course content.\\ \tt Imagine a self-contained, hypothetical world with only the following conditions, and without any unmentioned factors or causal relationships: Gene has a direct effect on smoking and lung cancer. Pollution has a direct effect on smoking and lung cancer. Smoking has a direct effect on lung cancer. Pollution is unobserved. For people with nonsmoking genes and nonsmokers, the probability of lung cancer is 56%. For people with nonsmoking genes and smokers, the probability of lung cancer is 42%. For people with smoking genes and nonsmokers, the probability of lung cancer is 60%. For people with smoking genes and smokers, the probability of lung cancer is 40%. For people with nonsmoking genes and with low pollution, the probability of smoking is 76%. For people with nonsmoking genes and with high pollution, the probability of smoking is 48%. For people with smoking genes and with low pollution, the probability of smoking is 46%. For people with smoking genes and with high pollution, the probability of smoking is 21%. The overall probability of high pollution is 41%. Does gene positively affect lung cancer through smoking?\\ \tt Start your answer with 'Yes' or 'No', followed by additional reasoning or evidence to support your explanation. 
	  } 
	   } 
	   } 
	 & \\ 
 

    \theutterance \stepcounter{utterance}  

    & \multicolumn{4}{p{0.6\linewidth}}{\cellcolor[rgb]{0.8,1,0.9}{%
	\makecell[{{p{\linewidth}}}]{% 
	  \tt {\tiny [A$\rangle$GM]}  
	 No, the gene does not positively affect lung cancer through smoking.\\ \tt \\ \tt To assess this, we can use the potential outcomes framework, which is a common tool in causal inference. We can define the following potential outcomes:\\ \tt \\ \tt - L: Lung cancer\\ \tt - S: Smoking\\ \tt - G: Gene (smoking or nonsmoking)\\ \tt - P: Pollution (high or low)\\ \tt \\ \tt We are given the following conditional probabilities:\\ \tt \\ \tt - P(L|G=non-smoking, S=nonsmoking) = 0.56\\ \tt - P(L|G=non-smoking, S=smoking) = 0.42\\ \tt - P(L|G=smoking, S=nonsmoking) = 0.60\\ \tt - P(L|G=smoking, S=smoking) = 0.40\\ \tt - P(S|G=non-smoking, P=low) = 0.76\\ \tt - P(S|G=non-smoking, P=high) = 0.48\\ \tt - P(S|G=smoking, P=low) = 0.46\\ \tt - P(S|G=smoking, P=high) = 0.21\\ \tt - P(P=high) = 0.41\\ \tt \\ \tt We can use the counterfactuals to assess the effect of the gene on lung cancer through smoking. Let's define the following counterfactuals:\\ \tt \\ \tt - L(G, S) = the potential outcome of lung cancer 
	  } 
	   } 
	   } 
	 & & \\ 
 

    \theutterance \stepcounter{utterance}  

    & & & \multicolumn{2}{p{0.3\linewidth}}{\cellcolor[rgb]{0.95,0.95,0.95}{%
	\makecell[{{p{\linewidth}}}]{% 
	  \tt {\tiny [GM$|$GM]}  
	 No 
	  } 
	   } 
	   } 
	 & & \\ 
 

    \theutterance \stepcounter{utterance}  

    & & & \multicolumn{2}{p{0.3\linewidth}}{\cellcolor[rgb]{0.95,0.95,0.95}{%
	\makecell[{{p{\linewidth}}}]{% 
	  \tt {\tiny [GM$|$GM]}  
	 yes 
	  } 
	   } 
	   } 
	 & & \\ 
 

    \theutterance \stepcounter{utterance}  

    & & & \multicolumn{2}{p{0.3\linewidth}}{\cellcolor[rgb]{0.95,0.95,0.95}{%
	\makecell[{{p{\linewidth}}}]{% 
	  \tt {\tiny [GM$|$GM]}  
	 game_result = LOSE 
	  } 
	   } 
	   } 
	 & & \\ 
 

\end{supertabular}
}

\end{document}
