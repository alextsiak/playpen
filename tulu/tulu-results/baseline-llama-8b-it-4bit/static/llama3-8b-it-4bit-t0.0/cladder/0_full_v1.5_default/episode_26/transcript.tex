
\documentclass{article}
\usepackage{colortbl}
\usepackage{makecell}
\usepackage{multirow}
\usepackage{supertabular}

\begin{document}

\newcounter{utterance}

\twocolumn

{ \footnotesize  \setcounter{utterance}{1}
\setlength{\tabcolsep}{0pt}
\begin{supertabular}{c@{$\;$}|p{.15\linewidth}@{}p{.15\linewidth}p{.15\linewidth}p{.15\linewidth}p{.15\linewidth}p{.15\linewidth}}

    \# & $\;$A & \multicolumn{4}{c}{Game Master} & $\;\:$B\\
    \hline 

    \theutterance \stepcounter{utterance}  

    & & \multicolumn{4}{p{0.6\linewidth}}{\cellcolor[rgb]{0.9,0.9,0.9}{%
	\makecell[{{p{\linewidth}}}]{% 
	  \tt {\tiny [A$\langle$GM]}  
	 You are an expert in causal inference. The following question is not a typical commonsense query, but rather a meticulously designed question created by a professor specializing in causal inference, intended to assess the students' mastery of the course content.\\ \tt Imagine a self-contained, hypothetical world with only the following conditions, and without any unmentioned factors or causal relationships: Gender has a direct effect on department competitiveness and admission status. Residency status has a direct effect on department competitiveness and admission status. Department competitiveness has a direct effect on admission status. Residency status is unobserved. The overall probability of male gender is 13%. For individuals who are not male, the probability of admission acceptance is 61%. For individuals who are male, the probability of admission acceptance is 65%. Is admission acceptance less likely than admission rejection overall?\\ \tt Start your answer with 'Yes' or 'No', followed by additional reasoning or evidence to support your explanation. 
	  } 
	   } 
	   } 
	 & \\ 
 

    \theutterance \stepcounter{utterance}  

    & \multicolumn{4}{p{0.6\linewidth}}{\cellcolor[rgb]{0.8,1,0.9}{%
	\makecell[{{p{\linewidth}}}]{% 
	  \tt {\tiny [A$\rangle$GM]}  
	 No, admission acceptance is not less likely than admission rejection overall.\\ \tt \\ \tt To understand this, let's break down the given information:\\ \tt \\ \tt 1. Gender has a direct effect on department competitiveness and admission status.\\ \tt 2. Residency status has a direct effect on department competitiveness and admission status.\\ \tt 3. Department competitiveness has a direct effect on admission status.\\ \tt 4. Residency status is unobserved.\\ \tt 5. The overall probability of male gender is 13%.\\ \tt 6. For individuals who are not male, the probability of admission acceptance is 61%.\\ \tt 7. For individuals who are male, the probability of admission acceptance is 65%.\\ \tt \\ \tt We are asked to determine if admission acceptance is less likely than admission rejection overall. To do this, we need to consider the potential effects of the unobserved residency status on the observed outcomes.\\ \tt \\ \tt Since residency status is unobserved, we can use the potential outcomes framework to analyze the situation. Let's denote the potential outcomes as follows:\\ \tt \\ \tt - A: admission acceptance\\ \tt - R: admission rejection\\ \tt \\ \tt We can define the following potential outcomes:\\ \tt \\ \tt - P(A|R=0): probability of admission acceptance given residency status is 0 (i.e., a specific unobserved value)\\ \tt - P(A|R=1): probability of admission acceptance given residency status is 1 (i.e., another specific unobserved value)\\ \tt - P(R|R=0): probability of admission rejection given residency status is 0\\ \tt - P(R|R=1): probability of admission rejection given residency status 
	  } 
	   } 
	   } 
	 & & \\ 
 

    \theutterance \stepcounter{utterance}  

    & & & \multicolumn{2}{p{0.3\linewidth}}{\cellcolor[rgb]{0.95,0.95,0.95}{%
	\makecell[{{p{\linewidth}}}]{% 
	  \tt {\tiny [GM$|$GM]}  
	 No 
	  } 
	   } 
	   } 
	 & & \\ 
 

    \theutterance \stepcounter{utterance}  

    & & & \multicolumn{2}{p{0.3\linewidth}}{\cellcolor[rgb]{0.95,0.95,0.95}{%
	\makecell[{{p{\linewidth}}}]{% 
	  \tt {\tiny [GM$|$GM]}  
	 no 
	  } 
	   } 
	   } 
	 & & \\ 
 

    \theutterance \stepcounter{utterance}  

    & & & \multicolumn{2}{p{0.3\linewidth}}{\cellcolor[rgb]{0.95,0.95,0.95}{%
	\makecell[{{p{\linewidth}}}]{% 
	  \tt {\tiny [GM$|$GM]}  
	 game_result = WIN 
	  } 
	   } 
	   } 
	 & & \\ 
 

\end{supertabular}
}

\end{document}
