
\documentclass{article}
\usepackage{colortbl}
\usepackage{makecell}
\usepackage{multirow}
\usepackage{supertabular}

\begin{document}

\newcounter{utterance}

\twocolumn

{ \footnotesize  \setcounter{utterance}{1}
\setlength{\tabcolsep}{0pt}
\begin{supertabular}{c@{$\;$}|p{.15\linewidth}@{}p{.15\linewidth}p{.15\linewidth}p{.15\linewidth}p{.15\linewidth}p{.15\linewidth}}

    \# & $\;$A & \multicolumn{4}{c}{Game Master} & $\;\:$B\\
    \hline 

    \theutterance \stepcounter{utterance}  

    & & \multicolumn{4}{p{0.6\linewidth}}{\cellcolor[rgb]{0.9,0.9,0.9}{%
	\makecell[{{p{\linewidth}}}]{% 
	  \tt {\tiny [A$\langle$GM]}  
	 You are an expert in causal inference. The following question is not a typical commonsense query, but rather a meticulously designed question created by a professor specializing in causal inference, intended to assess the students' mastery of the course content.\\ \tt Imagine a self-contained, hypothetical world with only the following conditions, and without any unmentioned factors or causal relationships: Gender has a direct effect on department competitiveness and peanut allergy. Department competitiveness has a direct effect on peanut allergy. For individuals who are not male, the probability of being allergic to peanuts is 48%. For individuals who are male, the probability of being allergic to peanuts is 36%. For individuals who are male, would it be more likely to see being allergic to peanuts if the individual had not been male?\\ \tt Start your answer with 'Yes' or 'No', followed by additional reasoning or evidence to support your explanation. 
	  } 
	   } 
	   } 
	 & \\ 
 

    \theutterance \stepcounter{utterance}  

    & \multicolumn{4}{p{0.6\linewidth}}{\cellcolor[rgb]{0.8,1,0.9}{%
	\makecell[{{p{\linewidth}}}]{% 
	  \tt {\tiny [A$\rangle$GM]}  
	 No\\ \tt \\ \tt To address this question, we need to apply the principles of causal inference, specifically the concept of counterfactuals. We are asked to compare the probability of being allergic to peanuts among males to the probability of being allergic to peanuts if the individual had not been male.\\ \tt \\ \tt Let's denote the following:\\ \tt \\ \tt - A: Being allergic to peanuts\\ \tt - M: Being male\\ \tt \\ \tt We are given the following information:\\ \tt \\ \tt - P(A|M) = 0.36 (probability of being allergic to peanuts given that the individual is male)\\ \tt - P(A|~M) = 0.48 (probability of being allergic to peanuts given that the individual is not male)\\ \tt \\ \tt We want to compare P(A|M) to P(A|~M~M), which is the probability of being allergic to peanuts if the individual had not been male. However, we cannot directly observe this counterfactual probability.\\ \tt \\ \tt To address this, we can use the concept of potential outcomes. Let's denote the potential outcome of being allergic to peanuts if the individual had not been male as A(~M). We can then use the following formula to estimate P(A|~M~M):\\ \tt \\ \tt P(A|~M~M) = P(A(~M)) - P(A|M)\\ \tt \\ \tt However, we don't have direct information about P(A(~M)), the probability of being allergic to peanuts if the individual had not been male. We can use the law of total probability to estimate this:\\ \tt \\ \tt P(A(~M)) = P 
	  } 
	   } 
	   } 
	 & & \\ 
 

    \theutterance \stepcounter{utterance}  

    & & & \multicolumn{2}{p{0.3\linewidth}}{\cellcolor[rgb]{0.95,0.95,0.95}{%
	\makecell[{{p{\linewidth}}}]{% 
	  \tt {\tiny [GM$|$GM]}  
	 No 
	  } 
	   } 
	   } 
	 & & \\ 
 

    \theutterance \stepcounter{utterance}  

    & & & \multicolumn{2}{p{0.3\linewidth}}{\cellcolor[rgb]{0.95,0.95,0.95}{%
	\makecell[{{p{\linewidth}}}]{% 
	  \tt {\tiny [GM$|$GM]}  
	 yes 
	  } 
	   } 
	   } 
	 & & \\ 
 

    \theutterance \stepcounter{utterance}  

    & & & \multicolumn{2}{p{0.3\linewidth}}{\cellcolor[rgb]{0.95,0.95,0.95}{%
	\makecell[{{p{\linewidth}}}]{% 
	  \tt {\tiny [GM$|$GM]}  
	 game_result = LOSE 
	  } 
	   } 
	   } 
	 & & \\ 
 

\end{supertabular}
}

\end{document}
