
\documentclass{article}
\usepackage{colortbl}
\usepackage{makecell}
\usepackage{multirow}
\usepackage{supertabular}

\begin{document}

\newcounter{utterance}

\twocolumn

{ \footnotesize  \setcounter{utterance}{1}
\setlength{\tabcolsep}{0pt}
\begin{supertabular}{c@{$\;$}|p{.15\linewidth}@{}p{.15\linewidth}p{.15\linewidth}p{.15\linewidth}p{.15\linewidth}p{.15\linewidth}}

    \# & $\;$A & \multicolumn{4}{c}{Game Master} & $\;\:$B\\
    \hline 

    \theutterance \stepcounter{utterance}  

    & & \multicolumn{4}{p{0.6\linewidth}}{\cellcolor[rgb]{0.9,0.9,0.9}{%
	\makecell[{{p{\linewidth}}}]{% 
	  \tt {\tiny [A$\langle$GM]}  
	 You are an expert in causal inference. The following question is not a typical commonsense query, but rather a meticulously designed question created by a professor specializing in causal inference, intended to assess the students' mastery of the course content.\\ \tt Imagine a self-contained, hypothetical world with only the following conditions, and without any unmentioned factors or causal relationships: Listening to jazz has a direct effect on tar deposit. Tar deposit has a direct effect on lung cancer. For people who do not listen to jazz, the probability of lung cancer is 61%. For people who listen to jazz, the probability of lung cancer is 79%. For people who listen to jazz, would it be less likely to see lung cancer if the person did not listen to jazz?\\ \tt Start your answer with 'Yes' or 'No', followed by additional reasoning or evidence to support your explanation. 
	  } 
	   } 
	   } 
	 & \\ 
 

    \theutterance \stepcounter{utterance}  

    & \multicolumn{4}{p{0.6\linewidth}}{\cellcolor[rgb]{0.8,1,0.9}{%
	\makecell[{{p{\linewidth}}}]{% 
	  \tt {\tiny [A$\rangle$GM]}  
	 No.\\ \tt \\ \tt To answer this question, we need to consider the causal relationships between the variables. Let's denote the variables as follows:\\ \tt \\ \tt - J: Listening to jazz\\ \tt - T: Tar deposit\\ \tt - L: Lung cancer\\ \tt \\ \tt Given the information:\\ \tt \\ \tt 1. J → T (Listening to jazz has a direct effect on tar deposit)\\ \tt 2. T → L (Tar deposit has a direct effect on lung cancer)\\ \tt 3. P(L|J) = 79% (Probability of lung cancer given that the person listens to jazz)\\ \tt 4. P(L|J') = 61% (Probability of lung cancer given that the person does not listen to jazz)\\ \tt \\ \tt We are asked to compare the probability of lung cancer if a person who listens to jazz did not listen to jazz. This is equivalent to comparing P(L|J,T') and P(L|J').\\ \tt \\ \tt Using the law of total probability, we can write:\\ \tt \\ \tt P(L|J,T') = P(L|J')P(T'|J) + P(L|J,T')P(T|J)\\ \tt \\ \tt However, we do not have the conditional probabilities P(T'|J) and P(T|J). But we can use the fact that P(L|J) = P(L|J,T)P(T|J) + P(L|J,T')P(T'|J) to derive the following equation:\\ \tt \\ \tt P(L|J) = P(L|J,T)P(T|J) + P(L|J,T')P 
	  } 
	   } 
	   } 
	 & & \\ 
 

    \theutterance \stepcounter{utterance}  

    & & & \multicolumn{2}{p{0.3\linewidth}}{\cellcolor[rgb]{0.95,0.95,0.95}{%
	\makecell[{{p{\linewidth}}}]{% 
	  \tt {\tiny [GM$|$GM]}  
	 No 
	  } 
	   } 
	   } 
	 & & \\ 
 

    \theutterance \stepcounter{utterance}  

    & & & \multicolumn{2}{p{0.3\linewidth}}{\cellcolor[rgb]{0.95,0.95,0.95}{%
	\makecell[{{p{\linewidth}}}]{% 
	  \tt {\tiny [GM$|$GM]}  
	 yes 
	  } 
	   } 
	   } 
	 & & \\ 
 

    \theutterance \stepcounter{utterance}  

    & & & \multicolumn{2}{p{0.3\linewidth}}{\cellcolor[rgb]{0.95,0.95,0.95}{%
	\makecell[{{p{\linewidth}}}]{% 
	  \tt {\tiny [GM$|$GM]}  
	 game_result = LOSE 
	  } 
	   } 
	   } 
	 & & \\ 
 

\end{supertabular}
}

\end{document}
