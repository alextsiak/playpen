
\documentclass{article}
\usepackage{colortbl}
\usepackage{makecell}
\usepackage{multirow}
\usepackage{supertabular}

\begin{document}

\newcounter{utterance}

\twocolumn

{ \footnotesize  \setcounter{utterance}{1}
\setlength{\tabcolsep}{0pt}
\begin{supertabular}{c@{$\;$}|p{.15\linewidth}@{}p{.15\linewidth}p{.15\linewidth}p{.15\linewidth}p{.15\linewidth}p{.15\linewidth}}

    \# & $\;$A & \multicolumn{4}{c}{Game Master} & $\;\:$B\\
    \hline 

    \theutterance \stepcounter{utterance}  

    & & \multicolumn{4}{p{0.6\linewidth}}{\cellcolor[rgb]{0.9,0.9,0.9}{%
	\makecell[{{p{\linewidth}}}]{% 
	  \tt {\tiny [A$\langle$GM]}  
	 You are an expert in causal inference. The following question is not a typical commonsense query, but rather a meticulously designed question created by a professor specializing in causal inference, intended to assess the students' mastery of the course content.\\ \tt Imagine a self-contained, hypothetical world with only the following conditions, and without any unmentioned factors or causal relationships: Jyka has a direct effect on zory and zuph. Zuph has a direct effect on glimx. Zory has a direct effect on glimx. We know that jyka causes not zuph and zory. zuph and zory causes glimx. We observed an individual is jyka. Would an individual is glimx if zuph instead of not zuph?\\ \tt Start your answer with 'Yes' or 'No', followed by additional reasoning or evidence to support your explanation. 
	  } 
	   } 
	   } 
	 & \\ 
 

    \theutterance \stepcounter{utterance}  

    & \multicolumn{4}{p{0.6\linewidth}}{\cellcolor[rgb]{0.8,1,0.9}{%
	\makecell[{{p{\linewidth}}}]{% 
	  \tt {\tiny [A$\rangle$GM]}  
	 No\\ \tt \\ \tt In this hypothetical world, we have the following causal relationships:\\ \tt \\ \tt 1. Jyka causes Zory and Zuph.\\ \tt 2. Zuph causes Glimx.\\ \tt 3. Zory causes Glimx.\\ \tt \\ \tt We are given that the individual is Jyka, which implies that the individual is both Zory and Zuph (due to the causal relationship between Jyka and both Zory and Zuph).\\ \tt \\ \tt Now, we are asked to consider the scenario where the individual is Zuph instead of not Zuph. However, this is a bit misleading, as the individual is already Zuph due to the causal relationship between Jyka and Zuph.\\ \tt \\ \tt The key insight here is to recognize that the causal relationship between Jyka and Zuph is direct, meaning that Jyka is the direct cause of Zuph, regardless of the individual's status as Zuph. In other words, the individual being Zuph is a consequence of the individual being Jyka, rather than the other way around.\\ \tt \\ \tt Since the individual is already Zuph, the question of whether the individual is Zuph instead of not Zuph is moot. The relevant causal relationship is between Jyka and Glimx, which is mediated by both Zory and Zuph. However, since the individual is already Zuph, the causal effect of Jyka on Glimx is already accounted for.\\ \tt \\ \tt In this 
	  } 
	   } 
	   } 
	 & & \\ 
 

    \theutterance \stepcounter{utterance}  

    & & & \multicolumn{2}{p{0.3\linewidth}}{\cellcolor[rgb]{0.95,0.95,0.95}{%
	\makecell[{{p{\linewidth}}}]{% 
	  \tt {\tiny [GM$|$GM]}  
	 No 
	  } 
	   } 
	   } 
	 & & \\ 
 

    \theutterance \stepcounter{utterance}  

    & & & \multicolumn{2}{p{0.3\linewidth}}{\cellcolor[rgb]{0.95,0.95,0.95}{%
	\makecell[{{p{\linewidth}}}]{% 
	  \tt {\tiny [GM$|$GM]}  
	 no 
	  } 
	   } 
	   } 
	 & & \\ 
 

    \theutterance \stepcounter{utterance}  

    & & & \multicolumn{2}{p{0.3\linewidth}}{\cellcolor[rgb]{0.95,0.95,0.95}{%
	\makecell[{{p{\linewidth}}}]{% 
	  \tt {\tiny [GM$|$GM]}  
	 game_result = WIN 
	  } 
	   } 
	   } 
	 & & \\ 
 

\end{supertabular}
}

\end{document}
