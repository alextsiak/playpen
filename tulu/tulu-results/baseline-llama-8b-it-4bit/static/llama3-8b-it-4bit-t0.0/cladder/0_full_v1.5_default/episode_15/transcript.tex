
\documentclass{article}
\usepackage{colortbl}
\usepackage{makecell}
\usepackage{multirow}
\usepackage{supertabular}

\begin{document}

\newcounter{utterance}

\twocolumn

{ \footnotesize  \setcounter{utterance}{1}
\setlength{\tabcolsep}{0pt}
\begin{supertabular}{c@{$\;$}|p{.15\linewidth}@{}p{.15\linewidth}p{.15\linewidth}p{.15\linewidth}p{.15\linewidth}p{.15\linewidth}}

    \# & $\;$A & \multicolumn{4}{c}{Game Master} & $\;\:$B\\
    \hline 

    \theutterance \stepcounter{utterance}  

    & & \multicolumn{4}{p{0.6\linewidth}}{\cellcolor[rgb]{0.9,0.9,0.9}{%
	\makecell[{{p{\linewidth}}}]{% 
	  \tt {\tiny [A$\langle$GM]}  
	 You are an expert in causal inference. The following question is not a typical commonsense query, but rather a meticulously designed question created by a professor specializing in causal inference, intended to assess the students' mastery of the course content.\\ \tt Imagine a self-contained, hypothetical world with only the following conditions, and without any unmentioned factors or causal relationships: Husband has a direct effect on wife and alarm clock. Wife has a direct effect on alarm clock. For husbands that don't set the alarm, the probability of ringing alarm is 78%. For husbands that set the alarm, the probability of ringing alarm is 32%. For husbands that set the alarm, would it be less likely to see ringing alarm if the husband had not set the alarm?\\ \tt Start your answer with 'Yes' or 'No', followed by additional reasoning or evidence to support your explanation. 
	  } 
	   } 
	   } 
	 & \\ 
 

    \theutterance \stepcounter{utterance}  

    & \multicolumn{4}{p{0.6\linewidth}}{\cellcolor[rgb]{0.8,1,0.9}{%
	\makecell[{{p{\linewidth}}}]{% 
	  \tt {\tiny [A$\rangle$GM]}  
	 No, it would not be less likely to see a ringing alarm if the husband had not set the alarm.\\ \tt \\ \tt To understand this, let's break down the given information:\\ \tt \\ \tt - The probability of a ringing alarm for husbands who don't set the alarm is 78%.\\ \tt - The probability of a ringing alarm for husbands who set the alarm is 32%.\\ \tt \\ \tt The question asks whether it would be less likely to see a ringing alarm if the husband had not set the alarm. This is essentially asking whether the probability of a ringing alarm is lower when the husband sets the alarm compared to when he doesn't set the alarm.\\ \tt \\ \tt To answer this, we can use the concept of conditional probabilities. Let's define two events:\\ \tt \\ \tt - A: The husband sets the alarm.\\ \tt - B: The alarm rings.\\ \tt \\ \tt We are given the following conditional probabilities:\\ \tt \\ \tt - P(B|A) = 0.32 (probability of alarm ringing given that the husband sets the alarm)\\ \tt - P(B|A') = 0.78 (probability of alarm ringing given that the husband does not set the alarm)\\ \tt \\ \tt We want to compare P(B|A) and P(B|A'). However, we are not directly given P(B|A') in terms of P(A|B) and P(B). But we can use the law of total probability to find P(B) in terms of P(A) and P(B|A) and P(B|A').\\ \tt \\ \tt However, we can use the fact that P(B) = P 
	  } 
	   } 
	   } 
	 & & \\ 
 

    \theutterance \stepcounter{utterance}  

    & & & \multicolumn{2}{p{0.3\linewidth}}{\cellcolor[rgb]{0.95,0.95,0.95}{%
	\makecell[{{p{\linewidth}}}]{% 
	  \tt {\tiny [GM$|$GM]}  
	 No 
	  } 
	   } 
	   } 
	 & & \\ 
 

    \theutterance \stepcounter{utterance}  

    & & & \multicolumn{2}{p{0.3\linewidth}}{\cellcolor[rgb]{0.95,0.95,0.95}{%
	\makecell[{{p{\linewidth}}}]{% 
	  \tt {\tiny [GM$|$GM]}  
	 no 
	  } 
	   } 
	   } 
	 & & \\ 
 

    \theutterance \stepcounter{utterance}  

    & & & \multicolumn{2}{p{0.3\linewidth}}{\cellcolor[rgb]{0.95,0.95,0.95}{%
	\makecell[{{p{\linewidth}}}]{% 
	  \tt {\tiny [GM$|$GM]}  
	 game_result = WIN 
	  } 
	   } 
	   } 
	 & & \\ 
 

\end{supertabular}
}

\end{document}
