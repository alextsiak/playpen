
\documentclass{article}
\usepackage{colortbl}
\usepackage{makecell}
\usepackage{multirow}
\usepackage{supertabular}

\begin{document}

\newcounter{utterance}

\twocolumn

{ \footnotesize  \setcounter{utterance}{1}
\setlength{\tabcolsep}{0pt}
\begin{supertabular}{c@{$\;$}|p{.15\linewidth}@{}p{.15\linewidth}p{.15\linewidth}p{.15\linewidth}p{.15\linewidth}p{.15\linewidth}}

    \# & $\;$A & \multicolumn{4}{c}{Game Master} & $\;\:$B\\
    \hline 

    \theutterance \stepcounter{utterance}  

    & & \multicolumn{4}{p{0.6\linewidth}}{\cellcolor[rgb]{0.9,0.9,0.9}{%
	\makecell[{{p{\linewidth}}}]{% 
	  \tt {\tiny [A$\langle$GM]}  
	 You are an expert in causal inference. The following question is not a typical commonsense query, but rather a meticulously designed question created by a professor specializing in causal inference, intended to assess the students' mastery of the course content.\\ \tt Imagine a self-contained, hypothetical world with only the following conditions, and without any unmentioned factors or causal relationships: Pexu has a direct effect on rukz. Hwax has a direct effect on rukz. For those who are not pexu, the probability of rukz is 9%. For those who are pexu, the probability of rukz is 18%. For those who are pexu, would it be less likely to see rukz if the individual was not pexu?\\ \tt Start your answer with 'Yes' or 'No', followed by additional reasoning or evidence to support your explanation. 
	  } 
	   } 
	   } 
	 & \\ 
 

    \theutterance \stepcounter{utterance}  

    & \multicolumn{4}{p{0.6\linewidth}}{\cellcolor[rgb]{0.8,1,0.9}{%
	\makecell[{{p{\linewidth}}}]{% 
	  \tt {\tiny [A$\rangle$GM]}  
	 No, it would not be less likely to see rukz if the individual was not pexu.\\ \tt \\ \tt To understand this, let's break down the given information:\\ \tt \\ \tt 1. Pexu has a direct effect on rukz.\\ \tt 2. Hwax has a direct effect on rukz.\\ \tt 3. For those who are not pexu, the probability of rukz is 9%.\\ \tt 4. For those who are pexu, the probability of rukz is 18%.\\ \tt \\ \tt The question asks whether the absence of pexu (i.e., being not pexu) would decrease the probability of rukz for those who are pexu. However, the given information does not provide a direct comparison between the probability of rukz for pexu and non-pexu individuals. \\ \tt \\ \tt To answer this question, we need to consider the potential effect of hwax on rukz. Since hwax has a direct effect on rukz, it's possible that the effect of pexu on rukz is mediated by hwax. In other words, the relationship between pexu and rukz might be influenced by the presence or absence of hwax.\\ \tt \\ \tt However, the given information does not provide any direct evidence about the relationship between pexu and hwax, or how hwax affects the probability of rukz for pexu individuals. \\ \tt \\ \tt Given the information provided, we cannot directly conclude that being not pexu would decrease the probability of rukz for those who 
	  } 
	   } 
	   } 
	 & & \\ 
 

    \theutterance \stepcounter{utterance}  

    & & & \multicolumn{2}{p{0.3\linewidth}}{\cellcolor[rgb]{0.95,0.95,0.95}{%
	\makecell[{{p{\linewidth}}}]{% 
	  \tt {\tiny [GM$|$GM]}  
	 No 
	  } 
	   } 
	   } 
	 & & \\ 
 

    \theutterance \stepcounter{utterance}  

    & & & \multicolumn{2}{p{0.3\linewidth}}{\cellcolor[rgb]{0.95,0.95,0.95}{%
	\makecell[{{p{\linewidth}}}]{% 
	  \tt {\tiny [GM$|$GM]}  
	 yes 
	  } 
	   } 
	   } 
	 & & \\ 
 

    \theutterance \stepcounter{utterance}  

    & & & \multicolumn{2}{p{0.3\linewidth}}{\cellcolor[rgb]{0.95,0.95,0.95}{%
	\makecell[{{p{\linewidth}}}]{% 
	  \tt {\tiny [GM$|$GM]}  
	 game_result = LOSE 
	  } 
	   } 
	   } 
	 & & \\ 
 

\end{supertabular}
}

\end{document}
