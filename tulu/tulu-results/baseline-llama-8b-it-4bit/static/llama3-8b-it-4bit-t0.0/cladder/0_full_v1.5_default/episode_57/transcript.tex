
\documentclass{article}
\usepackage{colortbl}
\usepackage{makecell}
\usepackage{multirow}
\usepackage{supertabular}

\begin{document}

\newcounter{utterance}

\twocolumn

{ \footnotesize  \setcounter{utterance}{1}
\setlength{\tabcolsep}{0pt}
\begin{supertabular}{c@{$\;$}|p{.15\linewidth}@{}p{.15\linewidth}p{.15\linewidth}p{.15\linewidth}p{.15\linewidth}p{.15\linewidth}}

    \# & $\;$A & \multicolumn{4}{c}{Game Master} & $\;\:$B\\
    \hline 

    \theutterance \stepcounter{utterance}  

    & & \multicolumn{4}{p{0.6\linewidth}}{\cellcolor[rgb]{0.9,0.9,0.9}{%
	\makecell[{{p{\linewidth}}}]{% 
	  \tt {\tiny [A$\langle$GM]}  
	 You are an expert in causal inference. The following question is not a typical commonsense query, but rather a meticulously designed question created by a professor specializing in causal inference, intended to assess the students' mastery of the course content.\\ \tt Imagine a self-contained, hypothetical world with only the following conditions, and without any unmentioned factors or causal relationships: My decision has a direct effect on penguin mood and curly hair. Penguin mood has a direct effect on curly hair. For those who choose to take the stairs, the probability of curly hair is 70%. For those who choose to take the elevator, the probability of curly hair is 17%. For those who choose to take the elevator, would it be less likely to see curly hair if I had taken the stairs?\\ \tt Start your answer with 'Yes' or 'No', followed by additional reasoning or evidence to support your explanation. 
	  } 
	   } 
	   } 
	 & \\ 
 

    \theutterance \stepcounter{utterance}  

    & \multicolumn{4}{p{0.6\linewidth}}{\cellcolor[rgb]{0.8,1,0.9}{%
	\makecell[{{p{\linewidth}}}]{% 
	  \tt {\tiny [A$\rangle$GM]}  
	 No.\\ \tt \\ \tt To address this question, we need to apply the principles of causal inference, specifically the concept of counterfactuals and potential outcomes. \\ \tt \\ \tt Let's break down the given information:\\ \tt \\ \tt 1. My decision (taking the stairs or the elevator) affects penguin mood.\\ \tt 2. Penguin mood affects curly hair.\\ \tt \\ \tt We are given the probabilities of curly hair for two groups: those who take the stairs (70%) and those who take the elevator (17%). However, we are not given the probability of curly hair for a group that takes the elevator but would have taken the stairs if they had chosen differently (i.e., the counterfactual probability).\\ \tt \\ \tt The question asks whether it would be less likely to see curly hair if I had taken the stairs, given that I took the elevator. This is a counterfactual scenario, and we need to consider the potential outcomes.\\ \tt \\ \tt The key point here is that the probability of curly hair is 70% for those who take the stairs, but we don't know the probability of curly hair for those who take the elevator but would have taken the stairs if they had chosen differently. \\ \tt \\ \tt In other words, we are comparing two different groups: those who take the elevator and those who take the stairs. We are not comparing the same group under different conditions (i.e., the counterfactual scenario).\\ \tt \\ \tt Therefore, we cannot conclude that it would be less likely to see curly hair if I had taken the stairs, given that I took the elevator. The information 
	  } 
	   } 
	   } 
	 & & \\ 
 

    \theutterance \stepcounter{utterance}  

    & & & \multicolumn{2}{p{0.3\linewidth}}{\cellcolor[rgb]{0.95,0.95,0.95}{%
	\makecell[{{p{\linewidth}}}]{% 
	  \tt {\tiny [GM$|$GM]}  
	 No 
	  } 
	   } 
	   } 
	 & & \\ 
 

    \theutterance \stepcounter{utterance}  

    & & & \multicolumn{2}{p{0.3\linewidth}}{\cellcolor[rgb]{0.95,0.95,0.95}{%
	\makecell[{{p{\linewidth}}}]{% 
	  \tt {\tiny [GM$|$GM]}  
	 no 
	  } 
	   } 
	   } 
	 & & \\ 
 

    \theutterance \stepcounter{utterance}  

    & & & \multicolumn{2}{p{0.3\linewidth}}{\cellcolor[rgb]{0.95,0.95,0.95}{%
	\makecell[{{p{\linewidth}}}]{% 
	  \tt {\tiny [GM$|$GM]}  
	 game_result = WIN 
	  } 
	   } 
	   } 
	 & & \\ 
 

\end{supertabular}
}

\end{document}
