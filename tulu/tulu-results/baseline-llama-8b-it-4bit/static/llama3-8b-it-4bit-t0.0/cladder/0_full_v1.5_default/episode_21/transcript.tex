
\documentclass{article}
\usepackage{colortbl}
\usepackage{makecell}
\usepackage{multirow}
\usepackage{supertabular}

\begin{document}

\newcounter{utterance}

\twocolumn

{ \footnotesize  \setcounter{utterance}{1}
\setlength{\tabcolsep}{0pt}
\begin{supertabular}{c@{$\;$}|p{.15\linewidth}@{}p{.15\linewidth}p{.15\linewidth}p{.15\linewidth}p{.15\linewidth}p{.15\linewidth}}

    \# & $\;$A & \multicolumn{4}{c}{Game Master} & $\;\:$B\\
    \hline 

    \theutterance \stepcounter{utterance}  

    & & \multicolumn{4}{p{0.6\linewidth}}{\cellcolor[rgb]{0.9,0.9,0.9}{%
	\makecell[{{p{\linewidth}}}]{% 
	  \tt {\tiny [A$\langle$GM]}  
	 You are an expert in causal inference. The following question is not a typical commonsense query, but rather a meticulously designed question created by a professor specializing in causal inference, intended to assess the students' mastery of the course content.\\ \tt Imagine a self-contained, hypothetical world with only the following conditions, and without any unmentioned factors or causal relationships: Alice waking up has a direct effect on Alice arriving to school. Traffic has a direct effect on Alice arriving to school. The overall probability of waking up late is 5%. For days when Alice wakes up on time, the probability of arriving to school on time is 52%. For days when Alice wakes up late, the probability of arriving to school on time is 87%. Is arriving to school on time more likely than arriving to school late overall?\\ \tt Start your answer with 'Yes' or 'No', followed by additional reasoning or evidence to support your explanation. 
	  } 
	   } 
	   } 
	 & \\ 
 

    \theutterance \stepcounter{utterance}  

    & \multicolumn{4}{p{0.6\linewidth}}{\cellcolor[rgb]{0.8,1,0.9}{%
	\makecell[{{p{\linewidth}}}]{% 
	  \tt {\tiny [A$\rangle$GM]}  
	 No, arriving to school on time is not more likely than arriving to school late overall.\\ \tt \\ \tt To understand this, let's break down the given information:\\ \tt \\ \tt 1. The probability of waking up late is 5%.\\ \tt 2. When Alice wakes up on time, the probability of arriving to school on time is 52%.\\ \tt 3. When Alice wakes up late, the probability of arriving to school on time is 87%.\\ \tt \\ \tt We can calculate the overall probability of arriving to school on time by considering two scenarios: waking up on time and waking up late.\\ \tt \\ \tt - Probability of waking up on time: 95% (100% - 5%)\\ \tt - Probability of arriving to school on time given waking up on time: 52%\\ \tt - Probability of arriving to school on time given waking up late: 87%\\ \tt \\ \tt Using the law of total probability, we can calculate the overall probability of arriving to school on time as follows:\\ \tt \\ \tt P(arriving on time) = P(waking up on time) * P(arriving on time | waking up on time) + P(waking up late) * P(arriving on time | waking up late)\\ \tt = 0.95 * 0.52 + 0.05 * 0.87\\ \tt = 0.494 + 0.0435\\ \tt = 0.5375\\ \tt \\ \tt Now, let's calculate the overall probability of arriving to school late:\\ \tt \\ \tt P(arriving late) = 1 - P(arriving on time)\\ \tt = 1 - 
	  } 
	   } 
	   } 
	 & & \\ 
 

    \theutterance \stepcounter{utterance}  

    & & & \multicolumn{2}{p{0.3\linewidth}}{\cellcolor[rgb]{0.95,0.95,0.95}{%
	\makecell[{{p{\linewidth}}}]{% 
	  \tt {\tiny [GM$|$GM]}  
	 No 
	  } 
	   } 
	   } 
	 & & \\ 
 

    \theutterance \stepcounter{utterance}  

    & & & \multicolumn{2}{p{0.3\linewidth}}{\cellcolor[rgb]{0.95,0.95,0.95}{%
	\makecell[{{p{\linewidth}}}]{% 
	  \tt {\tiny [GM$|$GM]}  
	 yes 
	  } 
	   } 
	   } 
	 & & \\ 
 

    \theutterance \stepcounter{utterance}  

    & & & \multicolumn{2}{p{0.3\linewidth}}{\cellcolor[rgb]{0.95,0.95,0.95}{%
	\makecell[{{p{\linewidth}}}]{% 
	  \tt {\tiny [GM$|$GM]}  
	 game_result = LOSE 
	  } 
	   } 
	   } 
	 & & \\ 
 

\end{supertabular}
}

\end{document}
