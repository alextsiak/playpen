
\documentclass{article}
\usepackage{colortbl}
\usepackage{makecell}
\usepackage{multirow}
\usepackage{supertabular}

\begin{document}

\newcounter{utterance}

\twocolumn

{ \footnotesize  \setcounter{utterance}{1}
\setlength{\tabcolsep}{0pt}
\begin{supertabular}{c@{$\;$}|p{.15\linewidth}@{}p{.15\linewidth}p{.15\linewidth}p{.15\linewidth}p{.15\linewidth}p{.15\linewidth}}

    \# & $\;$A & \multicolumn{4}{c}{Game Master} & $\;\:$B\\
    \hline 

    \theutterance \stepcounter{utterance}  

    & & \multicolumn{4}{p{0.6\linewidth}}{\cellcolor[rgb]{0.9,0.9,0.9}{%
	\makecell[{{p{\linewidth}}}]{% 
	  \tt {\tiny [A$\langle$GM]}  
	 You are an expert in causal inference. The following question is not a typical commonsense query, but rather a meticulously designed question created by a professor specializing in causal inference, intended to assess the students' mastery of the course content.\\ \tt Imagine a self-contained, hypothetical world with only the following conditions, and without any unmentioned factors or causal relationships: Tijv has a direct effect on tijw and xevo. Xevo has a direct effect on gyzp. Tijw has a direct effect on gyzp. For those who are not tijv and are not xevo, the probability of gyzp is 29%. For those who are not tijv and are xevo, the probability of gyzp is 14%. For those who are tijv and are not xevo, the probability of gyzp is 11%. For those who are tijv and are xevo, the probability of gyzp is 16%. The overall probability of tijv is 88%. Will xevo increase the chance of gyzp?\\ \tt Start your answer with 'Yes' or 'No', followed by additional reasoning or evidence to support your explanation. 
	  } 
	   } 
	   } 
	 & \\ 
 

    \theutterance \stepcounter{utterance}  

    & \multicolumn{4}{p{0.6\linewidth}}{\cellcolor[rgb]{0.8,1,0.9}{%
	\makecell[{{p{\linewidth}}}]{% 
	  \tt {\tiny [A$\rangle$GM]}  
	 No, xevo will not increase the chance of gyzp.\\ \tt \\ \tt To understand this, we need to apply the concept of causal mediation analysis. We are given the following probabilities:\\ \tt \\ \tt - P(gyzp | not tijv, not xevo) = 29%\\ \tt - P(gyzp | not tijv, xevo) = 14%\\ \tt - P(gyzp | tijv, not xevo) = 11%\\ \tt - P(gyzp | tijv, xevo) = 16%\\ \tt \\ \tt We are also given that the overall probability of tijv is 88%. This implies that the probability of not tijv is 12%.\\ \tt \\ \tt To assess the effect of xevo on gyzp, we need to compare the probabilities of gyzp among those who are not tijv and those who are tijv. However, we cannot directly compare these probabilities because they are conditional on the presence or absence of xevo.\\ \tt \\ \tt To address this, we can use the concept of causal mediation analysis. We can calculate the probability of gyzp among those who are not tijv and xevo, and those who are tijv and xevo, by considering the probabilities of gyzp among those who are not tijv and those who are tijv, and the probability of xevo among those who are not tijv and those who are tijv.\\ \tt \\ \tt Let's denote the probability of xevo among those who 
	  } 
	   } 
	   } 
	 & & \\ 
 

    \theutterance \stepcounter{utterance}  

    & & & \multicolumn{2}{p{0.3\linewidth}}{\cellcolor[rgb]{0.95,0.95,0.95}{%
	\makecell[{{p{\linewidth}}}]{% 
	  \tt {\tiny [GM$|$GM]}  
	 No 
	  } 
	   } 
	   } 
	 & & \\ 
 

    \theutterance \stepcounter{utterance}  

    & & & \multicolumn{2}{p{0.3\linewidth}}{\cellcolor[rgb]{0.95,0.95,0.95}{%
	\makecell[{{p{\linewidth}}}]{% 
	  \tt {\tiny [GM$|$GM]}  
	 yes 
	  } 
	   } 
	   } 
	 & & \\ 
 

    \theutterance \stepcounter{utterance}  

    & & & \multicolumn{2}{p{0.3\linewidth}}{\cellcolor[rgb]{0.95,0.95,0.95}{%
	\makecell[{{p{\linewidth}}}]{% 
	  \tt {\tiny [GM$|$GM]}  
	 game_result = LOSE 
	  } 
	   } 
	   } 
	 & & \\ 
 

\end{supertabular}
}

\end{document}
