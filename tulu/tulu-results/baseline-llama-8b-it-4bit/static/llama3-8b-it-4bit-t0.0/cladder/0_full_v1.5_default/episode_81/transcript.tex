
\documentclass{article}
\usepackage{colortbl}
\usepackage{makecell}
\usepackage{multirow}
\usepackage{supertabular}

\begin{document}

\newcounter{utterance}

\twocolumn

{ \footnotesize  \setcounter{utterance}{1}
\setlength{\tabcolsep}{0pt}
\begin{supertabular}{c@{$\;$}|p{.15\linewidth}@{}p{.15\linewidth}p{.15\linewidth}p{.15\linewidth}p{.15\linewidth}p{.15\linewidth}}

    \# & $\;$A & \multicolumn{4}{c}{Game Master} & $\;\:$B\\
    \hline 

    \theutterance \stepcounter{utterance}  

    & & \multicolumn{4}{p{0.6\linewidth}}{\cellcolor[rgb]{0.9,0.9,0.9}{%
	\makecell[{{p{\linewidth}}}]{% 
	  \tt {\tiny [A$\langle$GM]}  
	 You are an expert in causal inference. The following question is not a typical commonsense query, but rather a meticulously designed question created by a professor specializing in causal inference, intended to assess the students' mastery of the course content.\\ \tt Imagine a self-contained, hypothetical world with only the following conditions, and without any unmentioned factors or causal relationships: Rixq has a direct effect on zuph and xevu. Swoy has a direct effect on zuph and xevu. Zuph has a direct effect on xevu. Swoy is unobserved. The overall probability of rixq is 3%. The probability of not rixq and xevu is 78%. The probability of rixq and xevu is 2%. Is the chance of xevu larger when observing rixq?\\ \tt Start your answer with 'Yes' or 'No', followed by additional reasoning or evidence to support your explanation. 
	  } 
	   } 
	   } 
	 & \\ 
 

    \theutterance \stepcounter{utterance}  

    & \multicolumn{4}{p{0.6\linewidth}}{\cellcolor[rgb]{0.8,1,0.9}{%
	\makecell[{{p{\linewidth}}}]{% 
	  \tt {\tiny [A$\rangle$GM]}  
	 No, the chance of xevu is not larger when observing rixq.\\ \tt \\ \tt To analyze this problem, we can use the potential outcomes framework, which is a common tool in causal inference. Let's define the following variables:\\ \tt \\ \tt - Rixq: the treatment or exposure of interest (3% probability)\\ \tt - Xevu: the outcome of interest\\ \tt - Zuph: a mediator or intermediate variable\\ \tt - Swoy: an unobserved confounding variable\\ \tt \\ \tt Given the information provided, we have the following conditional probabilities:\\ \tt \\ \tt - P(Xevu | Rixq) = 2% (probability of xevu given rixq)\\ \tt - P(Xevu | not Rixq) = 78% (probability of xevu given not rixq)\\ \tt \\ \tt To determine whether the chance of xevu is larger when observing rixq, we need to compare these two probabilities. However, we also need to consider the effect of the unobserved confounding variable Swoy.\\ \tt \\ \tt Since Swoy is unobserved, we cannot directly estimate its effect on the relationship between Rixq and Xevu. However, we can use the fact that Zuph has a direct effect on Xevu to infer the potential outcomes.\\ \tt \\ \tt Let's assume that the probability of Xevu is 78% when Swoy is 0 (i.e., when Swoy is not present) and 100% when Swoy is 
	  } 
	   } 
	   } 
	 & & \\ 
 

    \theutterance \stepcounter{utterance}  

    & & & \multicolumn{2}{p{0.3\linewidth}}{\cellcolor[rgb]{0.95,0.95,0.95}{%
	\makecell[{{p{\linewidth}}}]{% 
	  \tt {\tiny [GM$|$GM]}  
	 No 
	  } 
	   } 
	   } 
	 & & \\ 
 

    \theutterance \stepcounter{utterance}  

    & & & \multicolumn{2}{p{0.3\linewidth}}{\cellcolor[rgb]{0.95,0.95,0.95}{%
	\makecell[{{p{\linewidth}}}]{% 
	  \tt {\tiny [GM$|$GM]}  
	 no 
	  } 
	   } 
	   } 
	 & & \\ 
 

    \theutterance \stepcounter{utterance}  

    & & & \multicolumn{2}{p{0.3\linewidth}}{\cellcolor[rgb]{0.95,0.95,0.95}{%
	\makecell[{{p{\linewidth}}}]{% 
	  \tt {\tiny [GM$|$GM]}  
	 game_result = WIN 
	  } 
	   } 
	   } 
	 & & \\ 
 

\end{supertabular}
}

\end{document}
