
\documentclass{article}
\usepackage{colortbl}
\usepackage{makecell}
\usepackage{multirow}
\usepackage{supertabular}

\begin{document}

\newcounter{utterance}

\twocolumn

{ \footnotesize  \setcounter{utterance}{1}
\setlength{\tabcolsep}{0pt}
\begin{supertabular}{c@{$\;$}|p{.15\linewidth}@{}p{.15\linewidth}p{.15\linewidth}p{.15\linewidth}p{.15\linewidth}p{.15\linewidth}}

    \# & $\;$A & \multicolumn{4}{c}{Game Master} & $\;\:$B\\
    \hline 

    \theutterance \stepcounter{utterance}  

    & & \multicolumn{4}{p{0.6\linewidth}}{\cellcolor[rgb]{0.9,0.9,0.9}{%
	\makecell[{{p{\linewidth}}}]{% 
	  \tt {\tiny [A$\langle$GM]}  
	 You are an expert in causal inference. The following question is not a typical commonsense query, but rather a meticulously designed question created by a professor specializing in causal inference, intended to assess the students' mastery of the course content.\\ \tt Imagine a self-contained, hypothetical world with only the following conditions, and without any unmentioned factors or causal relationships: Smoking has a direct effect on effort and college admission. Effort has a direct effect on college admission. For nonsmokers, the probability of college admission is 67%. For smokers, the probability of college admission is 55%. For smokers, would it be less likely to see college admission if the student had been a nonsmoker?\\ \tt Start your answer with 'Yes' or 'No', followed by additional reasoning or evidence to support your explanation. 
	  } 
	   } 
	   } 
	 & \\ 
 

    \theutterance \stepcounter{utterance}  

    & \multicolumn{4}{p{0.6\linewidth}}{\cellcolor[rgb]{0.8,1,0.9}{%
	\makecell[{{p{\linewidth}}}]{% 
	  \tt {\tiny [A$\rangle$GM]}  
	 No.\\ \tt \\ \tt To address this question, we need to consider the causal relationships between smoking, effort, and college admission. Let's denote the variables as follows:\\ \tt \\ \tt - S: Smoking (1 = smoker, 0 = nonsmoker)\\ \tt - E: Effort (1 = high effort, 0 = low effort)\\ \tt - A: College admission (1 = admitted, 0 = not admitted)\\ \tt \\ \tt Given the information:\\ \tt \\ \tt - P(A|E=1, S=0) = 67% (probability of admission for nonsmokers with high effort)\\ \tt - P(A|E=1, S=1) = 55% (probability of admission for smokers with high effort)\\ \tt \\ \tt We are asked to compare the probability of admission for smokers with high effort to the probability of admission for nonsmokers with high effort. However, the given information does not directly provide this comparison.\\ \tt \\ \tt To answer the question, we need to consider the potential effect of smoking on effort. If smoking reduces effort, then the probability of admission for smokers with high effort would be lower than the probability of admission for nonsmokers with high effort. However, the given information does not provide any direct evidence of this relationship.\\ \tt \\ \tt One possible interpretation is that smoking has a direct negative effect on effort, which in turn reduces the probability of admission. If this is the case, then the probability of admission for smokers with high effort would be lower than the probability of admission for nonsmokers with high effort.\\ \tt \\ \tt However, the given information 
	  } 
	   } 
	   } 
	 & & \\ 
 

    \theutterance \stepcounter{utterance}  

    & & & \multicolumn{2}{p{0.3\linewidth}}{\cellcolor[rgb]{0.95,0.95,0.95}{%
	\makecell[{{p{\linewidth}}}]{% 
	  \tt {\tiny [GM$|$GM]}  
	 No 
	  } 
	   } 
	   } 
	 & & \\ 
 

    \theutterance \stepcounter{utterance}  

    & & & \multicolumn{2}{p{0.3\linewidth}}{\cellcolor[rgb]{0.95,0.95,0.95}{%
	\makecell[{{p{\linewidth}}}]{% 
	  \tt {\tiny [GM$|$GM]}  
	 no 
	  } 
	   } 
	   } 
	 & & \\ 
 

    \theutterance \stepcounter{utterance}  

    & & & \multicolumn{2}{p{0.3\linewidth}}{\cellcolor[rgb]{0.95,0.95,0.95}{%
	\makecell[{{p{\linewidth}}}]{% 
	  \tt {\tiny [GM$|$GM]}  
	 game_result = WIN 
	  } 
	   } 
	   } 
	 & & \\ 
 

\end{supertabular}
}

\end{document}
