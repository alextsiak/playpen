
\documentclass{article}
\usepackage{colortbl}
\usepackage{makecell}
\usepackage{multirow}
\usepackage{supertabular}

\begin{document}

\newcounter{utterance}

\twocolumn

{ \footnotesize  \setcounter{utterance}{1}
\setlength{\tabcolsep}{0pt}
\begin{supertabular}{c@{$\;$}|p{.15\linewidth}@{}p{.15\linewidth}p{.15\linewidth}p{.15\linewidth}p{.15\linewidth}p{.15\linewidth}}

    \# & $\;$A & \multicolumn{4}{c}{Game Master} & $\;\:$B\\
    \hline 

    \theutterance \stepcounter{utterance}  

    & & \multicolumn{4}{p{0.6\linewidth}}{\cellcolor[rgb]{0.9,0.9,0.9}{%
	\makecell[{{p{\linewidth}}}]{% 
	  \tt {\tiny [A$\langle$GM]}  
	 You are an expert in causal inference. The following question is not a typical commonsense query, but rather a meticulously designed question created by a professor specializing in causal inference, intended to assess the students' mastery of the course content.\\ \tt Imagine a self-contained, hypothetical world with only the following conditions, and without any unmentioned factors or causal relationships: Poverty has a direct effect on liking spicy food and cholera. Water company has a direct effect on liking spicy food. Liking spicy food has a direct effect on cholera. Poverty is unobserved. The overall probability of liking spicy food is 41%. For people who do not like spicy food, the probability of cholera contraction is 59%. For people who like spicy food, the probability of cholera contraction is 33%. Is cholera contraction more likely than cholera prevention overall?\\ \tt Start your answer with 'Yes' or 'No', followed by additional reasoning or evidence to support your explanation. 
	  } 
	   } 
	   } 
	 & \\ 
 

    \theutterance \stepcounter{utterance}  

    & \multicolumn{4}{p{0.6\linewidth}}{\cellcolor[rgb]{0.8,1,0.9}{%
	\makecell[{{p{\linewidth}}}]{% 
	  \tt {\tiny [A$\rangle$GM]}  
	 No, cholera contraction is not more likely than cholera prevention overall.\\ \tt \\ \tt To understand this, let's break down the given information and apply the principles of causal inference. We have the following relationships:\\ \tt \\ \tt 1. Poverty → Liking spicy food\\ \tt 2. Water company → Liking spicy food\\ \tt 3. Liking spicy food → Cholera\\ \tt \\ \tt We are given the following probabilities:\\ \tt \\ \tt - P(Liking spicy food) = 41%\\ \tt - P(Cholera | Not liking spicy food) = 59%\\ \tt - P(Cholera | Liking spicy food) = 33%\\ \tt \\ \tt We want to determine if cholera contraction is more likely than cholera prevention overall. To do this, we need to calculate the overall probability of cholera contraction and compare it to the overall probability of cholera prevention.\\ \tt \\ \tt Let's denote the probability of cholera contraction as P(Cholera) and the probability of cholera prevention as P(No cholera). We know that P(No cholera) = 1 - P(Cholera).\\ \tt \\ \tt We can use the law of total probability to calculate P(Cholera). We have two groups: those who like spicy food and those who do not. The probability of cholera contraction for each group is given:\\ \tt \\ \tt - P(Cholera | Liking spicy food) = 33%\\ \tt - P(Cholera | Not liking spicy food) = 59%\\ \tt \\ \tt We also know the probability of liking spicy food: P 
	  } 
	   } 
	   } 
	 & & \\ 
 

    \theutterance \stepcounter{utterance}  

    & & & \multicolumn{2}{p{0.3\linewidth}}{\cellcolor[rgb]{0.95,0.95,0.95}{%
	\makecell[{{p{\linewidth}}}]{% 
	  \tt {\tiny [GM$|$GM]}  
	 No 
	  } 
	   } 
	   } 
	 & & \\ 
 

    \theutterance \stepcounter{utterance}  

    & & & \multicolumn{2}{p{0.3\linewidth}}{\cellcolor[rgb]{0.95,0.95,0.95}{%
	\makecell[{{p{\linewidth}}}]{% 
	  \tt {\tiny [GM$|$GM]}  
	 no 
	  } 
	   } 
	   } 
	 & & \\ 
 

    \theutterance \stepcounter{utterance}  

    & & & \multicolumn{2}{p{0.3\linewidth}}{\cellcolor[rgb]{0.95,0.95,0.95}{%
	\makecell[{{p{\linewidth}}}]{% 
	  \tt {\tiny [GM$|$GM]}  
	 game_result = WIN 
	  } 
	   } 
	   } 
	 & & \\ 
 

\end{supertabular}
}

\end{document}
