
\documentclass{article}
\usepackage{colortbl}
\usepackage{makecell}
\usepackage{multirow}
\usepackage{supertabular}

\begin{document}

\newcounter{utterance}

\twocolumn

{ \footnotesize  \setcounter{utterance}{1}
\setlength{\tabcolsep}{0pt}
\begin{supertabular}{c@{$\;$}|p{.15\linewidth}@{}p{.15\linewidth}p{.15\linewidth}p{.15\linewidth}p{.15\linewidth}p{.15\linewidth}}

    \# & $\;$A & \multicolumn{4}{c}{Game Master} & $\;\:$B\\
    \hline 

    \theutterance \stepcounter{utterance}  

    & & \multicolumn{4}{p{0.6\linewidth}}{\cellcolor[rgb]{0.9,0.9,0.9}{%
	\makecell[{{p{\linewidth}}}]{% 
	  \tt {\tiny [A$\langle$GM]}  
	 You are an expert in causal inference. The following question is not a typical commonsense query, but rather a meticulously designed question created by a professor specializing in causal inference, intended to assess the students' mastery of the course content.\\ \tt Imagine a self-contained, hypothetical world with only the following conditions, and without any unmentioned factors or causal relationships: Ability to speak english has a direct effect on the forest. The smoker has a direct effect on the forest. The overall probability of speaking english is 93%. For people who do not speak english, the probability of the forest on fire is 79%. For people who speak english, the probability of the forest on fire is 65%. Is the forest on fire more likely than the forest not on fire overall?\\ \tt Start your answer with 'Yes' or 'No', followed by additional reasoning or evidence to support your explanation. 
	  } 
	   } 
	   } 
	 & \\ 
 

    \theutterance \stepcounter{utterance}  

    & \multicolumn{4}{p{0.6\linewidth}}{\cellcolor[rgb]{0.8,1,0.9}{%
	\makecell[{{p{\linewidth}}}]{% 
	  \tt {\tiny [A$\rangle$GM]}  
	 No, the forest is not more likely to be on fire overall.\\ \tt \\ \tt To address this question, we need to apply the principles of causal inference, specifically the concept of confounding variables. In this hypothetical world, we have two variables: the ability to speak English (E) and the state of the forest (F). We also have two direct effects: E → F and S → F, where S is the smoker (which is not explicitly mentioned in the question, but we can infer its presence from the context).\\ \tt \\ \tt The given probabilities are:\\ \tt \\ \tt - P(F|~E) = 0.79 (probability of the forest on fire given that the person does not speak English)\\ \tt - P(F|E) = 0.65 (probability of the forest on fire given that the person speaks English)\\ \tt - P(E) = 0.93 (probability of speaking English)\\ \tt \\ \tt We can use the law of total probability to calculate the overall probability of the forest being on fire:\\ \tt \\ \tt P(F) = P(F|E) \* P(E) + P(F|~E) \* P(~E)\\ \tt \\ \tt Since P(~E) = 1 - P(E) = 1 - 0.93 = 0.07, we can substitute the values:\\ \tt \\ \tt P(F) = 0.65 \* 0.93 + 0.79 \* 0.07\\ \tt P(F) = 0.6035 + 0.0553\\ \tt P(F 
	  } 
	   } 
	   } 
	 & & \\ 
 

    \theutterance \stepcounter{utterance}  

    & & & \multicolumn{2}{p{0.3\linewidth}}{\cellcolor[rgb]{0.95,0.95,0.95}{%
	\makecell[{{p{\linewidth}}}]{% 
	  \tt {\tiny [GM$|$GM]}  
	 No 
	  } 
	   } 
	   } 
	 & & \\ 
 

    \theutterance \stepcounter{utterance}  

    & & & \multicolumn{2}{p{0.3\linewidth}}{\cellcolor[rgb]{0.95,0.95,0.95}{%
	\makecell[{{p{\linewidth}}}]{% 
	  \tt {\tiny [GM$|$GM]}  
	 yes 
	  } 
	   } 
	   } 
	 & & \\ 
 

    \theutterance \stepcounter{utterance}  

    & & & \multicolumn{2}{p{0.3\linewidth}}{\cellcolor[rgb]{0.95,0.95,0.95}{%
	\makecell[{{p{\linewidth}}}]{% 
	  \tt {\tiny [GM$|$GM]}  
	 game_result = LOSE 
	  } 
	   } 
	   } 
	 & & \\ 
 

\end{supertabular}
}

\end{document}
