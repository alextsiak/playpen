
\documentclass{article}
\usepackage{colortbl}
\usepackage{makecell}
\usepackage{multirow}
\usepackage{supertabular}

\begin{document}

\newcounter{utterance}

\twocolumn

{ \footnotesize  \setcounter{utterance}{1}
\setlength{\tabcolsep}{0pt}
\begin{supertabular}{c@{$\;$}|p{.15\linewidth}@{}p{.15\linewidth}p{.15\linewidth}p{.15\linewidth}p{.15\linewidth}p{.15\linewidth}}

    \# & $\;$A & \multicolumn{4}{c}{Game Master} & $\;\:$B\\
    \hline 

    \theutterance \stepcounter{utterance}  

    & & \multicolumn{4}{p{0.6\linewidth}}{\cellcolor[rgb]{0.9,0.9,0.9}{%
	\makecell[{{p{\linewidth}}}]{% 
	  \tt {\tiny [A$\langle$GM]}  
	 You are an expert in causal inference. The following question is not a typical commonsense query, but rather a meticulously designed question created by a professor specializing in causal inference, intended to assess the students' mastery of the course content.\\ \tt Imagine a self-contained, hypothetical world with only the following conditions, and without any unmentioned factors or causal relationships: Personality has a direct effect on relationship status. Appearance has a direct effect on relationship status. The overall probability of kindness is 85%. For people with mean personalities and single people, the probability of attractive appearance is 38%. For people with mean personalities and in a relationship, the probability of attractive appearance is 71%. For people with kind personalities and single people, the probability of attractive appearance is 32%. For people with kind personalities and in a relationship, the probability of attractive appearance is 57%. If we look at people in a relationship, does the chance of attractive appearance decrease when kindness?\\ \tt Start your answer with 'Yes' or 'No', followed by additional reasoning or evidence to support your explanation. 
	  } 
	   } 
	   } 
	 & \\ 
 

    \theutterance \stepcounter{utterance}  

    & \multicolumn{4}{p{0.6\linewidth}}{\cellcolor[rgb]{0.8,1,0.9}{%
	\makecell[{{p{\linewidth}}}]{% 
	  \tt {\tiny [A$\rangle$GM]}  
	 No, the chance of attractive appearance does not decrease when kindness, given that we are looking at people in a relationship.\\ \tt \\ \tt To understand this, let's break down the given information:\\ \tt \\ \tt 1. For people with mean personalities and single people, the probability of attractive appearance is 38%.\\ \tt 2. For people with mean personalities and in a relationship, the probability of attractive appearance is 71%.\\ \tt 3. For people with kind personalities and single people, the probability of attractive appearance is 32%.\\ \tt 4. For people with kind personalities and in a relationship, the probability of attractive appearance is 57%.\\ \tt \\ \tt When we compare the probabilities of attractive appearance for people with mean and kind personalities, we see that the difference is more pronounced for single people (38% vs 32%) than for people in a relationship (71% vs 57%). This suggests that the effect of personality on appearance is more significant when people are single.\\ \tt \\ \tt However, when we are looking at people in a relationship, the difference in probabilities of attractive appearance between mean and kind personalities (71% vs 57%) is relatively small. This implies that the effect of kindness on appearance is less significant when people are already in a relationship.\\ \tt \\ \tt In other words, the presence of a relationship seems to mitigate the effect of kindness on appearance, making the difference between mean and kind personalities less pronounced. Therefore, the chance of attractive appearance does not decrease when kindness, given that we are looking at people in a relationship. 
	  } 
	   } 
	   } 
	 & & \\ 
 

    \theutterance \stepcounter{utterance}  

    & & & \multicolumn{2}{p{0.3\linewidth}}{\cellcolor[rgb]{0.95,0.95,0.95}{%
	\makecell[{{p{\linewidth}}}]{% 
	  \tt {\tiny [GM$|$GM]}  
	 No 
	  } 
	   } 
	   } 
	 & & \\ 
 

    \theutterance \stepcounter{utterance}  

    & & & \multicolumn{2}{p{0.3\linewidth}}{\cellcolor[rgb]{0.95,0.95,0.95}{%
	\makecell[{{p{\linewidth}}}]{% 
	  \tt {\tiny [GM$|$GM]}  
	 yes 
	  } 
	   } 
	   } 
	 & & \\ 
 

    \theutterance \stepcounter{utterance}  

    & & & \multicolumn{2}{p{0.3\linewidth}}{\cellcolor[rgb]{0.95,0.95,0.95}{%
	\makecell[{{p{\linewidth}}}]{% 
	  \tt {\tiny [GM$|$GM]}  
	 game_result = LOSE 
	  } 
	   } 
	   } 
	 & & \\ 
 

\end{supertabular}
}

\end{document}
