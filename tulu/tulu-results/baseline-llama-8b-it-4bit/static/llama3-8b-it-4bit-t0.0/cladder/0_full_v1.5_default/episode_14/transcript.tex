
\documentclass{article}
\usepackage{colortbl}
\usepackage{makecell}
\usepackage{multirow}
\usepackage{supertabular}

\begin{document}

\newcounter{utterance}

\twocolumn

{ \footnotesize  \setcounter{utterance}{1}
\setlength{\tabcolsep}{0pt}
\begin{supertabular}{c@{$\;$}|p{.15\linewidth}@{}p{.15\linewidth}p{.15\linewidth}p{.15\linewidth}p{.15\linewidth}p{.15\linewidth}}

    \# & $\;$A & \multicolumn{4}{c}{Game Master} & $\;\:$B\\
    \hline 

    \theutterance \stepcounter{utterance}  

    & & \multicolumn{4}{p{0.6\linewidth}}{\cellcolor[rgb]{0.9,0.9,0.9}{%
	\makecell[{{p{\linewidth}}}]{% 
	  \tt {\tiny [A$\langle$GM]}  
	 You are an expert in causal inference. The following question is not a typical commonsense query, but rather a meticulously designed question created by a professor specializing in causal inference, intended to assess the students' mastery of the course content.\\ \tt Imagine a self-contained, hypothetical world with only the following conditions, and without any unmentioned factors or causal relationships: Gender has a direct effect on smoking and lung cancer. Smoking has a direct effect on tar deposit. Tar deposit has a direct effect on lung cancer. Gender is unobserved. The overall probability of smoking is 12%. For nonsmokers, the probability of lung cancer is 58%. For smokers, the probability of lung cancer is 61%. Is lung cancer more likely than absence of lung cancer overall?\\ \tt Start your answer with 'Yes' or 'No', followed by additional reasoning or evidence to support your explanation. 
	  } 
	   } 
	   } 
	 & \\ 
 

    \theutterance \stepcounter{utterance}  

    & \multicolumn{4}{p{0.6\linewidth}}{\cellcolor[rgb]{0.8,1,0.9}{%
	\makecell[{{p{\linewidth}}}]{% 
	  \tt {\tiny [A$\rangle$GM]}  
	 No, lung cancer is not more likely than its absence overall.\\ \tt \\ \tt To understand this, let's break down the given information:\\ \tt \\ \tt 1. The probability of smoking is 12%.\\ \tt 2. For nonsmokers, the probability of lung cancer is 58%.\\ \tt 3. For smokers, the probability of lung cancer is 61%.\\ \tt \\ \tt We are not given the probability of lung cancer among nonsmokers or smokers directly, but we can infer it from the given information. Let's denote the probability of lung cancer among nonsmokers as P(LC|NS) and the probability of lung cancer among smokers as P(LC|S).\\ \tt \\ \tt We know that the overall probability of lung cancer is a weighted average of the probabilities of lung cancer among smokers and nonsmokers, where the weights are the probabilities of smoking and nonsmoking. However, we don't know the exact probabilities of lung cancer among smokers and nonsmokers.\\ \tt \\ \tt To assess whether lung cancer is more likely than its absence overall, we need to compare the overall probability of lung cancer to the overall probability of its absence. However, we are not given the overall probability of lung cancer or its absence directly.\\ \tt \\ \tt To make progress, let's consider the following: if we knew the overall probability of lung cancer, we could compare it to the overall probability of its absence. However, we are not given this information. \\ \tt \\ \tt However, we can use the given information to derive the overall probability of lung cancer. Let's denote the overall probability of lung cancer 
	  } 
	   } 
	   } 
	 & & \\ 
 

    \theutterance \stepcounter{utterance}  

    & & & \multicolumn{2}{p{0.3\linewidth}}{\cellcolor[rgb]{0.95,0.95,0.95}{%
	\makecell[{{p{\linewidth}}}]{% 
	  \tt {\tiny [GM$|$GM]}  
	 No 
	  } 
	   } 
	   } 
	 & & \\ 
 

    \theutterance \stepcounter{utterance}  

    & & & \multicolumn{2}{p{0.3\linewidth}}{\cellcolor[rgb]{0.95,0.95,0.95}{%
	\makecell[{{p{\linewidth}}}]{% 
	  \tt {\tiny [GM$|$GM]}  
	 yes 
	  } 
	   } 
	   } 
	 & & \\ 
 

    \theutterance \stepcounter{utterance}  

    & & & \multicolumn{2}{p{0.3\linewidth}}{\cellcolor[rgb]{0.95,0.95,0.95}{%
	\makecell[{{p{\linewidth}}}]{% 
	  \tt {\tiny [GM$|$GM]}  
	 game_result = LOSE 
	  } 
	   } 
	   } 
	 & & \\ 
 

\end{supertabular}
}

\end{document}
