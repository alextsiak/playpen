
\documentclass{article}
\usepackage{colortbl}
\usepackage{makecell}
\usepackage{multirow}
\usepackage{supertabular}

\begin{document}

\newcounter{utterance}

\twocolumn

{ \footnotesize  \setcounter{utterance}{1}
\setlength{\tabcolsep}{0pt}
\begin{supertabular}{c@{$\;$}|p{.15\linewidth}@{}p{.15\linewidth}p{.15\linewidth}p{.15\linewidth}p{.15\linewidth}p{.15\linewidth}}

    \# & $\;$A & \multicolumn{4}{c}{Game Master} & $\;\:$B\\
    \hline 

    \theutterance \stepcounter{utterance}  

    & & \multicolumn{4}{p{0.6\linewidth}}{\cellcolor[rgb]{0.9,0.9,0.9}{%
	\makecell[{{p{\linewidth}}}]{% 
	  \tt {\tiny [A$\langle$GM]}  
	 You are an expert in causal inference. The following question is not a typical commonsense query, but rather a meticulously designed question created by a professor specializing in causal inference, intended to assess the students' mastery of the course content.\\ \tt Imagine a self-contained, hypothetical world with only the following conditions, and without any unmentioned factors or causal relationships: CEO has a direct effect on director and having visited England. Having visited England has a direct effect on employee. Director has a direct effect on employee. For CEOs who fire employees and have not visited England, the probability of employee being fired is 21%. For CEOs who fire employees and have visited England, the probability of employee being fired is 51%. For CEOs who fire employees and have not visited England, the probability of employee being fired is 35%. For CEOs who fire employees and have visited England, the probability of employee being fired is 66%. The overall probability of CEO's decision to fire the employee is 28%. Will having visited England increase the chance of employee being fired?\\ \tt Start your answer with 'Yes' or 'No', followed by additional reasoning or evidence to support your explanation. 
	  } 
	   } 
	   } 
	 & \\ 
 

    \theutterance \stepcounter{utterance}  

    & \multicolumn{4}{p{0.6\linewidth}}{\cellcolor[rgb]{0.8,1,0.9}{%
	\makecell[{{p{\linewidth}}}]{% 
	  \tt {\tiny [A$\rangle$GM]}  
	 No, having visited England will not increase the chance of employee being fired.\\ \tt \\ \tt To understand this, let's break down the given information:\\ \tt \\ \tt 1. The probability of employee being fired given that the CEO has fired employees and has not visited England is 21%.\\ \tt 2. The probability of employee being fired given that the CEO has fired employees and has visited England is 51%.\\ \tt 3. However, the overall probability of CEO's decision to fire the employee is 28%.\\ \tt \\ \tt The key insight here is to recognize that the effect of visiting England on the probability of employee being fired is not a direct effect, but rather a mediated effect through the director. The given probabilities are conditional on the CEO's decision to fire employees, which is the common cause.\\ \tt \\ \tt To assess the effect of visiting England on the probability of employee being fired, we need to consider the joint effect of visiting England and the CEO's decision to fire employees. The given probabilities suggest that visiting England increases the probability of employee being fired given that the CEO has already decided to fire employees.\\ \tt \\ \tt However, the overall probability of CEO's decision to fire the employee is 28%, which is lower than the probability of employee being fired given that the CEO has visited England (51%). This suggests that visiting England does not increase the overall probability of employee being fired.\\ \tt \\ \tt In fact, the probability of employee being fired given that the CEO has not visited England (21%) is lower than the overall probability of CEO's decision to fire the employee (28%). This implies that 
	  } 
	   } 
	   } 
	 & & \\ 
 

    \theutterance \stepcounter{utterance}  

    & & & \multicolumn{2}{p{0.3\linewidth}}{\cellcolor[rgb]{0.95,0.95,0.95}{%
	\makecell[{{p{\linewidth}}}]{% 
	  \tt {\tiny [GM$|$GM]}  
	 No 
	  } 
	   } 
	   } 
	 & & \\ 
 

    \theutterance \stepcounter{utterance}  

    & & & \multicolumn{2}{p{0.3\linewidth}}{\cellcolor[rgb]{0.95,0.95,0.95}{%
	\makecell[{{p{\linewidth}}}]{% 
	  \tt {\tiny [GM$|$GM]}  
	 yes 
	  } 
	   } 
	   } 
	 & & \\ 
 

    \theutterance \stepcounter{utterance}  

    & & & \multicolumn{2}{p{0.3\linewidth}}{\cellcolor[rgb]{0.95,0.95,0.95}{%
	\makecell[{{p{\linewidth}}}]{% 
	  \tt {\tiny [GM$|$GM]}  
	 game_result = LOSE 
	  } 
	   } 
	   } 
	 & & \\ 
 

\end{supertabular}
}

\end{document}
