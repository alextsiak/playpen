
\documentclass{article}
\usepackage{colortbl}
\usepackage{makecell}
\usepackage{multirow}
\usepackage{supertabular}

\begin{document}

\newcounter{utterance}

\twocolumn

{ \footnotesize  \setcounter{utterance}{1}
\setlength{\tabcolsep}{0pt}
\begin{supertabular}{c@{$\;$}|p{.15\linewidth}@{}p{.15\linewidth}p{.15\linewidth}p{.15\linewidth}p{.15\linewidth}p{.15\linewidth}}

    \# & $\;$A & \multicolumn{4}{c}{Game Master} & $\;\:$B\\
    \hline 

    \theutterance \stepcounter{utterance}  

    & & \multicolumn{4}{p{0.6\linewidth}}{\cellcolor[rgb]{0.9,0.9,0.9}{%
	\makecell[{{p{\linewidth}}}]{% 
	  \tt {\tiny [A$\langle$GM]}  
	 You are an expert in causal inference. The following question is not a typical commonsense query, but rather a meticulously designed question created by a professor specializing in causal inference, intended to assess the students' mastery of the course content.\\ \tt Imagine a self-contained, hypothetical world with only the following conditions, and without any unmentioned factors or causal relationships: Yomx has a direct effect on gwet. Gwet has a direct effect on xevu. For those who are not yomx, the probability of xevu is 68%. For those who are yomx, the probability of xevu is 40%. For those who are yomx, would it be less likely to see xevu if the individual was not yomx?\\ \tt Start your answer with 'Yes' or 'No', followed by additional reasoning or evidence to support your explanation. 
	  } 
	   } 
	   } 
	 & \\ 
 

    \theutterance \stepcounter{utterance}  

    & \multicolumn{4}{p{0.6\linewidth}}{\cellcolor[rgb]{0.8,1,0.9}{%
	\makecell[{{p{\linewidth}}}]{% 
	  \tt {\tiny [A$\rangle$GM]}  
	 No.\\ \tt \\ \tt To answer this question, we need to apply the concept of conditional probability and the causal relationship between the variables. \\ \tt \\ \tt Let's denote the probability of xevu given yomx as P(xevu|yomx) and the probability of xevu given not yomx as P(xevu|not yomx). We are given that P(xevu|yomx) = 40% and P(xevu|not yomx) = 68%.\\ \tt \\ \tt The question asks whether it would be less likely to see xevu if the individual was not yomx, given that they are yomx. This is a counterfactual question, which can be addressed using the potential outcomes framework.\\ \tt \\ \tt Let's define the potential outcomes as follows:\\ \tt \\ \tt - Yomx = 1 (individual is yomx)\\ \tt - Yomx = 0 (individual is not yomx)\\ \tt - Xevu = 1 (individual has xevu)\\ \tt - Xevu = 0 (individual does not have xevu)\\ \tt \\ \tt We can represent the given probabilities as follows:\\ \tt \\ \tt - P(Xevu = 1 | Yomx = 1) = 40%\\ \tt - P(Xevu = 1 | Yomx = 0) = 68%\\ \tt \\ \tt The question asks whether it would be less likely to see xevu if the individual was not yomx, given that they are yomx. 
	  } 
	   } 
	   } 
	 & & \\ 
 

    \theutterance \stepcounter{utterance}  

    & & & \multicolumn{2}{p{0.3\linewidth}}{\cellcolor[rgb]{0.95,0.95,0.95}{%
	\makecell[{{p{\linewidth}}}]{% 
	  \tt {\tiny [GM$|$GM]}  
	 No 
	  } 
	   } 
	   } 
	 & & \\ 
 

    \theutterance \stepcounter{utterance}  

    & & & \multicolumn{2}{p{0.3\linewidth}}{\cellcolor[rgb]{0.95,0.95,0.95}{%
	\makecell[{{p{\linewidth}}}]{% 
	  \tt {\tiny [GM$|$GM]}  
	 no 
	  } 
	   } 
	   } 
	 & & \\ 
 

    \theutterance \stepcounter{utterance}  

    & & & \multicolumn{2}{p{0.3\linewidth}}{\cellcolor[rgb]{0.95,0.95,0.95}{%
	\makecell[{{p{\linewidth}}}]{% 
	  \tt {\tiny [GM$|$GM]}  
	 game_result = WIN 
	  } 
	   } 
	   } 
	 & & \\ 
 

\end{supertabular}
}

\end{document}
