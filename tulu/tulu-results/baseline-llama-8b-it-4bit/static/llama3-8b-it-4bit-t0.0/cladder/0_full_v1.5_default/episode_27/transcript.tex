
\documentclass{article}
\usepackage{colortbl}
\usepackage{makecell}
\usepackage{multirow}
\usepackage{supertabular}

\begin{document}

\newcounter{utterance}

\twocolumn

{ \footnotesize  \setcounter{utterance}{1}
\setlength{\tabcolsep}{0pt}
\begin{supertabular}{c@{$\;$}|p{.15\linewidth}@{}p{.15\linewidth}p{.15\linewidth}p{.15\linewidth}p{.15\linewidth}p{.15\linewidth}}

    \# & $\;$A & \multicolumn{4}{c}{Game Master} & $\;\:$B\\
    \hline 

    \theutterance \stepcounter{utterance}  

    & & \multicolumn{4}{p{0.6\linewidth}}{\cellcolor[rgb]{0.9,0.9,0.9}{%
	\makecell[{{p{\linewidth}}}]{% 
	  \tt {\tiny [A$\langle$GM]}  
	 You are an expert in causal inference. The following question is not a typical commonsense query, but rather a meticulously designed question created by a professor specializing in causal inference, intended to assess the students' mastery of the course content.\\ \tt Imagine a self-contained, hypothetical world with only the following conditions, and without any unmentioned factors or causal relationships: Parents' intelligence has a direct effect on parents' social status and child's intelligence. Other unobserved factors has a direct effect on parents' social status and child's intelligence. Parents' social status has a direct effect on child's intelligence. Other unobserved factors is unobserved. The overall probability of intelligent parents is 15%. The probability of unintelligent parents and intelligent child is 54%. The probability of intelligent parents and intelligent child is 4%. Is the chance of intelligent child smaller when observing intelligent parents?\\ \tt Start your answer with 'Yes' or 'No', followed by additional reasoning or evidence to support your explanation. 
	  } 
	   } 
	   } 
	 & \\ 
 

    \theutterance \stepcounter{utterance}  

    & \multicolumn{4}{p{0.6\linewidth}}{\cellcolor[rgb]{0.8,1,0.9}{%
	\makecell[{{p{\linewidth}}}]{% 
	  \tt {\tiny [A$\rangle$GM]}  
	 No.\\ \tt \\ \tt To address this question, we need to apply the principles of causal inference, specifically the concept of confounding variables and the potential outcome (PO) and counterfactual (CF) framework. \\ \tt \\ \tt Let's break down the given information:\\ \tt \\ \tt 1. Parents' intelligence has a direct effect on parents' social status and child's intelligence.\\ \tt 2. Other unobserved factors have a direct effect on parents' social status and child's intelligence.\\ \tt 3. Parents' social status has a direct effect on child's intelligence.\\ \tt 4. The overall probability of intelligent parents is 15%.\\ \tt 5. The probability of unintelligent parents and intelligent child is 54%.\\ \tt 6. The probability of intelligent parents and intelligent child is 4%.\\ \tt \\ \tt We are asked to compare the probability of an intelligent child when observing intelligent parents versus when not observing intelligent parents. \\ \tt \\ \tt To do this, we need to consider the potential outcomes (PO) and counterfactual (CF) framework. Let's define the following variables:\\ \tt \\ \tt - I: Intelligent parents\\ \tt - U: Unintelligent parents\\ \tt - C: Intelligent child\\ \tt - S: High social status\\ \tt - L: Low social status\\ \tt \\ \tt We can represent the given probabilities as follows:\\ \tt \\ \tt - P(I) = 0.15\\ \tt - P(U) = 0.85 (since the overall probability of intelligent parents is 15%)\\ \tt - P(C|U) = 0.54 (probability of intelligent child given unintelligent parents)\\ \tt - P(C|I) = 
	  } 
	   } 
	   } 
	 & & \\ 
 

    \theutterance \stepcounter{utterance}  

    & & & \multicolumn{2}{p{0.3\linewidth}}{\cellcolor[rgb]{0.95,0.95,0.95}{%
	\makecell[{{p{\linewidth}}}]{% 
	  \tt {\tiny [GM$|$GM]}  
	 No 
	  } 
	   } 
	   } 
	 & & \\ 
 

    \theutterance \stepcounter{utterance}  

    & & & \multicolumn{2}{p{0.3\linewidth}}{\cellcolor[rgb]{0.95,0.95,0.95}{%
	\makecell[{{p{\linewidth}}}]{% 
	  \tt {\tiny [GM$|$GM]}  
	 yes 
	  } 
	   } 
	   } 
	 & & \\ 
 

    \theutterance \stepcounter{utterance}  

    & & & \multicolumn{2}{p{0.3\linewidth}}{\cellcolor[rgb]{0.95,0.95,0.95}{%
	\makecell[{{p{\linewidth}}}]{% 
	  \tt {\tiny [GM$|$GM]}  
	 game_result = LOSE 
	  } 
	   } 
	   } 
	 & & \\ 
 

\end{supertabular}
}

\end{document}
