
\documentclass{article}
\usepackage{colortbl}
\usepackage{makecell}
\usepackage{multirow}
\usepackage{supertabular}

\begin{document}

\newcounter{utterance}

\twocolumn

{ \footnotesize  \setcounter{utterance}{1}
\setlength{\tabcolsep}{0pt}
\begin{supertabular}{c@{$\;$}|p{.15\linewidth}@{}p{.15\linewidth}p{.15\linewidth}p{.15\linewidth}p{.15\linewidth}p{.15\linewidth}}

    \# & $\;$A & \multicolumn{4}{c}{Game Master} & $\;\:$B\\
    \hline 

    \theutterance \stepcounter{utterance}  

    & & \multicolumn{4}{p{0.6\linewidth}}{\cellcolor[rgb]{0.9,0.9,0.9}{%
	\makecell[{{p{\linewidth}}}]{% 
	  \tt {\tiny [A$\langle$GM]}  
	 You are an expert in causal inference. The following question is not a typical commonsense query, but rather a meticulously designed question created by a professor specializing in causal inference, intended to assess the students' mastery of the course content.\\ \tt Imagine a self-contained, hypothetical world with only the following conditions, and without any unmentioned factors or causal relationships: CEO has a direct effect on director and manager. Manager has a direct effect on foot size. Director has a direct effect on foot size. For CEOs who fire employees and managers who don't sign termination letters, the probability of large feet is 62%. For CEOs who fire employees and managers who sign termination letters, the probability of large feet is 15%. For CEOs who fire employees and managers who don't sign termination letters, the probability of large feet is 83%. For CEOs who fire employees and managers who sign termination letters, the probability of large feet is 35%. The overall probability of CEO's decision to fire the employee is 94%. Will manager signing the termination letter increase the chance of large feet?\\ \tt Start your answer with 'Yes' or 'No', followed by additional reasoning or evidence to support your explanation. 
	  } 
	   } 
	   } 
	 & \\ 
 

    \theutterance \stepcounter{utterance}  

    & \multicolumn{4}{p{0.6\linewidth}}{\cellcolor[rgb]{0.8,1,0.9}{%
	\makecell[{{p{\linewidth}}}]{% 
	  \tt {\tiny [A$\rangle$GM]}  
	 Yes, manager signing the termination letter will increase the chance of large feet.\\ \tt \\ \tt To understand this, let's break down the given probabilities and the causal relationships:\\ \tt \\ \tt 1. CEO has a direct effect on director and manager.\\ \tt 2. Manager has a direct effect on foot size.\\ \tt 3. Director has a direct effect on foot size.\\ \tt \\ \tt We are given the following probabilities:\\ \tt \\ \tt - For CEOs who fire employees and managers who don't sign termination letters, the probability of large feet is 83%.\\ \tt - For CEOs who fire employees and managers who sign termination letters, the probability of large feet is 35%.\\ \tt \\ \tt We are also given the overall probability of CEO's decision to fire the employee, which is 94%. This implies that the probability of a CEO not firing an employee is 6%.\\ \tt \\ \tt We can calculate the probability of large feet for CEOs who don't fire employees by using the law of total probability:\\ \tt \\ \tt P(large feet | don't fire) = P(large feet | fire and sign) * P(fire and sign) + P(large feet | fire and don't sign) * P(fire and don't sign)\\ \tt = 35% * (6% / 94%) + 83% * (6% / 94%)\\ \tt = 35% * 0.0638 + 83% * 0.0638\\ \tt = 2.23% + 5.29%\\ \tt = 7.52%\\ \tt \\ \tt Now, we can compare the probabilities of large feet for two 
	  } 
	   } 
	   } 
	 & & \\ 
 

    \theutterance \stepcounter{utterance}  

    & & & \multicolumn{2}{p{0.3\linewidth}}{\cellcolor[rgb]{0.95,0.95,0.95}{%
	\makecell[{{p{\linewidth}}}]{% 
	  \tt {\tiny [GM$|$GM]}  
	 Yes 
	  } 
	   } 
	   } 
	 & & \\ 
 

    \theutterance \stepcounter{utterance}  

    & & & \multicolumn{2}{p{0.3\linewidth}}{\cellcolor[rgb]{0.95,0.95,0.95}{%
	\makecell[{{p{\linewidth}}}]{% 
	  \tt {\tiny [GM$|$GM]}  
	 no 
	  } 
	   } 
	   } 
	 & & \\ 
 

    \theutterance \stepcounter{utterance}  

    & & & \multicolumn{2}{p{0.3\linewidth}}{\cellcolor[rgb]{0.95,0.95,0.95}{%
	\makecell[{{p{\linewidth}}}]{% 
	  \tt {\tiny [GM$|$GM]}  
	 game_result = LOSE 
	  } 
	   } 
	   } 
	 & & \\ 
 

\end{supertabular}
}

\end{document}
