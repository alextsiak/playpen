
\documentclass{article}
\usepackage{colortbl}
\usepackage{makecell}
\usepackage{multirow}
\usepackage{supertabular}

\begin{document}

\newcounter{utterance}

\twocolumn

{ \footnotesize  \setcounter{utterance}{1}
\setlength{\tabcolsep}{0pt}
\begin{supertabular}{c@{$\;$}|p{.15\linewidth}@{}p{.15\linewidth}p{.15\linewidth}p{.15\linewidth}p{.15\linewidth}p{.15\linewidth}}

    \# & $\;$A & \multicolumn{4}{c}{Game Master} & $\;\:$B\\
    \hline 

    \theutterance \stepcounter{utterance}  

    & & \multicolumn{4}{p{0.6\linewidth}}{\cellcolor[rgb]{0.9,0.9,0.9}{%
	\makecell[{{p{\linewidth}}}]{% 
	  \tt {\tiny [A$\langle$GM]}  
	 You are an expert in causal inference. The following question is not a typical commonsense query, but rather a meticulously designed question created by a professor specializing in causal inference, intended to assess the students' mastery of the course content.\\ \tt Imagine a self-contained, hypothetical world with only the following conditions, and without any unmentioned factors or causal relationships: Appearance has a direct effect on fame. Talent has a direct effect on fame. The overall probability of attractive appearance is 56%. For people considered unattractive, the probability of talent is 9%. For people considered attractive, the probability of talent is 9%. Is talent more likely than lack of talent overall?\\ \tt Start your answer with 'Yes' or 'No', followed by additional reasoning or evidence to support your explanation. 
	  } 
	   } 
	   } 
	 & \\ 
 

    \theutterance \stepcounter{utterance}  

    & \multicolumn{4}{p{0.6\linewidth}}{\cellcolor[rgb]{0.8,1,0.9}{%
	\makecell[{{p{\linewidth}}}]{% 
	  \tt {\tiny [A$\rangle$GM]}  
	 No, talent is not more likely than lack of talent overall.\\ \tt \\ \tt To understand this, let's break down the given information:\\ \tt \\ \tt 1. Appearance has a direct effect on fame.\\ \tt 2. Talent has a direct effect on fame.\\ \tt 3. The overall probability of attractive appearance is 56%.\\ \tt 4. For people considered unattractive, the probability of talent is 9%.\\ \tt 5. For people considered attractive, the probability of talent is 9%.\\ \tt \\ \tt We are asked to compare the overall probability of talent versus lack of talent. To do this, we need to consider the joint probability of talent and appearance.\\ \tt \\ \tt Since the probability of talent is 9% for both attractive and unattractive individuals, and the overall probability of attractive appearance is 56%, we can infer that the remaining 44% of the population is unattractive.\\ \tt \\ \tt Now, let's calculate the overall probability of talent. We can do this by considering two scenarios:\\ \tt \\ \tt Scenario 1: Attractive individuals with talent\\ \tt - Probability of attractive appearance: 56%\\ \tt - Probability of talent given attractive appearance: 9%\\ \tt - Joint probability of attractive appearance and talent: 56% * 9% = 5.04%\\ \tt \\ \tt Scenario 2: Unattractive individuals with talent\\ \tt - Probability of unattractive appearance: 44% (1 - 56%)\\ \tt - Probability of talent given unattractive appearance: 9%\\ \tt - Joint probability of unattractive appearance and talent: 44% * 9% = 3 
	  } 
	   } 
	   } 
	 & & \\ 
 

    \theutterance \stepcounter{utterance}  

    & & & \multicolumn{2}{p{0.3\linewidth}}{\cellcolor[rgb]{0.95,0.95,0.95}{%
	\makecell[{{p{\linewidth}}}]{% 
	  \tt {\tiny [GM$|$GM]}  
	 No 
	  } 
	   } 
	   } 
	 & & \\ 
 

    \theutterance \stepcounter{utterance}  

    & & & \multicolumn{2}{p{0.3\linewidth}}{\cellcolor[rgb]{0.95,0.95,0.95}{%
	\makecell[{{p{\linewidth}}}]{% 
	  \tt {\tiny [GM$|$GM]}  
	 no 
	  } 
	   } 
	   } 
	 & & \\ 
 

    \theutterance \stepcounter{utterance}  

    & & & \multicolumn{2}{p{0.3\linewidth}}{\cellcolor[rgb]{0.95,0.95,0.95}{%
	\makecell[{{p{\linewidth}}}]{% 
	  \tt {\tiny [GM$|$GM]}  
	 game_result = WIN 
	  } 
	   } 
	   } 
	 & & \\ 
 

\end{supertabular}
}

\end{document}
