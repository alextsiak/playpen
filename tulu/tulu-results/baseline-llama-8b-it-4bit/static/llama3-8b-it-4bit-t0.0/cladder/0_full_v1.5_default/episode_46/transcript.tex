
\documentclass{article}
\usepackage{colortbl}
\usepackage{makecell}
\usepackage{multirow}
\usepackage{supertabular}

\begin{document}

\newcounter{utterance}

\twocolumn

{ \footnotesize  \setcounter{utterance}{1}
\setlength{\tabcolsep}{0pt}
\begin{supertabular}{c@{$\;$}|p{.15\linewidth}@{}p{.15\linewidth}p{.15\linewidth}p{.15\linewidth}p{.15\linewidth}p{.15\linewidth}}

    \# & $\;$A & \multicolumn{4}{c}{Game Master} & $\;\:$B\\
    \hline 

    \theutterance \stepcounter{utterance}  

    & & \multicolumn{4}{p{0.6\linewidth}}{\cellcolor[rgb]{0.9,0.9,0.9}{%
	\makecell[{{p{\linewidth}}}]{% 
	  \tt {\tiny [A$\langle$GM]}  
	 You are an expert in causal inference. The following question is not a typical commonsense query, but rather a meticulously designed question created by a professor specializing in causal inference, intended to assess the students' mastery of the course content.\\ \tt Imagine a self-contained, hypothetical world with only the following conditions, and without any unmentioned factors or causal relationships: Kidney stone size has a direct effect on treatment and lip thickness. Treatment has a direct effect on lip thickness. The overall probability of receives treatment is 48%. For patients not receiving treatment, the probability of thick lips is 64%. For patients receiving treatment, the probability of thick lips is 29%. Is thick lips more likely than thin lips overall?\\ \tt Start your answer with 'Yes' or 'No', followed by additional reasoning or evidence to support your explanation. 
	  } 
	   } 
	   } 
	 & \\ 
 

    \theutterance \stepcounter{utterance}  

    & \multicolumn{4}{p{0.6\linewidth}}{\cellcolor[rgb]{0.8,1,0.9}{%
	\makecell[{{p{\linewidth}}}]{% 
	  \tt {\tiny [A$\rangle$GM]}  
	 No, thick lips are not more likely than thin lips overall.\\ \tt \\ \tt To analyze this problem, we need to apply the principles of causal inference, specifically the concept of confounding variables. In this scenario, treatment is a confounding variable that affects both the outcome (lip thickness) and the exposure (kidney stone size). \\ \tt \\ \tt We can use the following steps to reason about the problem:\\ \tt \\ \tt 1. **Identify the confounding variable**: Treatment is the confounding variable that affects both lip thickness and kidney stone size.\\ \tt \\ \tt 2. **Estimate the effect of treatment on lip thickness**: We are given that for patients not receiving treatment, the probability of thick lips is 64%, and for patients receiving treatment, the probability of thick lips is 29%. This suggests that treatment has a negative effect on lip thickness.\\ \tt \\ \tt 3. **Estimate the effect of kidney stone size on treatment**: We are not given direct information about the effect of kidney stone size on treatment, but we can infer it from the given probabilities. Since the overall probability of receiving treatment is 48%, and the probability of thick lips is 64% for patients not receiving treatment, it suggests that patients with larger kidney stones are more likely to receive treatment.\\ \tt \\ \tt 4. **Apply the concept of confounding variables**: Since treatment is a confounding variable, we need to adjust for its effect on lip thickness. If we were to estimate the effect of kidney stone size on lip thickness without adjusting for treatment, we would likely overestimate the effect 
	  } 
	   } 
	   } 
	 & & \\ 
 

    \theutterance \stepcounter{utterance}  

    & & & \multicolumn{2}{p{0.3\linewidth}}{\cellcolor[rgb]{0.95,0.95,0.95}{%
	\makecell[{{p{\linewidth}}}]{% 
	  \tt {\tiny [GM$|$GM]}  
	 No 
	  } 
	   } 
	   } 
	 & & \\ 
 

    \theutterance \stepcounter{utterance}  

    & & & \multicolumn{2}{p{0.3\linewidth}}{\cellcolor[rgb]{0.95,0.95,0.95}{%
	\makecell[{{p{\linewidth}}}]{% 
	  \tt {\tiny [GM$|$GM]}  
	 no 
	  } 
	   } 
	   } 
	 & & \\ 
 

    \theutterance \stepcounter{utterance}  

    & & & \multicolumn{2}{p{0.3\linewidth}}{\cellcolor[rgb]{0.95,0.95,0.95}{%
	\makecell[{{p{\linewidth}}}]{% 
	  \tt {\tiny [GM$|$GM]}  
	 game_result = WIN 
	  } 
	   } 
	   } 
	 & & \\ 
 

\end{supertabular}
}

\end{document}
