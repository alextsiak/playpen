
\documentclass{article}
\usepackage{colortbl}
\usepackage{makecell}
\usepackage{multirow}
\usepackage{supertabular}

\begin{document}

\newcounter{utterance}

\twocolumn

{ \footnotesize  \setcounter{utterance}{1}
\setlength{\tabcolsep}{0pt}
\begin{supertabular}{c@{$\;$}|p{.15\linewidth}@{}p{.15\linewidth}p{.15\linewidth}p{.15\linewidth}p{.15\linewidth}p{.15\linewidth}}

    \# & $\;$A & \multicolumn{4}{c}{Game Master} & $\;\:$B\\
    \hline 

    \theutterance \stepcounter{utterance}  

    & & \multicolumn{4}{p{0.6\linewidth}}{\cellcolor[rgb]{0.9,0.9,0.9}{%
	\makecell[{{p{\linewidth}}}]{% 
	  \tt {\tiny [A$\langle$GM]}  
	 You are an expert in causal inference. The following question is not a typical commonsense query, but rather a meticulously designed question created by a professor specializing in causal inference, intended to assess the students' mastery of the course content.\\ \tt Imagine a self-contained, hypothetical world with only the following conditions, and without any unmentioned factors or causal relationships: Kidney stone size has a direct effect on treatment and recovery. Treatment has a direct effect on recovery. Method 1: We look directly at how treatment correlates with recovery in general. Method 2: We look at this correlation case by case according to kidney stone size. To understand how treatment affects recovery, is it more correct to use the Method 1 than Method 2?\\ \tt Start your answer with 'Yes' or 'No', followed by additional reasoning or evidence to support your explanation. 
	  } 
	   } 
	   } 
	 & \\ 
 

    \theutterance \stepcounter{utterance}  

    & \multicolumn{4}{p{0.6\linewidth}}{\cellcolor[rgb]{0.8,1,0.9}{%
	\makecell[{{p{\linewidth}}}]{% 
	  \tt {\tiny [A$\rangle$GM]}  
	 No, it is not more correct to use Method 1 than Method 2.\\ \tt \\ \tt In this hypothetical world, the relationship between treatment and recovery is confounded by the underlying variable, kidney stone size. When we use Method 1, we are essentially looking at the correlation between treatment and recovery in general, without accounting for the potential effect of kidney stone size on both variables. This can lead to biased estimates of the treatment effect, as the relationship between treatment and recovery may be influenced by the distribution of kidney stone sizes in the population.\\ \tt \\ \tt In contrast, Method 2, which examines the correlation between treatment and recovery on a case-by-case basis according to kidney stone size, allows us to control for the confounding effect of kidney stone size. By stratifying the analysis by kidney stone size, we can estimate the treatment effect within each stratum, which can provide a more accurate representation of the relationship between treatment and recovery.\\ \tt \\ \tt This is an example of the fundamental problem of causal inference, where the relationship between the treatment and outcome is confounded by one or more underlying variables. By using Method 2, we can address this issue through stratification, which is a common technique used in causal inference to control for confounding variables.\\ \tt \\ \tt In fact, this problem is analogous to the well-known Simpson's paradox, where a correlation between two variables appears to be present when data are aggregated, but disappears when the data are stratified by a third variable. In this case, the third variable is kidney stone size, 
	  } 
	   } 
	   } 
	 & & \\ 
 

    \theutterance \stepcounter{utterance}  

    & & & \multicolumn{2}{p{0.3\linewidth}}{\cellcolor[rgb]{0.95,0.95,0.95}{%
	\makecell[{{p{\linewidth}}}]{% 
	  \tt {\tiny [GM$|$GM]}  
	 No 
	  } 
	   } 
	   } 
	 & & \\ 
 

    \theutterance \stepcounter{utterance}  

    & & & \multicolumn{2}{p{0.3\linewidth}}{\cellcolor[rgb]{0.95,0.95,0.95}{%
	\makecell[{{p{\linewidth}}}]{% 
	  \tt {\tiny [GM$|$GM]}  
	 no 
	  } 
	   } 
	   } 
	 & & \\ 
 

    \theutterance \stepcounter{utterance}  

    & & & \multicolumn{2}{p{0.3\linewidth}}{\cellcolor[rgb]{0.95,0.95,0.95}{%
	\makecell[{{p{\linewidth}}}]{% 
	  \tt {\tiny [GM$|$GM]}  
	 game_result = WIN 
	  } 
	   } 
	   } 
	 & & \\ 
 

\end{supertabular}
}

\end{document}
