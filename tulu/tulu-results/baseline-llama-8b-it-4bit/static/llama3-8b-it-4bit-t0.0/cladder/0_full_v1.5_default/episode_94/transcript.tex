
\documentclass{article}
\usepackage{colortbl}
\usepackage{makecell}
\usepackage{multirow}
\usepackage{supertabular}

\begin{document}

\newcounter{utterance}

\twocolumn

{ \footnotesize  \setcounter{utterance}{1}
\setlength{\tabcolsep}{0pt}
\begin{supertabular}{c@{$\;$}|p{.15\linewidth}@{}p{.15\linewidth}p{.15\linewidth}p{.15\linewidth}p{.15\linewidth}p{.15\linewidth}}

    \# & $\;$A & \multicolumn{4}{c}{Game Master} & $\;\:$B\\
    \hline 

    \theutterance \stepcounter{utterance}  

    & & \multicolumn{4}{p{0.6\linewidth}}{\cellcolor[rgb]{0.9,0.9,0.9}{%
	\makecell[{{p{\linewidth}}}]{% 
	  \tt {\tiny [A$\langle$GM]}  
	 You are an expert in causal inference. The following question is not a typical commonsense query, but rather a meticulously designed question created by a professor specializing in causal inference, intended to assess the students' mastery of the course content.\\ \tt Imagine a self-contained, hypothetical world with only the following conditions, and without any unmentioned factors or causal relationships: Xevo has a direct effect on gyzp. Tijv has a direct effect on gyzp. We know that xevo or tijv causes gyzp. We observed an individual is not tijv. Would an individual is gyzp if not xevo instead of xevo?\\ \tt Start your answer with 'Yes' or 'No', followed by additional reasoning or evidence to support your explanation. 
	  } 
	   } 
	   } 
	 & \\ 
 

    \theutterance \stepcounter{utterance}  

    & \multicolumn{4}{p{0.6\linewidth}}{\cellcolor[rgb]{0.8,1,0.9}{%
	\makecell[{{p{\linewidth}}}]{% 
	  \tt {\tiny [A$\rangle$GM]}  
	 No\\ \tt \\ \tt In this hypothetical world, we have two direct causes of gyzp: Xevo and Tijv. We are given that Xevo or Tijv causes gyzp, which implies that gyzp is a result of either Xevo or Tijv (or both). \\ \tt \\ \tt The question asks whether an individual would be gyzp if not Xevo instead of Xevo. To answer this, we need to consider the concept of causal sufficiency. Causal sufficiency implies that if a cause is present, the effect will occur, and if the cause is absent, the effect will not occur.\\ \tt \\ \tt Given that we know the individual is not Tijv, we can infer that Tijv is not a cause of gyzp for this individual. However, the presence of Xevo is still a cause of gyzp for this individual. \\ \tt \\ \tt The question is asking whether the absence of Xevo would prevent gyzp. Since we know that Xevo or Tijv causes gyzp, and Tijv is not a cause for this individual, we can conclude that Xevo is the cause of gyzp for this individual. \\ \tt \\ \tt Therefore, even if Xevo is absent, gyzp would still occur because Tijv is not a cause for this individual, and the question states that Xevo or Tijv causes gyzp. 
	  } 
	   } 
	   } 
	 & & \\ 
 

    \theutterance \stepcounter{utterance}  

    & & & \multicolumn{2}{p{0.3\linewidth}}{\cellcolor[rgb]{0.95,0.95,0.95}{%
	\makecell[{{p{\linewidth}}}]{% 
	  \tt {\tiny [GM$|$GM]}  
	 No 
	  } 
	   } 
	   } 
	 & & \\ 
 

    \theutterance \stepcounter{utterance}  

    & & & \multicolumn{2}{p{0.3\linewidth}}{\cellcolor[rgb]{0.95,0.95,0.95}{%
	\makecell[{{p{\linewidth}}}]{% 
	  \tt {\tiny [GM$|$GM]}  
	 no 
	  } 
	   } 
	   } 
	 & & \\ 
 

    \theutterance \stepcounter{utterance}  

    & & & \multicolumn{2}{p{0.3\linewidth}}{\cellcolor[rgb]{0.95,0.95,0.95}{%
	\makecell[{{p{\linewidth}}}]{% 
	  \tt {\tiny [GM$|$GM]}  
	 game_result = WIN 
	  } 
	   } 
	   } 
	 & & \\ 
 

\end{supertabular}
}

\end{document}
