
\documentclass{article}
\usepackage{colortbl}
\usepackage{makecell}
\usepackage{multirow}
\usepackage{supertabular}

\begin{document}

\newcounter{utterance}

\twocolumn

{ \footnotesize  \setcounter{utterance}{1}
\setlength{\tabcolsep}{0pt}
\begin{supertabular}{c@{$\;$}|p{.15\linewidth}@{}p{.15\linewidth}p{.15\linewidth}p{.15\linewidth}p{.15\linewidth}p{.15\linewidth}}

    \# & $\;$A & \multicolumn{4}{c}{Game Master} & $\;\:$B\\
    \hline 

    \theutterance \stepcounter{utterance}  

    & & \multicolumn{4}{p{0.6\linewidth}}{\cellcolor[rgb]{0.9,0.9,0.9}{%
	\makecell[{{p{\linewidth}}}]{% 
	  \tt {\tiny [A$\langle$GM]}  
	 You are an expert in causal inference. The following question is not a typical commonsense query, but rather a meticulously designed question created by a professor specializing in causal inference, intended to assess the students' mastery of the course content.\\ \tt Imagine a self-contained, hypothetical world with only the following conditions, and without any unmentioned factors or causal relationships: Jyka has a direct effect on hwax. Hwax has a direct effect on lirg. The overall probability of jyka is 14%. The probability of not jyka and lirg is 72%. The probability of jyka and lirg is 12%. Is the chance of lirg smaller when observing jyka?\\ \tt Start your answer with 'Yes' or 'No', followed by additional reasoning or evidence to support your explanation. 
	  } 
	   } 
	   } 
	 & \\ 
 

    \theutterance \stepcounter{utterance}  

    & \multicolumn{4}{p{0.6\linewidth}}{\cellcolor[rgb]{0.8,1,0.9}{%
	\makecell[{{p{\linewidth}}}]{% 
	  \tt {\tiny [A$\rangle$GM]}  
	 No, the chance of lirg is not smaller when observing jyka.\\ \tt \\ \tt To analyze this problem, we can use the concept of conditional probability and the causal diagram. Given the information:\\ \tt \\ \tt 1. Jyka has a direct effect on hwax.\\ \tt 2. Hwax has a direct effect on lirg.\\ \tt 3. P(Jyka) = 14%\\ \tt 4. P(¬Jyka ∧ Lirg) = 72%\\ \tt 5. P(Jyka ∧ Lirg) = 12%\\ \tt \\ \tt We can use the law of total probability to find the unconditional probability of Lirg:\\ \tt \\ \tt P(Lirg) = P(Lirg|Jyka)P(Jyka) + P(Lirg|¬Jyka)P(¬Jyka)\\ \tt \\ \tt We know that P(Jyka) = 14% and P(¬Jyka) = 1 - P(Jyka) = 86%. We also know that P(Lirg|Jyka ∧ ¬Jyka) = P(Lirg) because Lirg is a common effect of both Jyka and hwax. \\ \tt \\ \tt However, we cannot directly calculate P(Lirg|Jyka) and P(Lirg|¬Jyka) from the given information. We can use the given probabilities to find the conditional probability of Lirg given Jyka and ¬Jyka.\\ \tt \\ \tt P(Jyka 
	  } 
	   } 
	   } 
	 & & \\ 
 

    \theutterance \stepcounter{utterance}  

    & & & \multicolumn{2}{p{0.3\linewidth}}{\cellcolor[rgb]{0.95,0.95,0.95}{%
	\makecell[{{p{\linewidth}}}]{% 
	  \tt {\tiny [GM$|$GM]}  
	 No 
	  } 
	   } 
	   } 
	 & & \\ 
 

    \theutterance \stepcounter{utterance}  

    & & & \multicolumn{2}{p{0.3\linewidth}}{\cellcolor[rgb]{0.95,0.95,0.95}{%
	\makecell[{{p{\linewidth}}}]{% 
	  \tt {\tiny [GM$|$GM]}  
	 no 
	  } 
	   } 
	   } 
	 & & \\ 
 

    \theutterance \stepcounter{utterance}  

    & & & \multicolumn{2}{p{0.3\linewidth}}{\cellcolor[rgb]{0.95,0.95,0.95}{%
	\makecell[{{p{\linewidth}}}]{% 
	  \tt {\tiny [GM$|$GM]}  
	 game_result = WIN 
	  } 
	   } 
	   } 
	 & & \\ 
 

\end{supertabular}
}

\end{document}
