
\documentclass{article}
\usepackage{colortbl}
\usepackage{makecell}
\usepackage{multirow}
\usepackage{supertabular}

\begin{document}

\newcounter{utterance}

\twocolumn

{ \footnotesize  \setcounter{utterance}{1}
\setlength{\tabcolsep}{0pt}
\begin{supertabular}{c@{$\;$}|p{.15\linewidth}@{}p{.15\linewidth}p{.15\linewidth}p{.15\linewidth}p{.15\linewidth}p{.15\linewidth}}

    \# & $\;$A & \multicolumn{4}{c}{Game Master} & $\;\:$B\\
    \hline 

    \theutterance \stepcounter{utterance}  

    & & \multicolumn{4}{p{0.6\linewidth}}{\cellcolor[rgb]{0.9,0.9,0.9}{%
	\makecell[{{p{\linewidth}}}]{% 
	  \tt {\tiny [A$\langle$GM]}  
	 You are an expert in causal inference. The following question is not a typical commonsense query, but rather a meticulously designed question created by a professor specializing in causal inference, intended to assess the students' mastery of the course content.\\ \tt Imagine a self-contained, hypothetical world with only the following conditions, and without any unmentioned factors or causal relationships: Solar eclipse has a direct effect on Alice arriving to school. Traffic has a direct effect on Alice arriving to school. Method 1: We look at how solar eclipse correlates with Alice arriving to school case by case according to traffic. Method 2: We look directly at how solar eclipse correlates with Alice arriving to school in general. To understand how solar eclipse affects Alice arriving to school, is it more correct to use the Method 1 than Method 2?\\ \tt Start your answer with 'Yes' or 'No', followed by additional reasoning or evidence to support your explanation. 
	  } 
	   } 
	   } 
	 & \\ 
 

    \theutterance \stepcounter{utterance}  

    & \multicolumn{4}{p{0.6\linewidth}}{\cellcolor[rgb]{0.8,1,0.9}{%
	\makecell[{{p{\linewidth}}}]{% 
	  \tt {\tiny [A$\rangle$GM]}  
	 No, it is not more correct to use Method 1 than Method 2.\\ \tt \\ \tt In this hypothetical world, we have two direct causes: solar eclipse and traffic, both affecting Alice's arrival to school. When we use Method 1, we are essentially controlling for traffic by examining the correlation between solar eclipse and Alice's arrival on a case-by-case basis according to traffic. However, this approach is problematic because it introduces a form of "confounding by proxy" (Hernán & Robins, 2006).\\ \tt \\ \tt In other words, by conditioning on traffic, we are effectively using traffic as a surrogate for the underlying causal relationship between solar eclipse and Alice's arrival. This can lead to biased estimates of the effect of solar eclipse on Alice's arrival, as traffic itself may be influenced by other factors that are not accounted for in this hypothetical world.\\ \tt \\ \tt In contrast, Method 2 directly examines the correlation between solar eclipse and Alice's arrival in general, without controlling for traffic. This approach is more accurate because it allows us to estimate the direct effect of solar eclipse on Alice's arrival, without introducing the potential biases associated with controlling for a proxy variable (traffic).\\ \tt \\ \tt In the context of causal inference, this scenario illustrates the importance of distinguishing between direct and indirect effects, as well as the potential pitfalls of controlling for proxy variables. By using Method 2, we can obtain a more accurate estimate of the effect of solar eclipse on Alice's arrival, without the potential biases introduced by controlling for traffic.\\ \tt \\ \tt References 
	  } 
	   } 
	   } 
	 & & \\ 
 

    \theutterance \stepcounter{utterance}  

    & & & \multicolumn{2}{p{0.3\linewidth}}{\cellcolor[rgb]{0.95,0.95,0.95}{%
	\makecell[{{p{\linewidth}}}]{% 
	  \tt {\tiny [GM$|$GM]}  
	 No 
	  } 
	   } 
	   } 
	 & & \\ 
 

    \theutterance \stepcounter{utterance}  

    & & & \multicolumn{2}{p{0.3\linewidth}}{\cellcolor[rgb]{0.95,0.95,0.95}{%
	\makecell[{{p{\linewidth}}}]{% 
	  \tt {\tiny [GM$|$GM]}  
	 no 
	  } 
	   } 
	   } 
	 & & \\ 
 

    \theutterance \stepcounter{utterance}  

    & & & \multicolumn{2}{p{0.3\linewidth}}{\cellcolor[rgb]{0.95,0.95,0.95}{%
	\makecell[{{p{\linewidth}}}]{% 
	  \tt {\tiny [GM$|$GM]}  
	 game_result = WIN 
	  } 
	   } 
	   } 
	 & & \\ 
 

\end{supertabular}
}

\end{document}
