
\documentclass{article}
\usepackage{colortbl}
\usepackage{makecell}
\usepackage{multirow}
\usepackage{supertabular}

\begin{document}

\newcounter{utterance}

\twocolumn

{ \footnotesize  \setcounter{utterance}{1}
\setlength{\tabcolsep}{0pt}
\begin{supertabular}{c@{$\;$}|p{.15\linewidth}@{}p{.15\linewidth}p{.15\linewidth}p{.15\linewidth}p{.15\linewidth}p{.15\linewidth}}

    \# & $\;$A & \multicolumn{4}{c}{Game Master} & $\;\:$B\\
    \hline 

    \theutterance \stepcounter{utterance}  

    & & \multicolumn{4}{p{0.6\linewidth}}{\cellcolor[rgb]{0.9,0.9,0.9}{%
	\makecell[{{p{\linewidth}}}]{% 
	  \tt {\tiny [A$\langle$GM]}  
	 You are an expert in causal inference. The following question is not a typical commonsense query, but rather a meticulously designed question created by a professor specializing in causal inference, intended to assess the students' mastery of the course content.\\ \tt Imagine a self-contained, hypothetical world with only the following conditions, and without any unmentioned factors or causal relationships: Parents' intelligence has a direct effect on parents' social status and child's intelligence. Other unobserved factors has a direct effect on parents' social status and child's intelligence. Parents' social status has a direct effect on child's intelligence. Other unobserved factors is unobserved. We know that intelligent parents and confounder active causes high parental social status. intelligent parents and confounder active and high parental social status causes intelligent child. We observed confounder active. Would the child is intelligent if intelligent parents instead of unintelligent parents?\\ \tt Start your answer with 'Yes' or 'No', followed by additional reasoning or evidence to support your explanation. 
	  } 
	   } 
	   } 
	 & \\ 
 

    \theutterance \stepcounter{utterance}  

    & \multicolumn{4}{p{0.6\linewidth}}{\cellcolor[rgb]{0.8,1,0.9}{%
	\makecell[{{p{\linewidth}}}]{% 
	  \tt {\tiny [A$\rangle$GM]}  
	 Yes\\ \tt \\ \tt The given conditions suggest a complex causal relationship between parents' intelligence, social status, and child's intelligence. However, we can break down the problem using the given information.\\ \tt \\ \tt We know that intelligent parents and confounder active causes high parental social status. This implies that there is a causal relationship between intelligent parents and high parental social status, mediated by the confounder.\\ \tt \\ \tt We also know that intelligent parents and confounder active and high parental social status causes intelligent child. This implies that there is a causal relationship between intelligent parents, high parental social status, and child's intelligence.\\ \tt \\ \tt Given that we observed confounder active, we can infer that the confounder is active in this scenario. This means that the causal relationship between intelligent parents and high parental social status is also active.\\ \tt \\ \tt Now, the question is whether the child is intelligent if intelligent parents instead of unintelligent parents. We can use the concept of causal mediation to analyze this.\\ \tt \\ \tt Since intelligent parents cause high parental social status, and high parental social status causes child's intelligence, we can say that intelligent parents have a causal effect on child's intelligence through high parental social status.\\ \tt \\ \tt However, the question is whether this effect is direct or indirect. If the effect is direct, then intelligent parents would cause child's intelligence regardless of the confounder. But if the effect is indirect, then the confounder would mediate the relationship between intelligent parents and child's intelligence.\\ \tt \\ \tt Given that we observed confounder active, we can infer that the confounder is mediating the 
	  } 
	   } 
	   } 
	 & & \\ 
 

    \theutterance \stepcounter{utterance}  

    & & & \multicolumn{2}{p{0.3\linewidth}}{\cellcolor[rgb]{0.95,0.95,0.95}{%
	\makecell[{{p{\linewidth}}}]{% 
	  \tt {\tiny [GM$|$GM]}  
	 Yes 
	  } 
	   } 
	   } 
	 & & \\ 
 

    \theutterance \stepcounter{utterance}  

    & & & \multicolumn{2}{p{0.3\linewidth}}{\cellcolor[rgb]{0.95,0.95,0.95}{%
	\makecell[{{p{\linewidth}}}]{% 
	  \tt {\tiny [GM$|$GM]}  
	 yes 
	  } 
	   } 
	   } 
	 & & \\ 
 

    \theutterance \stepcounter{utterance}  

    & & & \multicolumn{2}{p{0.3\linewidth}}{\cellcolor[rgb]{0.95,0.95,0.95}{%
	\makecell[{{p{\linewidth}}}]{% 
	  \tt {\tiny [GM$|$GM]}  
	 game_result = WIN 
	  } 
	   } 
	   } 
	 & & \\ 
 

\end{supertabular}
}

\end{document}
