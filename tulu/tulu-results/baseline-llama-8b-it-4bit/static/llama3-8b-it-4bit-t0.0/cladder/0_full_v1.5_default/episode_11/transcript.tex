
\documentclass{article}
\usepackage{colortbl}
\usepackage{makecell}
\usepackage{multirow}
\usepackage{supertabular}

\begin{document}

\newcounter{utterance}

\twocolumn

{ \footnotesize  \setcounter{utterance}{1}
\setlength{\tabcolsep}{0pt}
\begin{supertabular}{c@{$\;$}|p{.15\linewidth}@{}p{.15\linewidth}p{.15\linewidth}p{.15\linewidth}p{.15\linewidth}p{.15\linewidth}}

    \# & $\;$A & \multicolumn{4}{c}{Game Master} & $\;\:$B\\
    \hline 

    \theutterance \stepcounter{utterance}  

    & & \multicolumn{4}{p{0.6\linewidth}}{\cellcolor[rgb]{0.9,0.9,0.9}{%
	\makecell[{{p{\linewidth}}}]{% 
	  \tt {\tiny [A$\langle$GM]}  
	 You are an expert in causal inference. The following question is not a typical commonsense query, but rather a meticulously designed question created by a professor specializing in causal inference, intended to assess the students' mastery of the course content.\\ \tt Imagine a self-contained, hypothetical world with only the following conditions, and without any unmentioned factors or causal relationships: Obesity has a direct effect on diabetes and lifespan. Smoking has a direct effect on diabetes and lifespan. Diabetes has a direct effect on lifespan. Smoking is unobserved. For normal weight people and without diabetes, the probability of long lifespan is 91%. For normal weight people and with diabetes, the probability of long lifespan is 71%. For obese people and without diabetes, the probability of long lifespan is 69%. For obese people and with diabetes, the probability of long lifespan is 43%. For normal weight people and nonsmokers, the probability of having diabetes is 28%. For normal weight people and smokers, the probability of having diabetes is 20%. For obese people and nonsmokers, the probability of having diabetes is 57%. For obese people and smokers, the probability of having diabetes is 70%. The overall probability of smoker is 8%. If we disregard the mediation effect through diabetes, would obesity negatively affect lifespan?\\ \tt Start your answer with 'Yes' or 'No', followed by additional reasoning or evidence to support your explanation. 
	  } 
	   } 
	   } 
	 & \\ 
 

    \theutterance \stepcounter{utterance}  

    & \multicolumn{4}{p{0.6\linewidth}}{\cellcolor[rgb]{0.8,1,0.9}{%
	\makecell[{{p{\linewidth}}}]{% 
	  \tt {\tiny [A$\rangle$GM]}  
	 No, obesity would not negatively affect lifespan when disregarding the mediation effect through diabetes.\\ \tt \\ \tt To understand this, let's break down the given probabilities and analyze the effect of obesity on lifespan, ignoring the diabetes pathway. We are given the following probabilities:\\ \tt \\ \tt - P(long lifespan | normal weight, no diabetes) = 91%\\ \tt - P(long lifespan | normal weight, with diabetes) = 71%\\ \tt - P(long lifespan | obese, no diabetes) = 69%\\ \tt - P(long lifespan | obese, with diabetes) = 43%\\ \tt \\ \tt We are also given the probabilities of having diabetes for different groups:\\ \tt \\ \tt - P(diabetes | normal weight, nonsmoker) = 28%\\ \tt - P(diabetes | normal weight, smoker) = 20%\\ \tt - P(diabetes | obese, nonsmoker) = 57%\\ \tt - P(diabetes | obese, smoker) = 70%\\ \tt \\ \tt The overall probability of being a smoker is 8%.\\ \tt \\ \tt To assess the effect of obesity on lifespan, we need to compare the probabilities of long lifespan between normal weight and obese individuals, while accounting for the probability of diabetes. However, since we are disregarding the diabetes pathway, we can directly compare the probabilities of long lifespan between normal weight and obese individuals.\\ \tt \\ \tt Let's calculate the probability of long lifespan for obese individuals, ignoring the diabetes pathway. We can do this by considering the probability of long lifespan for obese individuals without diabetes (69%) and the probability of long lifespan for obese individuals with diabetes (43%). However 
	  } 
	   } 
	   } 
	 & & \\ 
 

    \theutterance \stepcounter{utterance}  

    & & & \multicolumn{2}{p{0.3\linewidth}}{\cellcolor[rgb]{0.95,0.95,0.95}{%
	\makecell[{{p{\linewidth}}}]{% 
	  \tt {\tiny [GM$|$GM]}  
	 No 
	  } 
	   } 
	   } 
	 & & \\ 
 

    \theutterance \stepcounter{utterance}  

    & & & \multicolumn{2}{p{0.3\linewidth}}{\cellcolor[rgb]{0.95,0.95,0.95}{%
	\makecell[{{p{\linewidth}}}]{% 
	  \tt {\tiny [GM$|$GM]}  
	 yes 
	  } 
	   } 
	   } 
	 & & \\ 
 

    \theutterance \stepcounter{utterance}  

    & & & \multicolumn{2}{p{0.3\linewidth}}{\cellcolor[rgb]{0.95,0.95,0.95}{%
	\makecell[{{p{\linewidth}}}]{% 
	  \tt {\tiny [GM$|$GM]}  
	 game_result = LOSE 
	  } 
	   } 
	   } 
	 & & \\ 
 

\end{supertabular}
}

\end{document}
