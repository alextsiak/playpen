
\documentclass{article}
\usepackage{colortbl}
\usepackage{makecell}
\usepackage{multirow}
\usepackage{supertabular}

\begin{document}

\newcounter{utterance}

\twocolumn

{ \footnotesize  \setcounter{utterance}{1}
\setlength{\tabcolsep}{0pt}
\begin{supertabular}{c@{$\;$}|p{.15\linewidth}@{}p{.15\linewidth}p{.15\linewidth}p{.15\linewidth}p{.15\linewidth}p{.15\linewidth}}

    \# & $\;$A & \multicolumn{4}{c}{Game Master} & $\;\:$B\\
    \hline 

    \theutterance \stepcounter{utterance}  

    & & \multicolumn{4}{p{0.6\linewidth}}{\cellcolor[rgb]{0.9,0.9,0.9}{%
	\makecell[{{p{\linewidth}}}]{% 
	  \tt {\tiny [A$\langle$GM]}  
	 You are an expert in causal inference. The following question is not a typical commonsense query, but rather a meticulously designed question created by a professor specializing in causal inference, intended to assess the students' mastery of the course content.\\ \tt Imagine a self-contained, hypothetical world with only the following conditions, and without any unmentioned factors or causal relationships: Ability to speak english has a direct effect on diabetes and lifespan. Smoking has a direct effect on diabetes and lifespan. Diabetes has a direct effect on lifespan. Smoking is unobserved. For people who do not speak english, the probability of long lifespan is 77%. For people who speak english, the probability of long lifespan is 45%. Will speaking english decrease the chance of long lifespan?\\ \tt Start your answer with 'Yes' or 'No', followed by additional reasoning or evidence to support your explanation. 
	  } 
	   } 
	   } 
	 & \\ 
 

    \theutterance \stepcounter{utterance}  

    & \multicolumn{4}{p{0.6\linewidth}}{\cellcolor[rgb]{0.8,1,0.9}{%
	\makecell[{{p{\linewidth}}}]{% 
	  \tt {\tiny [A$\rangle$GM]}  
	 No\\ \tt \\ \tt To address this question, we need to apply the principles of causal inference, specifically the concept of confounding variables and the potential outcomes framework. \\ \tt \\ \tt In this hypothetical world, we have three variables: speaking English (E), smoking (S), diabetes (D), and lifespan (L). The relationships between these variables are as follows:\\ \tt \\ \tt 1. E → D (speaking English affects diabetes)\\ \tt 2. S → D (smoking affects diabetes)\\ \tt 3. D → L (diabetes affects lifespan)\\ \tt \\ \tt We are given the following probabilities:\\ \tt \\ \tt - P(L|D=0) = 77% (probability of long lifespan for people who do not speak English, i.e., D=0)\\ \tt - P(L|D=1) = 45% (probability of long lifespan for people who speak English, i.e., D=1)\\ \tt \\ \tt The question asks whether speaking English (E=1) decreases the chance of long lifespan. To answer this, we need to consider the potential outcomes framework, which involves the following:\\ \tt \\ \tt - P(L|E=0, D=0) = 77% (probability of long lifespan for people who do not speak English and do not have diabetes)\\ \tt - P(L|E=0, D=1) =? (probability of long lifespan for people who do not speak English and have diabetes)\\ \tt - P(L|E=1, D=0) =? (probability of long lifespan for people who speak English and do not have diabetes 
	  } 
	   } 
	   } 
	 & & \\ 
 

    \theutterance \stepcounter{utterance}  

    & & & \multicolumn{2}{p{0.3\linewidth}}{\cellcolor[rgb]{0.95,0.95,0.95}{%
	\makecell[{{p{\linewidth}}}]{% 
	  \tt {\tiny [GM$|$GM]}  
	 No 
	  } 
	   } 
	   } 
	 & & \\ 
 

    \theutterance \stepcounter{utterance}  

    & & & \multicolumn{2}{p{0.3\linewidth}}{\cellcolor[rgb]{0.95,0.95,0.95}{%
	\makecell[{{p{\linewidth}}}]{% 
	  \tt {\tiny [GM$|$GM]}  
	 yes 
	  } 
	   } 
	   } 
	 & & \\ 
 

    \theutterance \stepcounter{utterance}  

    & & & \multicolumn{2}{p{0.3\linewidth}}{\cellcolor[rgb]{0.95,0.95,0.95}{%
	\makecell[{{p{\linewidth}}}]{% 
	  \tt {\tiny [GM$|$GM]}  
	 game_result = LOSE 
	  } 
	   } 
	   } 
	 & & \\ 
 

\end{supertabular}
}

\end{document}
