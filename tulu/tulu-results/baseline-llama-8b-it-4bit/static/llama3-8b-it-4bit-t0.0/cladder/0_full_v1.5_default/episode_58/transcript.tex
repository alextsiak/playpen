
\documentclass{article}
\usepackage{colortbl}
\usepackage{makecell}
\usepackage{multirow}
\usepackage{supertabular}

\begin{document}

\newcounter{utterance}

\twocolumn

{ \footnotesize  \setcounter{utterance}{1}
\setlength{\tabcolsep}{0pt}
\begin{supertabular}{c@{$\;$}|p{.15\linewidth}@{}p{.15\linewidth}p{.15\linewidth}p{.15\linewidth}p{.15\linewidth}p{.15\linewidth}}

    \# & $\;$A & \multicolumn{4}{c}{Game Master} & $\;\:$B\\
    \hline 

    \theutterance \stepcounter{utterance}  

    & & \multicolumn{4}{p{0.6\linewidth}}{\cellcolor[rgb]{0.9,0.9,0.9}{%
	\makecell[{{p{\linewidth}}}]{% 
	  \tt {\tiny [A$\langle$GM]}  
	 You are an expert in causal inference. The following question is not a typical commonsense query, but rather a meticulously designed question created by a professor specializing in causal inference, intended to assess the students' mastery of the course content.\\ \tt Imagine a self-contained, hypothetical world with only the following conditions, and without any unmentioned factors or causal relationships: Vaccination status has a direct effect on vaccination reaction and getting smallpox. Getting smallpox has a direct effect on black hair. Vaccination reaction has a direct effect on black hair. For unvaccinated individuals, the probability of black hair is 27%. For vaccinated individuals, the probability of black hair is 37%. Will vaccination decrease the chance of black hair?\\ \tt Start your answer with 'Yes' or 'No', followed by additional reasoning or evidence to support your explanation. 
	  } 
	   } 
	   } 
	 & \\ 
 

    \theutterance \stepcounter{utterance}  

    & \multicolumn{4}{p{0.6\linewidth}}{\cellcolor[rgb]{0.8,1,0.9}{%
	\makecell[{{p{\linewidth}}}]{% 
	  \tt {\tiny [A$\rangle$GM]}  
	 No\\ \tt \\ \tt To answer this question, we need to carefully consider the given causal relationships and their implications. \\ \tt \\ \tt First, let's break down the information provided:\\ \tt \\ \tt 1. Vaccination status affects vaccination reaction.\\ \tt 2. Vaccination reaction affects black hair.\\ \tt 3. Getting smallpox affects black hair.\\ \tt 4. Vaccination status affects the probability of getting smallpox (implied, but not directly stated).\\ \tt \\ \tt We are given the probabilities of black hair for unvaccinated and vaccinated individuals: 27% and 37%, respectively.\\ \tt \\ \tt To assess the effect of vaccination on the probability of black hair, we need to consider the potential pathways through which vaccination might influence black hair. There are two direct pathways:\\ \tt \\ \tt 1. Vaccination status → Vaccination reaction → Black hair\\ \tt 2. Vaccination status → Getting smallpox → Black hair\\ \tt \\ \tt However, the effect of vaccination on black hair is not a direct effect, but rather an indirect effect through the intermediate variables (vaccination reaction and getting smallpox).\\ \tt \\ \tt To determine the net effect of vaccination on black hair, we need to consider the following:\\ \tt \\ \tt - The probability of black hair for unvaccinated individuals is 27%.\\ \tt - The probability of black hair for vaccinated individuals is 37%.\\ \tt \\ \tt However, the question does not provide the probability of getting smallpox for unvaccinated and vaccinated individuals. Without this information, we cannot directly compare the probabilities of black hair for unvaccinated and vaccinated individuals.\\ \tt \\ \tt Moreover, the effect of 
	  } 
	   } 
	   } 
	 & & \\ 
 

    \theutterance \stepcounter{utterance}  

    & & & \multicolumn{2}{p{0.3\linewidth}}{\cellcolor[rgb]{0.95,0.95,0.95}{%
	\makecell[{{p{\linewidth}}}]{% 
	  \tt {\tiny [GM$|$GM]}  
	 No 
	  } 
	   } 
	   } 
	 & & \\ 
 

    \theutterance \stepcounter{utterance}  

    & & & \multicolumn{2}{p{0.3\linewidth}}{\cellcolor[rgb]{0.95,0.95,0.95}{%
	\makecell[{{p{\linewidth}}}]{% 
	  \tt {\tiny [GM$|$GM]}  
	 no 
	  } 
	   } 
	   } 
	 & & \\ 
 

    \theutterance \stepcounter{utterance}  

    & & & \multicolumn{2}{p{0.3\linewidth}}{\cellcolor[rgb]{0.95,0.95,0.95}{%
	\makecell[{{p{\linewidth}}}]{% 
	  \tt {\tiny [GM$|$GM]}  
	 game_result = WIN 
	  } 
	   } 
	   } 
	 & & \\ 
 

\end{supertabular}
}

\end{document}
