
\documentclass{article}
\usepackage{colortbl}
\usepackage{makecell}
\usepackage{multirow}
\usepackage{supertabular}

\begin{document}

\newcounter{utterance}

\twocolumn

{ \footnotesize  \setcounter{utterance}{1}
\setlength{\tabcolsep}{0pt}
\begin{supertabular}{c@{$\;$}|p{.15\linewidth}@{}p{.15\linewidth}p{.15\linewidth}p{.15\linewidth}p{.15\linewidth}p{.15\linewidth}}

    \# & $\;$A & \multicolumn{4}{c}{Game Master} & $\;\:$B\\
    \hline 

    \theutterance \stepcounter{utterance}  

    & & \multicolumn{4}{p{0.6\linewidth}}{\cellcolor[rgb]{0.9,0.9,0.9}{%
	\makecell[{{p{\linewidth}}}]{% 
	  \tt {\tiny [A$\langle$GM]}  
	 You are an expert in causal inference. The following question is not a typical commonsense query, but rather a meticulously designed question created by a professor specializing in causal inference, intended to assess the students' mastery of the course content.\\ \tt Imagine a self-contained, hypothetical world with only the following conditions, and without any unmentioned factors or causal relationships: Drinking coffee has a direct effect on hospitalization status. Broken bones has a direct effect on hospitalization status. Method 1: We look directly at how drinking coffee correlates with broken bones in general. Method 2: We look at this correlation case by case according to hospitalization status. To understand how drinking coffee affects broken bones, is it more correct to use the Method 1 than Method 2?\\ \tt Start your answer with 'Yes' or 'No', followed by additional reasoning or evidence to support your explanation. 
	  } 
	   } 
	   } 
	 & \\ 
 

    \theutterance \stepcounter{utterance}  

    & \multicolumn{4}{p{0.6\linewidth}}{\cellcolor[rgb]{0.8,1,0.9}{%
	\makecell[{{p{\linewidth}}}]{% 
	  \tt {\tiny [A$\rangle$GM]}  
	 No, it is not more correct to use Method 1 than Method 2 to understand how drinking coffee affects broken bones.\\ \tt \\ \tt In this hypothetical world, we are given that drinking coffee has a direct effect on hospitalization status and broken bones has a direct effect on hospitalization status. This implies that there is a potential confounding variable (hospitalization status) that affects both the exposure (drinking coffee) and the outcome (broken bones).\\ \tt \\ \tt Method 1, which looks directly at the correlation between drinking coffee and broken bones in general, is likely to produce a biased estimate of the effect of coffee on broken bones. This is because hospitalization status is a confounding variable that is correlated with both coffee consumption and the occurrence of broken bones. If people who drink coffee are more likely to be hospitalized due to other reasons (e.g., coffee has a direct effect on hospitalization status), and hospitalization status is also a risk factor for broken bones, then the observed correlation between coffee consumption and broken bones will be inflated.\\ \tt \\ \tt In contrast, Method 2, which looks at the correlation case by case according to hospitalization status, is a more refined approach that takes into account the potential confounding effect of hospitalization status. By stratifying the analysis by hospitalization status, we can estimate the effect of coffee on broken bones within each stratum, which can help to control for the confounding effect of hospitalization status.\\ \tt \\ \tt Therefore, Method 2 is a more correct approach than Method 1 to understand 
	  } 
	   } 
	   } 
	 & & \\ 
 

    \theutterance \stepcounter{utterance}  

    & & & \multicolumn{2}{p{0.3\linewidth}}{\cellcolor[rgb]{0.95,0.95,0.95}{%
	\makecell[{{p{\linewidth}}}]{% 
	  \tt {\tiny [GM$|$GM]}  
	 No 
	  } 
	   } 
	   } 
	 & & \\ 
 

    \theutterance \stepcounter{utterance}  

    & & & \multicolumn{2}{p{0.3\linewidth}}{\cellcolor[rgb]{0.95,0.95,0.95}{%
	\makecell[{{p{\linewidth}}}]{% 
	  \tt {\tiny [GM$|$GM]}  
	 yes 
	  } 
	   } 
	   } 
	 & & \\ 
 

    \theutterance \stepcounter{utterance}  

    & & & \multicolumn{2}{p{0.3\linewidth}}{\cellcolor[rgb]{0.95,0.95,0.95}{%
	\makecell[{{p{\linewidth}}}]{% 
	  \tt {\tiny [GM$|$GM]}  
	 game_result = LOSE 
	  } 
	   } 
	   } 
	 & & \\ 
 

\end{supertabular}
}

\end{document}
