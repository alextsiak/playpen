
\documentclass{article}
\usepackage{colortbl}
\usepackage{makecell}
\usepackage{multirow}
\usepackage{supertabular}

\begin{document}

\newcounter{utterance}

\twocolumn

{ \footnotesize  \setcounter{utterance}{1}
\setlength{\tabcolsep}{0pt}
\begin{supertabular}{c@{$\;$}|p{.15\linewidth}@{}p{.15\linewidth}p{.15\linewidth}p{.15\linewidth}p{.15\linewidth}p{.15\linewidth}}

    \# & $\;$A & \multicolumn{4}{c}{Game Master} & $\;\:$B\\
    \hline 

    \theutterance \stepcounter{utterance}  

    & & \multicolumn{4}{p{0.6\linewidth}}{\cellcolor[rgb]{0.9,0.9,0.9}{%
	\makecell[{{p{\linewidth}}}]{% 
	  \tt {\tiny [A$\langle$GM]}  
	 You are an expert in causal inference. The following question is not a typical commonsense query, but rather a meticulously designed question created by a professor specializing in causal inference, intended to assess the students' mastery of the course content.\\ \tt Imagine a self-contained, hypothetical world with only the following conditions, and without any unmentioned factors or causal relationships: Jyka has a direct effect on yupt. Yupt has a direct effect on kwox. The overall probability of jyka is 25%. The probability of not jyka and kwox is 71%. The probability of jyka and kwox is 24%. Is the chance of kwox smaller when observing jyka?\\ \tt Start your answer with 'Yes' or 'No', followed by additional reasoning or evidence to support your explanation. 
	  } 
	   } 
	   } 
	 & \\ 
 

    \theutterance \stepcounter{utterance}  

    & \multicolumn{4}{p{0.6\linewidth}}{\cellcolor[rgb]{0.8,1,0.9}{%
	\makecell[{{p{\linewidth}}}]{% 
	  \tt {\tiny [A$\rangle$GM]}  
	 No, the chance of kwox is not smaller when observing jyka.\\ \tt \\ \tt To analyze this problem, we can use the concept of conditional probability and the causal relationships between the variables. Let's denote the probability of kwox given jyka as P(kwox|jyka) and the probability of kwox given not jyka as P(kwox|not jyka).\\ \tt \\ \tt We are given the following probabilities:\\ \tt \\ \tt 1. P(jyka) = 0.25\\ \tt 2. P(not jyka and kwox) = 0.71\\ \tt 3. P(jyka and kwox) = 0.24\\ \tt \\ \tt We can use the law of total probability to find P(kwox):\\ \tt \\ \tt P(kwox) = P(kwox|jyka)P(jyka) + P(kwox|not jyka)P(not jyka)\\ \tt \\ \tt We know that P(kwox and not jyka) = 0.71, and P(jyka and kwox) = 0.24. We can use these values to find P(kwox|not jyka):\\ \tt \\ \tt P(kwox|not jyka) = P(kwox and not jyka) / P(not jyka)\\ \tt = 0.71 / (1 - P(jyka))\\ \tt = 0.71 / (1 - 0.25)\\ \tt = 
	  } 
	   } 
	   } 
	 & & \\ 
 

    \theutterance \stepcounter{utterance}  

    & & & \multicolumn{2}{p{0.3\linewidth}}{\cellcolor[rgb]{0.95,0.95,0.95}{%
	\makecell[{{p{\linewidth}}}]{% 
	  \tt {\tiny [GM$|$GM]}  
	 No 
	  } 
	   } 
	   } 
	 & & \\ 
 

    \theutterance \stepcounter{utterance}  

    & & & \multicolumn{2}{p{0.3\linewidth}}{\cellcolor[rgb]{0.95,0.95,0.95}{%
	\makecell[{{p{\linewidth}}}]{% 
	  \tt {\tiny [GM$|$GM]}  
	 no 
	  } 
	   } 
	   } 
	 & & \\ 
 

    \theutterance \stepcounter{utterance}  

    & & & \multicolumn{2}{p{0.3\linewidth}}{\cellcolor[rgb]{0.95,0.95,0.95}{%
	\makecell[{{p{\linewidth}}}]{% 
	  \tt {\tiny [GM$|$GM]}  
	 game_result = WIN 
	  } 
	   } 
	   } 
	 & & \\ 
 

\end{supertabular}
}

\end{document}
