
\documentclass{article}
\usepackage{colortbl}
\usepackage{makecell}
\usepackage{multirow}
\usepackage{supertabular}

\begin{document}

\newcounter{utterance}

\twocolumn

{ \footnotesize  \setcounter{utterance}{1}
\setlength{\tabcolsep}{0pt}
\begin{supertabular}{c@{$\;$}|p{.15\linewidth}@{}p{.15\linewidth}p{.15\linewidth}p{.15\linewidth}p{.15\linewidth}p{.15\linewidth}}

    \# & $\;$A & \multicolumn{4}{c}{Game Master} & $\;\:$B\\
    \hline 

    \theutterance \stepcounter{utterance}  

    & & \multicolumn{4}{p{0.6\linewidth}}{\cellcolor[rgb]{0.9,0.9,0.9}{%
	\makecell[{{p{\linewidth}}}]{% 
	  \tt {\tiny [A$\langle$GM]}  
	 You are an expert in causal inference. The following question is not a typical commonsense query, but rather a meticulously designed question created by a professor specializing in causal inference, intended to assess the students' mastery of the course content.\\ \tt Imagine a self-contained, hypothetical world with only the following conditions, and without any unmentioned factors or causal relationships: Tanning salon treatment has a direct effect on foot size. Going to the beach has a direct effect on foot size. The overall probability of tanning salon treatment is 10%. For people not using tanning salon treatments, the probability of large feet is 13%. For people who went to tanning salons, the probability of large feet is 70%. Is large feet more likely than small feet overall?\\ \tt Start your answer with 'Yes' or 'No', followed by additional reasoning or evidence to support your explanation. 
	  } 
	   } 
	   } 
	 & \\ 
 

    \theutterance \stepcounter{utterance}  

    & \multicolumn{4}{p{0.6\linewidth}}{\cellcolor[rgb]{0.8,1,0.9}{%
	\makecell[{{p{\linewidth}}}]{% 
	  \tt {\tiny [A$\rangle$GM]}  
	 No, large feet are not more likely than small feet overall.\\ \tt \\ \tt To understand this, let's break down the given information:\\ \tt \\ \tt 1. Tanning salon treatment has a direct effect on foot size.\\ \tt 2. Going to the beach has a direct effect on foot size.\\ \tt 3. The overall probability of tanning salon treatment is 10%.\\ \tt 4. For people not using tanning salon treatments, the probability of large feet is 13%.\\ \tt 5. For people who went to tanning salons, the probability of large feet is 70%.\\ \tt \\ \tt We can use the potential outcomes framework to analyze this problem. Let's define two potential outcomes: large feet (L) and small feet (S). We can also define two treatment groups: tanning salon treatment (T) and not using tanning salon treatment (NT).\\ \tt \\ \tt Given the information, we can construct a 2x2 table with the following probabilities:\\ \tt \\ \tt |  | T | NT |\\ \tt | --- | --- | --- |\\ \tt | L | 0.7 | 0.13 |\\ \tt | S | 0.3 | 0.87 |\\ \tt \\ \tt The overall probability of large feet (P(L)) can be calculated as follows:\\ \tt \\ \tt P(L) = P(L|T) \* P(T) + P(L|NT) \* P(NT)\\ \tt = 0.7 \* 0.1 + 0.13 \* 0.9\\ \tt = 0.07 + 0.117\\ \tt = 0 
	  } 
	   } 
	   } 
	 & & \\ 
 

    \theutterance \stepcounter{utterance}  

    & & & \multicolumn{2}{p{0.3\linewidth}}{\cellcolor[rgb]{0.95,0.95,0.95}{%
	\makecell[{{p{\linewidth}}}]{% 
	  \tt {\tiny [GM$|$GM]}  
	 No 
	  } 
	   } 
	   } 
	 & & \\ 
 

    \theutterance \stepcounter{utterance}  

    & & & \multicolumn{2}{p{0.3\linewidth}}{\cellcolor[rgb]{0.95,0.95,0.95}{%
	\makecell[{{p{\linewidth}}}]{% 
	  \tt {\tiny [GM$|$GM]}  
	 no 
	  } 
	   } 
	   } 
	 & & \\ 
 

    \theutterance \stepcounter{utterance}  

    & & & \multicolumn{2}{p{0.3\linewidth}}{\cellcolor[rgb]{0.95,0.95,0.95}{%
	\makecell[{{p{\linewidth}}}]{% 
	  \tt {\tiny [GM$|$GM]}  
	 game_result = WIN 
	  } 
	   } 
	   } 
	 & & \\ 
 

\end{supertabular}
}

\end{document}
