
\documentclass{article}
\usepackage{colortbl}
\usepackage{makecell}
\usepackage{multirow}
\usepackage{supertabular}

\begin{document}

\newcounter{utterance}

\twocolumn

{ \footnotesize  \setcounter{utterance}{1}
\setlength{\tabcolsep}{0pt}
\begin{supertabular}{c@{$\;$}|p{.15\linewidth}@{}p{.15\linewidth}p{.15\linewidth}p{.15\linewidth}p{.15\linewidth}p{.15\linewidth}}

    \# & $\;$A & \multicolumn{4}{c}{Game Master} & $\;\:$B\\
    \hline 

    \theutterance \stepcounter{utterance}  

    & & \multicolumn{4}{p{0.6\linewidth}}{\cellcolor[rgb]{0.9,0.9,0.9}{%
	\makecell[{{p{\linewidth}}}]{% 
	  \tt {\tiny [A$\langle$GM]}  
	 You are an expert in causal inference. The following question is not a typical commonsense query, but rather a meticulously designed question created by a professor specializing in causal inference, intended to assess the students' mastery of the course content.\\ \tt Imagine a self-contained, hypothetical world with only the following conditions, and without any unmentioned factors or causal relationships: Having a brother has a direct effect on wife and alarm clock. Wife has a direct effect on alarm clock. For people who do not have a brother and wives that don't set the alarm, the probability of ringing alarm is 88%. For people who do not have a brother and wives that set the alarm, the probability of ringing alarm is 45%. For people who have a brother and wives that don't set the alarm, the probability of ringing alarm is 58%. For people who have a brother and wives that set the alarm, the probability of ringing alarm is 13%. For people who do not have a brother, the probability of alarm set by wife is 9%. For people who have a brother, the probability of alarm set by wife is 64%. Does having a brother negatively affect alarm clock through wife?\\ \tt Start your answer with 'Yes' or 'No', followed by additional reasoning or evidence to support your explanation. 
	  } 
	   } 
	   } 
	 & \\ 
 

    \theutterance \stepcounter{utterance}  

    & \multicolumn{4}{p{0.6\linewidth}}{\cellcolor[rgb]{0.8,1,0.9}{%
	\makecell[{{p{\linewidth}}}]{% 
	  \tt {\tiny [A$\rangle$GM]}  
	 No, having a brother does not negatively affect alarm clock through wife.\\ \tt \\ \tt To understand this, let's break down the given information and apply the principles of causal inference. We are interested in the effect of having a brother on the alarm clock, mediated by the wife's behavior.\\ \tt \\ \tt First, we need to establish the causal relationships between the variables. The given information implies the following causal relationships:\\ \tt \\ \tt - Having a brother (B) affects the wife's behavior (W) in two ways: it increases the probability of the wife setting the alarm (P(W|B) > P(W|~B)) and it increases the probability of the wife not setting the alarm (P(~W|B) > P(~W|~B)).\\ \tt - The wife's behavior (W) affects the alarm clock (A) in two ways: it increases the probability of the alarm ringing when the wife sets the alarm (P(A|W) > P(A|~W)) and it decreases the probability of the alarm ringing when the wife does not set the alarm (P(A|~W) < P(A|W)).\\ \tt \\ \tt Now, let's examine the effect of having a brother on the alarm clock through the wife's behavior. We can use the potential outcomes framework to analyze this effect. The potential outcomes are the probabilities of the alarm ringing (A) and not ringing (~A) under different combinations of the variables.\\ \tt \\ \tt Given the information, we can calculate the following probabilities:\\ \tt \\ \tt - P(A|~B 
	  } 
	   } 
	   } 
	 & & \\ 
 

    \theutterance \stepcounter{utterance}  

    & & & \multicolumn{2}{p{0.3\linewidth}}{\cellcolor[rgb]{0.95,0.95,0.95}{%
	\makecell[{{p{\linewidth}}}]{% 
	  \tt {\tiny [GM$|$GM]}  
	 No 
	  } 
	   } 
	   } 
	 & & \\ 
 

    \theutterance \stepcounter{utterance}  

    & & & \multicolumn{2}{p{0.3\linewidth}}{\cellcolor[rgb]{0.95,0.95,0.95}{%
	\makecell[{{p{\linewidth}}}]{% 
	  \tt {\tiny [GM$|$GM]}  
	 yes 
	  } 
	   } 
	   } 
	 & & \\ 
 

    \theutterance \stepcounter{utterance}  

    & & & \multicolumn{2}{p{0.3\linewidth}}{\cellcolor[rgb]{0.95,0.95,0.95}{%
	\makecell[{{p{\linewidth}}}]{% 
	  \tt {\tiny [GM$|$GM]}  
	 game_result = LOSE 
	  } 
	   } 
	   } 
	 & & \\ 
 

\end{supertabular}
}

\end{document}
