
\documentclass{article}
\usepackage{colortbl}
\usepackage{makecell}
\usepackage{multirow}
\usepackage{supertabular}

\begin{document}

\newcounter{utterance}

\twocolumn

{ \footnotesize  \setcounter{utterance}{1}
\setlength{\tabcolsep}{0pt}
\begin{supertabular}{c@{$\;$}|p{.15\linewidth}@{}p{.15\linewidth}p{.15\linewidth}p{.15\linewidth}p{.15\linewidth}p{.15\linewidth}}

    \# & $\;$A & \multicolumn{4}{c}{Game Master} & $\;\:$B\\
    \hline 

    \theutterance \stepcounter{utterance}  

    & & \multicolumn{4}{p{0.6\linewidth}}{\cellcolor[rgb]{0.9,0.9,0.9}{%
	\makecell[{{p{\linewidth}}}]{% 
	  \tt {\tiny [A$\langle$GM]}  
	 You are an expert in causal inference. The following question is not a typical commonsense query, but rather a meticulously designed question created by a professor specializing in causal inference, intended to assess the students' mastery of the course content.\\ \tt Imagine a self-contained, hypothetical world with only the following conditions, and without any unmentioned factors or causal relationships: Tanning salon treatment has a direct effect on black hair. Going to the beach has a direct effect on black hair. The overall probability of tanning salon treatment is 99%. For people not using tanning salon treatments, the probability of black hair is 66%. For people who went to tanning salons, the probability of black hair is 24%. Is black hair less likely than blond hair overall?\\ \tt Start your answer with 'Yes' or 'No', followed by additional reasoning or evidence to support your explanation. 
	  } 
	   } 
	   } 
	 & \\ 
 

    \theutterance \stepcounter{utterance}  

    & \multicolumn{4}{p{0.6\linewidth}}{\cellcolor[rgb]{0.8,1,0.9}{%
	\makecell[{{p{\linewidth}}}]{% 
	  \tt {\tiny [A$\rangle$GM]}  
	 No, black hair is not less likely than blond hair overall.\\ \tt \\ \tt To understand this, let's break down the given information:\\ \tt \\ \tt 1. Tanning salon treatment has a direct effect on black hair.\\ \tt 2. Going to the beach has a direct effect on black hair.\\ \tt 3. The overall probability of tanning salon treatment is 99%.\\ \tt 4. For people not using tanning salon treatments, the probability of black hair is 66%.\\ \tt 5. For people who went to tanning salons, the probability of black hair is 24%.\\ \tt \\ \tt We can infer that the primary cause of black hair in this world is not the tanning salon treatment itself, but rather the fact that people who go to tanning salons are also more likely to go to the beach, which directly affects the probability of black hair. This is because the probability of black hair is 66% for people not using tanning salon treatments, but only 24% for people who use tanning salon treatments. This suggests that the tanning salon treatment is a proxy for the beach-going behavior, which is the actual cause of black hair.\\ \tt \\ \tt To calculate the overall probability of black hair, we need to consider both the probability of tanning salon treatment and the probability of black hair for people who do not use tanning salon treatments. Since the probability of tanning salon treatment is 99%, the probability of not using tanning salon treatment is 1%.\\ \tt \\ \tt Let's assume the probability of black hair for people who do not use 
	  } 
	   } 
	   } 
	 & & \\ 
 

    \theutterance \stepcounter{utterance}  

    & & & \multicolumn{2}{p{0.3\linewidth}}{\cellcolor[rgb]{0.95,0.95,0.95}{%
	\makecell[{{p{\linewidth}}}]{% 
	  \tt {\tiny [GM$|$GM]}  
	 No 
	  } 
	   } 
	   } 
	 & & \\ 
 

    \theutterance \stepcounter{utterance}  

    & & & \multicolumn{2}{p{0.3\linewidth}}{\cellcolor[rgb]{0.95,0.95,0.95}{%
	\makecell[{{p{\linewidth}}}]{% 
	  \tt {\tiny [GM$|$GM]}  
	 yes 
	  } 
	   } 
	   } 
	 & & \\ 
 

    \theutterance \stepcounter{utterance}  

    & & & \multicolumn{2}{p{0.3\linewidth}}{\cellcolor[rgb]{0.95,0.95,0.95}{%
	\makecell[{{p{\linewidth}}}]{% 
	  \tt {\tiny [GM$|$GM]}  
	 game_result = LOSE 
	  } 
	   } 
	   } 
	 & & \\ 
 

\end{supertabular}
}

\end{document}
