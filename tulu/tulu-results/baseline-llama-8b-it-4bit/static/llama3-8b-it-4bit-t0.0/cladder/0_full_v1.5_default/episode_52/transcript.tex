
\documentclass{article}
\usepackage{colortbl}
\usepackage{makecell}
\usepackage{multirow}
\usepackage{supertabular}

\begin{document}

\newcounter{utterance}

\twocolumn

{ \footnotesize  \setcounter{utterance}{1}
\setlength{\tabcolsep}{0pt}
\begin{supertabular}{c@{$\;$}|p{.15\linewidth}@{}p{.15\linewidth}p{.15\linewidth}p{.15\linewidth}p{.15\linewidth}p{.15\linewidth}}

    \# & $\;$A & \multicolumn{4}{c}{Game Master} & $\;\:$B\\
    \hline 

    \theutterance \stepcounter{utterance}  

    & & \multicolumn{4}{p{0.6\linewidth}}{\cellcolor[rgb]{0.9,0.9,0.9}{%
	\makecell[{{p{\linewidth}}}]{% 
	  \tt {\tiny [A$\langle$GM]}  
	 You are an expert in causal inference. The following question is not a typical commonsense query, but rather a meticulously designed question created by a professor specializing in causal inference, intended to assess the students' mastery of the course content.\\ \tt Imagine a self-contained, hypothetical world with only the following conditions, and without any unmentioned factors or causal relationships: CEO has a direct effect on director and manager. Manager has a direct effect on foot size. Director has a direct effect on foot size. For managers who don't sign termination letters and directors who don't sign termination letters, the probability of large feet is 81%. For managers who don't sign termination letters and directors who sign termination letters, the probability of large feet is 46%. For managers who sign termination letters and directors who don't sign termination letters, the probability of large feet is 49%. For managers who sign termination letters and directors who sign termination letters, the probability of large feet is 10%. For managers who don't sign termination letters, the probability of director signing the termination letter is 15%. For managers who sign termination letters, the probability of director signing the termination letter is 39%. For managers who sign termination letters, would it be less likely to see large feet if the manager had signed the termination letter?\\ \tt Start your answer with 'Yes' or 'No', followed by additional reasoning or evidence to support your explanation. 
	  } 
	   } 
	   } 
	 & \\ 
 

    \theutterance \stepcounter{utterance}  

    & \multicolumn{4}{p{0.6\linewidth}}{\cellcolor[rgb]{0.8,1,0.9}{%
	\makecell[{{p{\linewidth}}}]{% 
	  \tt {\tiny [A$\rangle$GM]}  
	 Yes, it would be less likely to see large feet if the manager had signed the termination letter.\\ \tt \\ \tt To answer this question, we need to apply the principles of causal inference, specifically the concept of conditional probabilities and the potential outcomes framework. We are given the following conditional probabilities:\\ \tt \\ \tt - P(Large feet | Manager doesn't sign, Director doesn't sign) = 81%\\ \tt - P(Large feet | Manager doesn't sign, Director signs) = 46%\\ \tt - P(Large feet | Manager signs, Director doesn't sign) = 49%\\ \tt - P(Large feet | Manager signs, Director signs) = 10%\\ \tt \\ \tt We are also given the following probabilities:\\ \tt \\ \tt - P(Director signs | Manager doesn't sign) = 15%\\ \tt - P(Director signs | Manager signs) = 39%\\ \tt \\ \tt We want to compare the probability of large feet when the manager signs the termination letter versus when the manager doesn't sign the termination letter. To do this, we need to calculate the probability of large feet in each scenario.\\ \tt \\ \tt Let's define the following events:\\ \tt \\ \tt - A: Manager signs the termination letter\\ \tt - B: Director signs the termination letter\\ \tt - C: Large feet\\ \tt \\ \tt We want to calculate P(C | A) and P(C | ~A), where ~A denotes the complement of A (i.e., the manager doesn't sign the termination letter).\\ \tt \\ \tt Using the law of total probability, we can write:\\ \tt \\ \tt P(C | A) = P(C | A, B)P 
	  } 
	   } 
	   } 
	 & & \\ 
 

    \theutterance \stepcounter{utterance}  

    & & & \multicolumn{2}{p{0.3\linewidth}}{\cellcolor[rgb]{0.95,0.95,0.95}{%
	\makecell[{{p{\linewidth}}}]{% 
	  \tt {\tiny [GM$|$GM]}  
	 Yes 
	  } 
	   } 
	   } 
	 & & \\ 
 

    \theutterance \stepcounter{utterance}  

    & & & \multicolumn{2}{p{0.3\linewidth}}{\cellcolor[rgb]{0.95,0.95,0.95}{%
	\makecell[{{p{\linewidth}}}]{% 
	  \tt {\tiny [GM$|$GM]}  
	 no 
	  } 
	   } 
	   } 
	 & & \\ 
 

    \theutterance \stepcounter{utterance}  

    & & & \multicolumn{2}{p{0.3\linewidth}}{\cellcolor[rgb]{0.95,0.95,0.95}{%
	\makecell[{{p{\linewidth}}}]{% 
	  \tt {\tiny [GM$|$GM]}  
	 game_result = LOSE 
	  } 
	   } 
	   } 
	 & & \\ 
 

\end{supertabular}
}

\end{document}
