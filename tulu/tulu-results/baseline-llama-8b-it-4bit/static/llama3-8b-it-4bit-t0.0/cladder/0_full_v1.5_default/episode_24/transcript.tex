
\documentclass{article}
\usepackage{colortbl}
\usepackage{makecell}
\usepackage{multirow}
\usepackage{supertabular}

\begin{document}

\newcounter{utterance}

\twocolumn

{ \footnotesize  \setcounter{utterance}{1}
\setlength{\tabcolsep}{0pt}
\begin{supertabular}{c@{$\;$}|p{.15\linewidth}@{}p{.15\linewidth}p{.15\linewidth}p{.15\linewidth}p{.15\linewidth}p{.15\linewidth}}

    \# & $\;$A & \multicolumn{4}{c}{Game Master} & $\;\:$B\\
    \hline 

    \theutterance \stepcounter{utterance}  

    & & \multicolumn{4}{p{0.6\linewidth}}{\cellcolor[rgb]{0.9,0.9,0.9}{%
	\makecell[{{p{\linewidth}}}]{% 
	  \tt {\tiny [A$\langle$GM]}  
	 You are an expert in causal inference. The following question is not a typical commonsense query, but rather a meticulously designed question created by a professor specializing in causal inference, intended to assess the students' mastery of the course content.\\ \tt Imagine a self-contained, hypothetical world with only the following conditions, and without any unmentioned factors or causal relationships: Unobserved confounders has a direct effect on education level and salary. Proximity to a college has a direct effect on education level. Education level has a direct effect on salary. Unobserved confounders is unobserved. For people living far from a college, the probability of high salary is 39%. For people living close to a college, the probability of high salary is 49%. For people living far from a college, the probability of college degree or higher is 57%. For people living close to a college, the probability of college degree or higher is 30%. Will college degree or higher increase the chance of high salary?\\ \tt Start your answer with 'Yes' or 'No', followed by additional reasoning or evidence to support your explanation. 
	  } 
	   } 
	   } 
	 & \\ 
 

    \theutterance \stepcounter{utterance}  

    & \multicolumn{4}{p{0.6\linewidth}}{\cellcolor[rgb]{0.8,1,0.9}{%
	\makecell[{{p{\linewidth}}}]{% 
	  \tt {\tiny [A$\rangle$GM]}  
	 No\\ \tt \\ \tt To answer this question, we need to use the potential outcomes framework and the concept of causal effects. Let's denote the potential outcomes as follows:\\ \tt \\ \tt - Y: high salary (1) or low salary (0)\\ \tt - X: college degree or higher (1) or not (0)\\ \tt - C: living close to a college (1) or far from a college (0)\\ \tt - U: unobserved confounders\\ \tt \\ \tt We are given the following conditional probabilities:\\ \tt \\ \tt - P(Y=1|C=0) = 0.39 (probability of high salary for people living far from a college)\\ \tt - P(Y=1|C=1) = 0.49 (probability of high salary for people living close to a college)\\ \tt - P(X=1|C=0) = 0.57 (probability of college degree or higher for people living far from a college)\\ \tt - P(X=1|C=1) = 0.30 (probability of college degree or higher for people living close to a college)\\ \tt \\ \tt We want to estimate the causal effect of X on Y, which is denoted as P(Y=1|X=1, C=0) - P(Y=1|X=0, C=0). However, since U is an unobserved confounder, we cannot directly estimate this effect using the given probabilities.\\ \tt \\ \tt To overcome this issue, we can use the concept of conditional probabilities and the law of total probability. 
	  } 
	   } 
	   } 
	 & & \\ 
 

    \theutterance \stepcounter{utterance}  

    & & & \multicolumn{2}{p{0.3\linewidth}}{\cellcolor[rgb]{0.95,0.95,0.95}{%
	\makecell[{{p{\linewidth}}}]{% 
	  \tt {\tiny [GM$|$GM]}  
	 No 
	  } 
	   } 
	   } 
	 & & \\ 
 

    \theutterance \stepcounter{utterance}  

    & & & \multicolumn{2}{p{0.3\linewidth}}{\cellcolor[rgb]{0.95,0.95,0.95}{%
	\makecell[{{p{\linewidth}}}]{% 
	  \tt {\tiny [GM$|$GM]}  
	 no 
	  } 
	   } 
	   } 
	 & & \\ 
 

    \theutterance \stepcounter{utterance}  

    & & & \multicolumn{2}{p{0.3\linewidth}}{\cellcolor[rgb]{0.95,0.95,0.95}{%
	\makecell[{{p{\linewidth}}}]{% 
	  \tt {\tiny [GM$|$GM]}  
	 game_result = WIN 
	  } 
	   } 
	   } 
	 & & \\ 
 

\end{supertabular}
}

\end{document}
