
\documentclass{article}
\usepackage{colortbl}
\usepackage{makecell}
\usepackage{multirow}
\usepackage{supertabular}

\begin{document}

\newcounter{utterance}

\twocolumn

{ \footnotesize  \setcounter{utterance}{1}
\setlength{\tabcolsep}{0pt}
\begin{supertabular}{c@{$\;$}|p{.15\linewidth}@{}p{.15\linewidth}p{.15\linewidth}p{.15\linewidth}p{.15\linewidth}p{.15\linewidth}}

    \# & $\;$A & \multicolumn{4}{c}{Game Master} & $\;\:$B\\
    \hline 

    \theutterance \stepcounter{utterance}  

    & & \multicolumn{4}{p{0.6\linewidth}}{\cellcolor[rgb]{0.9,0.9,0.9}{%
	\makecell[{{p{\linewidth}}}]{% 
	  \tt {\tiny [A$\langle$GM]}  
	 You are an expert in causal inference. The following question is not a typical commonsense query, but rather a meticulously designed question created by a professor specializing in causal inference, intended to assess the students' mastery of the course content.\\ \tt Imagine a self-contained, hypothetical world with only the following conditions, and without any unmentioned factors or causal relationships: Rixq has a direct effect on zuph and xevu. Swoy has a direct effect on zuph and xevu. Zuph has a direct effect on xevu. Swoy is unobserved. We know that rixq or swoy causes zuph. rixq and swoy and zuph causes xevu. We observed an individual is not swoy. Would an individual is xevu if rixq instead of not rixq?\\ \tt Start your answer with 'Yes' or 'No', followed by additional reasoning or evidence to support your explanation. 
	  } 
	   } 
	   } 
	 & \\ 
 

    \theutterance \stepcounter{utterance}  

    & \multicolumn{4}{p{0.6\linewidth}}{\cellcolor[rgb]{0.8,1,0.9}{%
	\makecell[{{p{\linewidth}}}]{% 
	  \tt {\tiny [A$\rangle$GM]}  
	 No\\ \tt \\ \tt In this hypothetical world, we have the following causal relationships:\\ \tt \\ \tt 1. Rixq → Zuph\\ \tt 2. Rixq → Xevu\\ \tt 3. Swoy → Zuph\\ \tt 4. Swoy → Xevu\\ \tt 5. Zuph → Xevu\\ \tt \\ \tt We are given that Rixq or Swoy causes Zuph, which can be represented as:\\ \tt \\ \tt Rixq → Zuph or Swoy → Zuph\\ \tt \\ \tt We are also given that Rixq, Swoy, and Zuph cause Xevu, which can be represented as:\\ \tt \\ \tt Rixq → Xevu\\ \tt Swoy → Xevu\\ \tt Zuph → Xevu\\ \tt \\ \tt Since we observe that the individual is not Swoy, we can infer that the individual is not Zuph (because Swoy → Zuph). \\ \tt \\ \tt Now, we need to determine whether the individual is Xevu if Rixq instead of not Rixq. To do this, we need to consider the following two scenarios:\\ \tt \\ \tt Scenario 1: Rixq is true\\ \tt Scenario 2: Rixq is false\\ \tt \\ \tt In Scenario 1, if Rixq is true, then Zuph is true (because Rixq → Zuph). Since Zuph is true, Xevu is true (because Zuph → Xevu). Therefore, in this scenario 
	  } 
	   } 
	   } 
	 & & \\ 
 

    \theutterance \stepcounter{utterance}  

    & & & \multicolumn{2}{p{0.3\linewidth}}{\cellcolor[rgb]{0.95,0.95,0.95}{%
	\makecell[{{p{\linewidth}}}]{% 
	  \tt {\tiny [GM$|$GM]}  
	 No 
	  } 
	   } 
	   } 
	 & & \\ 
 

    \theutterance \stepcounter{utterance}  

    & & & \multicolumn{2}{p{0.3\linewidth}}{\cellcolor[rgb]{0.95,0.95,0.95}{%
	\makecell[{{p{\linewidth}}}]{% 
	  \tt {\tiny [GM$|$GM]}  
	 no 
	  } 
	   } 
	   } 
	 & & \\ 
 

    \theutterance \stepcounter{utterance}  

    & & & \multicolumn{2}{p{0.3\linewidth}}{\cellcolor[rgb]{0.95,0.95,0.95}{%
	\makecell[{{p{\linewidth}}}]{% 
	  \tt {\tiny [GM$|$GM]}  
	 game_result = WIN 
	  } 
	   } 
	   } 
	 & & \\ 
 

\end{supertabular}
}

\end{document}
