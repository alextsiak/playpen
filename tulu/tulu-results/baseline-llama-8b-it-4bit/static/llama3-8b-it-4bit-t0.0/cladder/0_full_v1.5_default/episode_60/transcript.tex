
\documentclass{article}
\usepackage{colortbl}
\usepackage{makecell}
\usepackage{multirow}
\usepackage{supertabular}

\begin{document}

\newcounter{utterance}

\twocolumn

{ \footnotesize  \setcounter{utterance}{1}
\setlength{\tabcolsep}{0pt}
\begin{supertabular}{c@{$\;$}|p{.15\linewidth}@{}p{.15\linewidth}p{.15\linewidth}p{.15\linewidth}p{.15\linewidth}p{.15\linewidth}}

    \# & $\;$A & \multicolumn{4}{c}{Game Master} & $\;\:$B\\
    \hline 

    \theutterance \stepcounter{utterance}  

    & & \multicolumn{4}{p{0.6\linewidth}}{\cellcolor[rgb]{0.9,0.9,0.9}{%
	\makecell[{{p{\linewidth}}}]{% 
	  \tt {\tiny [A$\langle$GM]}  
	 You are an expert in causal inference. The following question is not a typical commonsense query, but rather a meticulously designed question created by a professor specializing in causal inference, intended to assess the students' mastery of the course content.\\ \tt Imagine a self-contained, hypothetical world with only the following conditions, and without any unmentioned factors or causal relationships: Unobserved confounders has a direct effect on education level and foot size. Proximity to a college has a direct effect on education level. Education level has a direct effect on foot size. Unobserved confounders is unobserved. The overall probability of college degree or higher is 47%. The probability of high school degree or lower and large feet is 32%. The probability of college degree or higher and large feet is 10%. Is the chance of large feet smaller when observing college degree or higher?\\ \tt Start your answer with 'Yes' or 'No', followed by additional reasoning or evidence to support your explanation. 
	  } 
	   } 
	   } 
	 & \\ 
 

    \theutterance \stepcounter{utterance}  

    & \multicolumn{4}{p{0.6\linewidth}}{\cellcolor[rgb]{0.8,1,0.9}{%
	\makecell[{{p{\linewidth}}}]{% 
	  \tt {\tiny [A$\rangle$GM]}  
	 No, the chance of large feet is not smaller when observing college degree or higher.\\ \tt \\ \tt To understand this, let's break down the given information:\\ \tt \\ \tt 1. Unobserved confounders have a direct effect on both education level and foot size.\\ \tt 2. Proximity to a college has a direct effect on education level.\\ \tt 3. Education level has a direct effect on foot size.\\ \tt 4. The overall probability of a college degree or higher is 47%.\\ \tt 5. The probability of a high school degree or lower and large feet is 32%.\\ \tt 6. The probability of a college degree or higher and large feet is 10%.\\ \tt \\ \tt We can use the potential outcomes framework to analyze this problem. Let's define the following variables:\\ \tt \\ \tt - U: Unobserved confounders\\ \tt - E: Education level\\ \tt - F: Foot size\\ \tt - C: College degree or higher (binary variable, 1 if college degree or higher, 0 otherwise)\\ \tt \\ \tt We can represent the relationships between these variables as follows:\\ \tt \\ \tt - P(F|U) = P(F|E) (the effect of U on F is mediated by E)\\ \tt - P(E|U) = P(E|C) (the effect of U on E is mediated by C)\\ \tt - P(F|E) = P(F|E=high)P(E=high|C) + P(F|E=low)P(E=low|C) (the effect of E on F is mediated by C)\\ \tt \\ \tt Using the given 
	  } 
	   } 
	   } 
	 & & \\ 
 

    \theutterance \stepcounter{utterance}  

    & & & \multicolumn{2}{p{0.3\linewidth}}{\cellcolor[rgb]{0.95,0.95,0.95}{%
	\makecell[{{p{\linewidth}}}]{% 
	  \tt {\tiny [GM$|$GM]}  
	 No 
	  } 
	   } 
	   } 
	 & & \\ 
 

    \theutterance \stepcounter{utterance}  

    & & & \multicolumn{2}{p{0.3\linewidth}}{\cellcolor[rgb]{0.95,0.95,0.95}{%
	\makecell[{{p{\linewidth}}}]{% 
	  \tt {\tiny [GM$|$GM]}  
	 yes 
	  } 
	   } 
	   } 
	 & & \\ 
 

    \theutterance \stepcounter{utterance}  

    & & & \multicolumn{2}{p{0.3\linewidth}}{\cellcolor[rgb]{0.95,0.95,0.95}{%
	\makecell[{{p{\linewidth}}}]{% 
	  \tt {\tiny [GM$|$GM]}  
	 game_result = LOSE 
	  } 
	   } 
	   } 
	 & & \\ 
 

\end{supertabular}
}

\end{document}
