
\documentclass{article}
\usepackage{colortbl}
\usepackage{makecell}
\usepackage{multirow}
\usepackage{supertabular}

\begin{document}

\newcounter{utterance}

\twocolumn

{ \footnotesize  \setcounter{utterance}{1}
\setlength{\tabcolsep}{0pt}
\begin{supertabular}{c@{$\;$}|p{.15\linewidth}@{}p{.15\linewidth}p{.15\linewidth}p{.15\linewidth}p{.15\linewidth}p{.15\linewidth}}

    \# & $\;$A & \multicolumn{4}{c}{Game Master} & $\;\:$B\\
    \hline 

    \theutterance \stepcounter{utterance}  

    & & \multicolumn{4}{p{0.6\linewidth}}{\cellcolor[rgb]{0.9,0.9,0.9}{%
	\makecell[{{p{\linewidth}}}]{% 
	  \tt {\tiny [A$\langle$GM]}  
	 The Jimenez family has 20 members. Each member has at least one pet. If there are 32 pets in total, what is the maximum number of members that can possibly have 3 pets? Solve this step by step, and wrap your entire response (reasoning process and the final answer) with double quotation marks. 
	  } 
	   } 
	   } 
	 & \\ 
 

    \theutterance \stepcounter{utterance}  

    & \multicolumn{4}{p{0.6\linewidth}}{\cellcolor[rgb]{0.8,1,0.9}{%
	\makecell[{{p{\linewidth}}}]{% 
	  \tt {\tiny [A$\rangle$GM]}  
	 "Let's break this down step by step.\\ \tt \\ \tt 1. The Jimenez family has 20 members, and each member has at least one pet. This means the total number of pets is 32.\\ \tt \\ \tt 2. To find the maximum number of members that can possibly have 3 pets, we need to consider the scenario where the remaining members have the minimum number of pets, which is 1.\\ \tt \\ \tt 3. If 17 members have 1 pet each, the total number of pets from these members would be 17.\\ \tt \\ \tt 4. This leaves us with 32 - 17 = 15 pets that can be distributed among the remaining members.\\ \tt \\ \tt 5. Since we want to maximize the number of members with 3 pets, we should minimize the number of members with 2 pets. Let's assume x members have 2 pets each. Then, the total number of pets from these members would be 2x.\\ \tt \\ \tt 6. We know that the total number of pets from members with 2 pets and 3 pets should be 15. So, we can set up the equation 2x + 3y = 15, where y is the number of members with 3 pets.\\ \tt \\ \tt 7. Since we want to maximize y, we should minimize x. The minimum value of x is 0, but that would mean all 15 pets are with members who have 3 pets, which is not possible since there are only 20 members in total. So, let's try x = 1.\\ \tt \\ \tt 8. If x = 1, then 2x = 2. This means 2 pets are with 1 member, and the remaining 13 pets are with members who have 3 pets. \\ \tt \\ \tt 9. Now, we can solve for y: 2 + 3y = 15. Subtracting 2 from both sides gives us 3y = 13. Dividing both sides by 3 gives us y = 13/3.\\ \tt \\ \tt 10. However, y must be an integer, so we need to adjust our previous assumption. Let's try x = 0.\\ \tt \\ \tt 11. If x = 0, then all 15 pets are with members who have 3 pets. This means 15 members have 3 pets each.\\ \tt \\ \tt 12. However, this would mean the total number of pets is 15 * 3 = 45, which is not possible since there are only 32 pets in total.\\ \tt \\ \tt 13. Let's try x = 1, but this time, let's adjust the number of members with 3 pets. If x = 1, then 2 pets are with 1 member, and the remaining 14 pets are with members who have 3 pets.\\ \tt \\ \tt 14. Now, we can solve for y: 2 + 3y = 14. Subtracting 2 from both sides gives us 3y = 12. Dividing both sides by 3 gives us y = 4.\\ \tt \\ \tt 15. Therefore, the maximum number of members that can possibly have 3 pets is '4'." 
	  } 
	   } 
	   } 
	 & & \\ 
 

    \theutterance \stepcounter{utterance}  

    & & & \multicolumn{2}{p{0.3\linewidth}}{\cellcolor[rgb]{0.95,0.95,0.95}{%
	\makecell[{{p{\linewidth}}}]{% 
	  \tt {\tiny [GM$|$GM]}  
	 {'startend:quotation': {}} 
	  } 
	   } 
	   } 
	 & & \\ 
 

    \theutterance \stepcounter{utterance}  

    & & & \multicolumn{2}{p{0.3\linewidth}}{\cellcolor[rgb]{0.95,0.95,0.95}{%
	\makecell[{{p{\linewidth}}}]{% 
	  \tt {\tiny [GM$|$GM]}  
	 game_result = WIN 
	  } 
	   } 
	   } 
	 & & \\ 
 

\end{supertabular}
}

\end{document}
