
\documentclass{article}
\usepackage{colortbl}
\usepackage{makecell}
\usepackage{multirow}
\usepackage{supertabular}

\begin{document}

\newcounter{utterance}

\twocolumn

{ \footnotesize  \setcounter{utterance}{1}
\setlength{\tabcolsep}{0pt}
\begin{supertabular}{c@{$\;$}|p{.15\linewidth}@{}p{.15\linewidth}p{.15\linewidth}p{.15\linewidth}p{.15\linewidth}p{.15\linewidth}}

    \# & $\;$A & \multicolumn{4}{c}{Game Master} & $\;\:$B\\
    \hline 

    \theutterance \stepcounter{utterance}  

    & & \multicolumn{4}{p{0.6\linewidth}}{\cellcolor[rgb]{0.9,0.9,0.9}{%
	\makecell[{{p{\linewidth}}}]{% 
	  \tt {\tiny [A$\langle$GM]}  
	 Translate the following sentence into German and then criticize it: Werner was a good friend of mine, but not very smart.\\ \tt Avoid the word "schlau" throughout your response. 
	  } 
	   } 
	   } 
	 & \\ 
 

    \theutterance \stepcounter{utterance}  

    & \multicolumn{4}{p{0.6\linewidth}}{\cellcolor[rgb]{0.8,1,0.9}{%
	\makecell[{{p{\linewidth}}}]{% 
	  \tt {\tiny [A$\rangle$GM]}  
	 The translation of the sentence into German is:\\ \tt \\ \tt Werner war ein guter Freund von mir, aber nicht sehr begabt.\\ \tt \\ \tt Now, let's criticize the sentence. \\ \tt \\ \tt The sentence is a bit informal and might not be suitable for formal writing or professional communication. The use of "begabt" to describe someone's intelligence is somewhat colloquial and might not be the most precise choice. A more formal or nuanced way to express the idea could be to say "aber nicht sehr aufgeklärt" or "aber nicht sehr verständig", which convey a similar meaning without using the word "begabt". \\ \tt \\ \tt However, in everyday conversation, the sentence is clear and effectively conveys the intended meaning. \\ \tt \\ \tt If you want to avoid the word "begabt" altogether, you could rephrase the sentence to something like:\\ \tt \\ \tt Werner war ein guter Freund von mir, aber nicht sehr präzise in seinen Urteilen.\\ \tt \\ \tt This version maintains the same meaning while using different vocabulary. \\ \tt \\ \tt Keep in mind that the criticism is aimed at the original sentence, and the translation provided is a direct equivalent. If you're looking for a more idiomatic or nuanced translation, I'd be happy to help with that as well. 
	  } 
	   } 
	   } 
	 & & \\ 
 

    \theutterance \stepcounter{utterance}  

    & & & \multicolumn{2}{p{0.3\linewidth}}{\cellcolor[rgb]{0.95,0.95,0.95}{%
	\makecell[{{p{\linewidth}}}]{% 
	  \tt {\tiny [GM$|$GM]}  
	 {'keywords:forbidden_words': {'forbidden_words': ['schlau']}} 
	  } 
	   } 
	   } 
	 & & \\ 
 

    \theutterance \stepcounter{utterance}  

    & & & \multicolumn{2}{p{0.3\linewidth}}{\cellcolor[rgb]{0.95,0.95,0.95}{%
	\makecell[{{p{\linewidth}}}]{% 
	  \tt {\tiny [GM$|$GM]}  
	 game_result = WIN 
	  } 
	   } 
	   } 
	 & & \\ 
 

\end{supertabular}
}

\end{document}
