
\documentclass{article}
\usepackage{colortbl}
\usepackage{makecell}
\usepackage{multirow}
\usepackage{supertabular}

\begin{document}

\newcounter{utterance}

\twocolumn

{ \footnotesize  \setcounter{utterance}{1}
\setlength{\tabcolsep}{0pt}
\begin{supertabular}{c@{$\;$}|p{.15\linewidth}@{}p{.15\linewidth}p{.15\linewidth}p{.15\linewidth}p{.15\linewidth}p{.15\linewidth}}

    \# & $\;$A & \multicolumn{4}{c}{Game Master} & $\;\:$B\\
    \hline 

    \theutterance \stepcounter{utterance}  

    & & \multicolumn{4}{p{0.6\linewidth}}{\cellcolor[rgb]{0.9,0.9,0.9}{%
	\makecell[{{p{\linewidth}}}]{% 
	  \tt {\tiny [A$\langle$GM]}  
	 The following are multiple choice questions (with answers) about economics. Answer only with "The answer is (X)" where X is the correct letter choice.\\ \tt \\ \tt Question:\\ \tt For a stationary autoregressive process, shocks will\\ \tt Options:\\ \tt A. Never occur\\ \tt B. Persist indefinitely\\ \tt C. Cause a permanent change in the trend\\ \tt D. Grow exponentially\\ \tt E. Only affect the next period\\ \tt F. Eventually die away\\ \tt G. Cause the process to become non-stationary\\ \tt H. Immediately disappear\\ \tt I. Lead to a cyclical pattern\\ \tt J. Be balanced out by subsequent shocks\\ \tt The answer is (F)\\ \tt \\ \tt Question:\\ \tt Consider the following AR(1) model with the disturbances having zero mean and unit variance\\ \tt yt = 0.2 + 0.4 yt-1 + ut\\ \tt The (unconditional) mean of y will be given by\\ \tt Options:\\ \tt A. 0.45\\ \tt B. 0.2\\ \tt C. 0.4\\ \tt D. 0.1\\ \tt E. 0.3\\ \tt F. 0.55\\ \tt G. 0.25\\ \tt H. 0.33\\ \tt I. 0.6\\ \tt J. 0.5\\ \tt The answer is (H)\\ \tt \\ \tt Question:\\ \tt Suppose that a test statistic has associated with it a p-value of 0.08. Which one of the following statements is true?\\ \tt (i) If the size of the test were exactly 8%, we would be indifferent between rejecting and not rejecting the null hypothesis\\ \tt (ii) The null would be rejected if a 10% size of test were used\\ \tt (iii) The null would not be rejected if a 1% size of test were used\\ \tt (iv) The null would be rejected if a 5% size of test were used.\\ \tt Options:\\ \tt A. (iii) and (iv) only\\ \tt B. (i) and (iv) only\\ \tt C. (i), (ii), and (iii) only\\ \tt D. (i) and (ii) only\\ \tt E. (i) and (iii) only\\ \tt F. (i), (ii), (iii), and (iv).\\ \tt G. (ii), (iii), and (iv) only\\ \tt H. (ii) and (iii) only\\ \tt I. (ii) and (iv) only\\ \tt J. (i), (iii), and (iv) only\\ \tt The answer is (C)\\ \tt \\ \tt Question:\\ \tt Which of the following policies best describes supply-side fiscal policy?\\ \tt Options:\\ \tt A. Higher taxes on household income\\ \tt B. Increased government spending\\ \tt C. Increased taxes on corporate profits\\ \tt D. Increased import tariffs\\ \tt E. Decreased interest rates\\ \tt F. Lower taxes on consumer goods\\ \tt G. Lower taxes on research and development of new technology\\ \tt H. Reduced government spending\\ \tt I. Higher taxes on capital gains\\ \tt J. An increase in the money supply\\ \tt The answer is (G)\\ \tt \\ \tt Question:\\ \tt The concentration ratio for a monopoly is\\ \tt Options:\\ \tt A. 50\\ \tt B. 5\\ \tt C. 10\\ \tt D. 90\\ \tt E. 15\\ \tt F. 100\\ \tt G. 0\\ \tt H. 25\\ \tt I. 75\\ \tt J. N/A\\ \tt The answer is (F)\\ \tt \\ \tt Question:\\ \tt Congress has embarked on another round of expansionary fiscal policy to boost employment and get reelected. As chair of the central bank how would you reduce the "crowding-out" effect and what macroeconomic problem might your policy exacerbate?\\ \tt Options:\\ \tt A. Buy government securities risking a recessionary gap\\ \tt B. Increase the discount rate risking deflation\\ \tt C. Sell less government securities risking deflation\\ \tt D. Increase the reserve ratio risking the devaluation of the dollar\\ \tt E. Lower the reserve ratio risking a devaluation of the dollar\\ \tt F. Decrease the reserve ratio risking inflation\\ \tt G. Sell government securities risking inflation\\ \tt H. Lower the discount rate risking inflation\\ \tt I. Buy more government securities risking hyperinflation\\ \tt J. Increase government spending risking a budget deficit\\ \tt  
	  } 
	   } 
	   } 
	 & \\ 
 

    \theutterance \stepcounter{utterance}  

    & \multicolumn{4}{p{0.6\linewidth}}{\cellcolor[rgb]{0.8,1,0.9}{%
	\makecell[{{p{\linewidth}}}]{% 
	  \tt {\tiny [A$\rangle$GM]}  
	 Here are the answers:\\ \tt \\ \tt  Question:\\ \tt For a stationary autoregressive process, shocks will\\ \tt The answer is (F)\\ \tt \\ \tt  Question:\\ \tt Consider the following AR(1) model with the disturbances having zero mean and unit variance\\ \tt The answer is (H) 
	  } 
	   } 
	   } 
	 & & \\ 
 

    \theutterance \stepcounter{utterance}  

    & & & \multicolumn{2}{p{0.3\linewidth}}{\cellcolor[rgb]{0.95,0.95,0.95}{%
	\makecell[{{p{\linewidth}}}]{% 
	  \tt {\tiny [GM$|$GM]}  
	 ParseError: Response contains more than a one answer, but the expected format is to only give one. 
	  } 
	   } 
	   } 
	 & & \\ 
 

    \theutterance \stepcounter{utterance}  

    & & & \multicolumn{2}{p{0.3\linewidth}}{\cellcolor[rgb]{0.95,0.95,0.95}{%
	\makecell[{{p{\linewidth}}}]{% 
	  \tt {\tiny [GM$|$GM]}  
	 game_result = ABORT 
	  } 
	   } 
	   } 
	 & & \\ 
 

\end{supertabular}
}

\end{document}
