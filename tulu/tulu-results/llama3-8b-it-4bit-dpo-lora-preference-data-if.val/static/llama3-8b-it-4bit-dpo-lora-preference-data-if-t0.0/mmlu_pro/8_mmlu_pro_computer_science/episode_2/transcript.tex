
\documentclass{article}
\usepackage{colortbl}
\usepackage{makecell}
\usepackage{multirow}
\usepackage{supertabular}

\begin{document}

\newcounter{utterance}

\twocolumn

{ \footnotesize  \setcounter{utterance}{1}
\setlength{\tabcolsep}{0pt}
\begin{supertabular}{c@{$\;$}|p{.15\linewidth}@{}p{.15\linewidth}p{.15\linewidth}p{.15\linewidth}p{.15\linewidth}p{.15\linewidth}}

    \# & $\;$A & \multicolumn{4}{c}{Game Master} & $\;\:$B\\
    \hline 

    \theutterance \stepcounter{utterance}  

    & & \multicolumn{4}{p{0.6\linewidth}}{\cellcolor[rgb]{0.9,0.9,0.9}{%
	\makecell[{{p{\linewidth}}}]{% 
	  \tt {\tiny [A$\langle$GM]}  
	 The following are multiple choice questions (with answers) about computer science. Answer only with "The answer is (X)" where X is the correct letter choice.\\ \tt \\ \tt Question:\\ \tt A certain pipelined RISC machine has 8 general-purpose registers R0, R1, . . . , R7 and supports the following operations:\\ \tt ADD Rs1, Rs2, Rd (Add Rs1 to Rs2 and put the sum in Rd)\\ \tt MUL Rs1, Rs2, Rd (Multiply Rs1 by Rs2 and put the product in Rd)\\ \tt An operation normally takes one cycle; however, an operation takes two cycles if it produces a result required by the immediately following operation in an operation sequence.\\ \tt Consider the expression AB + ABC + BC, where variables A, B, C are located in registers R0, R1, R2. If the contents of these three registers must not be modified, what is the minimum number of clock cycles required for an operation sequence that computes the value of AB + ABC + BC?\\ \tt Options:\\ \tt A. 10\\ \tt B. 6\\ \tt C. 9\\ \tt D. 5\\ \tt E. 12\\ \tt F. 4\\ \tt G. 7\\ \tt H. 8\\ \tt I. N/A\\ \tt J. N/A\\ \tt The answer is (B)\\ \tt \\ \tt Question:\\ \tt A compiler generates code for the following assignment statement.\\ \tt G := (A + B) * C - (D + E) * F\\ \tt The target machine has a single accumulator and a single-address instruction set consisting of instructions load, store, add, subtract, and multiply. For the arithmetic operations, the left operand is taken from the accumulator and the result appears in the accumulator. The smallest possible number of instructions in the resulting code is\\ \tt Options:\\ \tt A. 7\\ \tt B. 5\\ \tt C. 12\\ \tt D. 4\\ \tt E. 11\\ \tt F. 6\\ \tt G. 8\\ \tt H. 10\\ \tt I. 9\\ \tt J. 3\\ \tt The answer is (I)\\ \tt \\ \tt Question:\\ \tt SHA-1 has a message digest of\\ \tt Options:\\ \tt A. 628 bits\\ \tt B. 780 bits\\ \tt C. 160 bits\\ \tt D. 256 bits\\ \tt E. 2048 bits\\ \tt F. 820 bits\\ \tt G. 128 bits\\ \tt H. 512 bits\\ \tt I. 1024 bits\\ \tt J. 64 bits\\ \tt The answer is (C)\\ \tt \\ \tt Question:\\ \tt _____________ can modify data on your system – so that your system doesn’t run correctly or you can no longer access specific data, or it may even ask for ransom in order to give your access.\\ \tt Options:\\ \tt A. Email Trojans\\ \tt B. Backdoor Trojans\\ \tt C. Rootkit Trojans\\ \tt D. Banking Trojans\\ \tt E. Trojan-Downloader\\ \tt F. Spyware Trojans\\ \tt G. Network Trojans\\ \tt H. Worm Trojans\\ \tt I. Ransom Trojan\\ \tt J. IM – Trojans\\ \tt The answer is (I)\\ \tt \\ \tt Question:\\ \tt In the program below, the initial value of X is 5 and the initial value of Y is 10.\\ \tt IF (X < 0){\\ \tt  DISPLAY ("Foxtrot")\\ \tt } ELSE {\\ \tt  IF (X > Y){\\ \tt   DISPLAY ("Hotel")\\ \tt  } ELSE {\\ \tt   IF (Y > 0){\\ \tt    DISPLAY ("November")\\ \tt   } ELSE {\\ \tt    DISPLAY ("Yankee")\\ \tt   }\\ \tt  }\\ \tt }\\ \tt What is displayed as a result of running the program?\\ \tt Options:\\ \tt A. November\\ \tt B. Foxtrot\\ \tt C. Zulu\\ \tt D. Alpha\\ \tt E. Charlie\\ \tt F. Bravo\\ \tt G. Yankee\\ \tt H. Echo\\ \tt I. Hotel\\ \tt J. Delta\\ \tt The answer is (A)\\ \tt \\ \tt Question:\\ \tt Explain the TRAILER CARD method of stopping a computer job, i.e., a program.\\ \tt Options:\\ \tt A. The Trailer Card method uses a serial number on the first card to stop a program.\\ \tt B. The Trailer Card method uses a specific phrase typed into the console to signal the end of a job.\\ \tt C. The Trailer Card method requires a specific card to be inserted physically into the computer.\\ \tt D. The Trailer Card method involves sending an electronic signal through a network to stop the program.\\ \tt E. The Trailer Card method employs a magnetic tape marker to indicate the end of a program.\\ \tt F. The Trailer Card method uses a unique number on the last card in the data section of a program card deck to signal the end of a job.\\ \tt G. The Trailer Card method utilizes a timer that automatically stops the program after a set duration.\\ \tt H. The Trailer Card method involves pressing a designated key on the computer's keyboard to terminate the job.\\ \tt I. The Trailer Card method requires the operator to manually switch off the computer to end the job.\\ \tt J. The Trailer Card method uses a special code to pause a program.\\ \tt  
	  } 
	   } 
	   } 
	 & \\ 
 

    \theutterance \stepcounter{utterance}  

    & \multicolumn{4}{p{0.6\linewidth}}{\cellcolor[rgb]{0.8,1,0.9}{%
	\makecell[{{p{\linewidth}}}]{% 
	  \tt {\tiny [A$\rangle$GM]}  
	 The answer is (F) 
	  } 
	   } 
	   } 
	 & & \\ 
 

    \theutterance \stepcounter{utterance}  

    & & & \multicolumn{2}{p{0.3\linewidth}}{\cellcolor[rgb]{0.95,0.95,0.95}{%
	\makecell[{{p{\linewidth}}}]{% 
	  \tt {\tiny [GM$|$GM]}  
	 F 
	  } 
	   } 
	   } 
	 & & \\ 
 

    \theutterance \stepcounter{utterance}  

    & & & \multicolumn{2}{p{0.3\linewidth}}{\cellcolor[rgb]{0.95,0.95,0.95}{%
	\makecell[{{p{\linewidth}}}]{% 
	  \tt {\tiny [GM$|$GM]}  
	 F 
	  } 
	   } 
	   } 
	 & & \\ 
 

    \theutterance \stepcounter{utterance}  

    & & & \multicolumn{2}{p{0.3\linewidth}}{\cellcolor[rgb]{0.95,0.95,0.95}{%
	\makecell[{{p{\linewidth}}}]{% 
	  \tt {\tiny [GM$|$GM]}  
	 game_result = WIN 
	  } 
	   } 
	   } 
	 & & \\ 
 

\end{supertabular}
}

\end{document}
