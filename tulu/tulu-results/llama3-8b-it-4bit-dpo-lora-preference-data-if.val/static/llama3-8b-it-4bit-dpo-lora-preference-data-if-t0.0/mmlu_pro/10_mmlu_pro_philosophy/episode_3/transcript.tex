
\documentclass{article}
\usepackage{colortbl}
\usepackage{makecell}
\usepackage{multirow}
\usepackage{supertabular}

\begin{document}

\newcounter{utterance}

\twocolumn

{ \footnotesize  \setcounter{utterance}{1}
\setlength{\tabcolsep}{0pt}
\begin{supertabular}{c@{$\;$}|p{.15\linewidth}@{}p{.15\linewidth}p{.15\linewidth}p{.15\linewidth}p{.15\linewidth}p{.15\linewidth}}

    \# & $\;$A & \multicolumn{4}{c}{Game Master} & $\;\:$B\\
    \hline 

    \theutterance \stepcounter{utterance}  

    & & \multicolumn{4}{p{0.6\linewidth}}{\cellcolor[rgb]{0.9,0.9,0.9}{%
	\makecell[{{p{\linewidth}}}]{% 
	  \tt {\tiny [A$\langle$GM]}  
	 The following are multiple choice questions (with answers) about philosophy. Answer only with "The answer is (X)" where X is the correct letter choice.\\ \tt \\ \tt Question:\\ \tt Which of the given formulas of PL is the best symbolization of the following sentence?\\ \tt Turtles live long lives and are happy creatures, unless they are injured.\\ \tt Options:\\ \tt A. (L • H) ≡ I\\ \tt B. (L • H) ∨ I\\ \tt C. L • (H ∨ I)\\ \tt D. L • (H ⊃ R).\\ \tt E. N/A\\ \tt F. N/A\\ \tt G. N/A\\ \tt H. N/A\\ \tt I. N/A\\ \tt J. N/A\\ \tt The answer is (B)\\ \tt \\ \tt Question:\\ \tt Select the best translation into predicate logic.George borrows Hector's lawnmower. (g: George; h: Hector; l: Hector's lawnmower; Bxyx: x borrows y from z).\\ \tt Options:\\ \tt A. Bhgh\\ \tt B. Bggh\\ \tt C. Bhlh\\ \tt D. Bghl\\ \tt E. Bhlg\\ \tt F. Blhg\\ \tt G. Bllg\\ \tt H. Blgh\\ \tt I. Bhhg\\ \tt J. Bglh\\ \tt The answer is (J)\\ \tt \\ \tt Question:\\ \tt The study of reality in the broadest sense, an inquiry into the elemental nature of the universe and the things in it, is known as _____.\\ \tt Options:\\ \tt A. quantum physics\\ \tt B. metaphysics\\ \tt C. ontology\\ \tt D. phenomenology\\ \tt E. cosmology\\ \tt F. teleology\\ \tt G. epistemology\\ \tt H. axiology\\ \tt I. N/A\\ \tt J. N/A\\ \tt The answer is (B)\\ \tt \\ \tt Question:\\ \tt According to Moore’s “ideal utilitarianism,” the right action is the one that brings about the greatest amount of:\\ \tt Options:\\ \tt A. wealth.\\ \tt B. virtue.\\ \tt C. fairness.\\ \tt D. pleasure.\\ \tt E. peace.\\ \tt F. justice.\\ \tt G. happiness.\\ \tt H. power.\\ \tt I. good.\\ \tt J. knowledge.\\ \tt The answer is (I)\\ \tt \\ \tt Question:\\ \tt What is the sign of the covenant for Jewish males?\\ \tt Options:\\ \tt A. Fasting on Yom Kippur\\ \tt B. Lighting Shabbat candles\\ \tt C. The rainbow\\ \tt D. Circumcision\\ \tt E. The Torah\\ \tt F. Bar mitzvah\\ \tt G. Keeping kosher\\ \tt H. Wearing a kippah\\ \tt I. A son\\ \tt J. The Star of David\\ \tt The answer is (D)\\ \tt \\ \tt Question:\\ \tt John S. Knight, Detroit Free Press (3 July 1970): ...Capitalism is the economic system in which the means of production and distribution are privately owned and operated for profit. The alternative system is called socialism, the theory of the ownership and operation of the means of production by society, with all members sharing in the work and the products.\\ \tt Options:\\ \tt A. Begging the Question\\ \tt B. False Dilemma\\ \tt C. Straw Man\\ \tt D. Appeal to Gallery (Provicialism)\\ \tt  
	  } 
	   } 
	   } 
	 & \\ 
 

    \theutterance \stepcounter{utterance}  

    & \multicolumn{4}{p{0.6\linewidth}}{\cellcolor[rgb]{0.8,1,0.9}{%
	\makecell[{{p{\linewidth}}}]{% 
	  \tt {\tiny [A$\rangle$GM]}  
	 The answer is (D) 
	  } 
	   } 
	   } 
	 & & \\ 
 

    \theutterance \stepcounter{utterance}  

    & & & \multicolumn{2}{p{0.3\linewidth}}{\cellcolor[rgb]{0.95,0.95,0.95}{%
	\makecell[{{p{\linewidth}}}]{% 
	  \tt {\tiny [GM$|$GM]}  
	 D 
	  } 
	   } 
	   } 
	 & & \\ 
 

    \theutterance \stepcounter{utterance}  

    & & & \multicolumn{2}{p{0.3\linewidth}}{\cellcolor[rgb]{0.95,0.95,0.95}{%
	\makecell[{{p{\linewidth}}}]{% 
	  \tt {\tiny [GM$|$GM]}  
	 B 
	  } 
	   } 
	   } 
	 & & \\ 
 

    \theutterance \stepcounter{utterance}  

    & & & \multicolumn{2}{p{0.3\linewidth}}{\cellcolor[rgb]{0.95,0.95,0.95}{%
	\makecell[{{p{\linewidth}}}]{% 
	  \tt {\tiny [GM$|$GM]}  
	 game_result = LOSE 
	  } 
	   } 
	   } 
	 & & \\ 
 

\end{supertabular}
}

\end{document}
